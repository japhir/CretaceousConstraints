%%%%%%%%%%%%%%%%%%%%%%%%%%%%%%%%%%%%%%%%%%%%%%%%%%%%%%%%%%%%%%%%%%%%%%%%%%%%
% adapted from AGUJournalTemplate.tex: this template file is for articles formatted with LaTeX
%
%% To submit your paper:
\documentclass[draft]{agujournal2019}
% packages from the template
\usepackage{url} %this package should fix any errors with URLs in refs.
\usepackage{lineno}
\usepackage[inline]{trackchanges} %for better track changes. finalnew option will compile document with changes incorporated.
\usepackage{soul}

% IJK added packages
% for units, type \qty{x}{\unit}
\usepackage{siunitx}
% for chemical equations or, in my case, d13C
\usepackage[version=4]{mhchem}
%\usepackage{chemformula} % alternative for above
% \usepackage[utf8]{luainputenc} % allow UTF-8 input? δ13C?
% \usepackage{hyperref} % links between references etc % doesn't seem to work
% with this template?
\usepackage[capitalise,nameinlink,noabbrev]{cleveref}
% use same macro for acronym, this writes out the first use and then
% abbreviates after (use \gls{key}).
\usepackage{glossaries}
%\makeglossaries % not needed if we don't want to print them at the end
\usepackage[section]{placeins} % figures were jumping to the end, I prefer them closer

\newcommand{\appr}{\raise.17ex\hbox{$\scriptstyle\sim$}} % approximately symbol

\newcommand{\rez}{\textcolor{magenta}}
% I'll use this to highlight sections that I want to focus your attention on!
% I liked violet best, but might be hard to distinguish quickly from magenta...
\newcommand{\ijk}{\textcolor{blue}}


% normally I strongly prefer biblatex, but the AGU template uses only \cite,
% \citeA, and \nocite from apacite :(
% \usepackage[giveninits=true,uniquename=false,uniquelist=false,date=year,hyperref=true,mincitenames=1,maxcitenames=2,backend=biber,backref,doi=true,url=false,isbn=false]{biblatex}
% \addbibresource{references.bib}

% IJK package config
\sisetup{%
detect-all,% Detect surrounding font context, like weight, italics etc.
%
% Alternative range-phrase:
% en-dash via '--', but inside \text{}, so it's not 'two minus signs'
range-phrase={\,\text{--}\,},
separate-uncertainty=true,
multi-part-units=single,
list-units=single,% single: Print unit only once, at end
range-units=single,% single: Print unit only once, at end
per-mode=symbol,
}%
\DeclareSIUnit\annum{a} % a year, used in years before present
\DeclareSIUnit\year{yr} % a year as a duration
\DeclareSIUnit\millionyearago{\mega\annum} % a time, e.g. 34 million years before present
\DeclareSIUnit\millionyear{\mega\year} % a duration, e.g. the event lasted for 2 million years
\DeclareSIUnit\kiloyearago{\kilo\annum} % a time, e.g. 14 thousand years ago (before present)
\DeclareSIUnit\kiloyear{\kilo\year} % a duration, e.g. the event lasted 200 thousand years

% this makes it so the first use spells it out, the next uses the abbreviation!
\newacronym{d13C}{\ensuremath{\delta}\ce{^13C}}{carbon isotope ratio}
\newacronym{MS}{MS}{magnetic susceptibility}
\newacronym{L*}{L*}{total light reflectance}

\newacronym{RMSD}{RMSD}{root mean square deviation}
% do we need RCSSD?

\newacronym{KT}{K/T}{Cretaceous--Paleogene}
\newacronym{PETM}{PETM}{Paleocene--Eocene thermal maximum}

\newacronym{GAM}{GAM}{generalized additive model}
\newacronym{FFT}{FFT}{fast Fourier transform}
\newacronym{MTM}{MTM}{multi-taper method}
\newacronym{MTLS}{MTLS}{multi-taper averaged Lomb-Scargle periodogram of (un)evenly spaced data}

\linenumbers
%%%%%%%
% As of 2018 we recommend use of the TrackChanges package to mark revisions.
% The trackchanges package adds five new LaTeX commands:
%
%  \note[editor]{The note}
%  \annote[editor]{Text to annotate}{The note}
%  \add[editor]{Text to add}
%  \remove[editor]{Text to remove}
%  \change[editor]{Text to remove}{Text to add}
%
% complete documentation is here: http://trackchanges.sourceforge.net/
%%%%%%%

% \draftfalse

%% Enter journal name below.
%% Choose from this list of Journals:
%
% JGR: Atmospheres
% JGR: Biogeosciences
% JGR: Earth Surface
% JGR: Oceans
% JGR: Planets
% JGR: Solid Earth
% JGR: Space Physics
% Global Biogeochemical Cycles
% Geophysical Research Letters
% Paleoceanography and Paleoclimatology
% Radio Science
% Reviews of Geophysics
% Tectonics
% Space Weather
% Water Resources Research
% Geochemistry, Geophysics, Geosystems
% Journal of Advances in Modeling Earth Systems (JAMES)
% Earth's Future
% Earth and Space Science
% Geohealth
%
% ie, \journalname{Water Resources Research}

\journalname{Paleoceanography and Paleoclimatology}


\begin{document}

%% ------------------------------------------------------------------------ %%
%  Title
%
% (A title should be specific, informative, and brief. Use
% abbreviations only if they are defined in the abstract. Titles that
% start with general keywords then specific terms are optimized in
% searches)
%
%% ------------------------------------------------------------------------ %%

% previously: Cretaceous Constraints on Astronomical Solutions
\title{An astronomically tuned time scale for the latest Cretaceous}
% technically, Maastrichtian would be more appropriate but that doesn't alliterate ;-)

%% ------------------------------------------------------------------------ %%
%
%  AUTHORS AND AFFILIATIONS
%
%% ------------------------------------------------------------------------ %%

% List authors by first name or initial followed by last name and
% separated by commas. Use \affil{} to number affiliations, and
% \thanks{} for author notes.
% Additional author notes should be indicated with \thanks{} (for
% example, for current addresses).

% Example: \authors{A. B. Author\affil{1}\thanks{Current address, Antartica}, B. C. Author\affil{2,3}, and D. E.
% Author\affil{3,4}\thanks{Also funded by Monsanto.}}

\authors{I. J. Kocken\affil{1} and R. E. Zeebe\affil{1}}
\affiliation{1}{
School of Ocean and Earth Science and Technology,
University of Hawaii at Manoa,
1000 Pope Road, MSB 629, Honolulu, HI 96822, USA}

% \affiliation{2}{Second Affiliation}
%(repeat as many times as is necessary)

%% Corresponding Author:
% Corresponding author mailing address and e-mail address:

% (include name and email addresses of the corresponding author.  More
% than one corresponding author is allowed in this LaTeX file and for
% publication; but only one corresponding author is allowed in our
% editorial system.)

\correspondingauthor{Ilja Kocken}{ikocken@hawaii.edu}

%% Keypoints, final entry on title page.

%  List up to three key points (at least one is required)
%  Key Points summarize the main points and conclusions of the article
%  Each must be 140 characters or fewer with no special characters or punctuation and must be complete sentences

% Example:
% \begin{keypoints}
% \item	List up to three key points (at least one is required)
% \item	Key Points summarize the main points and conclusions of the article
% \item	Each must be 140 characters or fewer with no special characters or punctuation and must be complete sentences
% \end{keypoints}

\begin{keypoints}
\item enter point 1 here
\item enter point 2 here
\item enter point 3 here
\end{keypoints}

%% ------------------------------------------------------------------------ %%
%
%  ABSTRACT and PLAIN LANGUAGE SUMMARY
%
% A good Abstract will begin with a short description of the problem
% being addressed, briefly describe the new data or analyses, then
% briefly states the main conclusion(s) and how they are supported and
% uncertainties.

% The Plain Language Summary should be written for a broad audience,
% including journalists and the science-interested public, that will not have
% a background in your field.
%
% A Plain Language Summary is required in GRL, JGR: Planets, JGR: Biogeosciences,
% JGR: Oceans, G-Cubed, Reviews of Geophysics, and JAMES.
% see http://sharingscience.agu.org/creating-plain-language-summary/)
%
%% ------------------------------------------------------------------------ %%

%% \begin{abstract} starts the second page

\begin{abstract}
[ enter your Abstract here ]
\end{abstract}

\section*{Plain Language Summary}
[ enter your Plain Language Summary here or delete this section]


%%% Suggested section heads:
% \section{Introduction}
%
% The main text should start with an introduction. Except for short
% manuscripts (such as comments and replies), the text should be divided
% into sections, each with its own heading.

% Headings should be sentence fragments and do not begin with a
% lowercase letter or number. Examples of good headings are:

% \section{Materials and Methods}
% Here is text on Materials and Methods.
%
% \subsection{A descriptive heading about methods}
% More about Methods.
%
% \section{Data} (Or section title might be a descriptive heading about data)
%
% \section{Results} (Or section title might be a descriptive heading about the
% results)
%
% \section{Conclusions}

%%% IJK some notes on how I write special symbols and units here:

% how do I want to write d13C?
% \(\delta^{13}\)C % plain
% \ce{\delta^13C} % mhchem
% \(\delta\)\ch{^{13}C} % chemformula
% \gls{d13C} % using glossaries and mhchem  <-----

% test my acronyms
% \Gls{MS}
% \Gls{d13C}
% \Gls{L*}
% \Gls{KT} boundary

% test my units. Define them the first time we use them too.
% \qty{5}{\kiloyear}
% \qtylist{2;3;5}{\kiloyear}
% \qtyrange{2}{35}{\kiloyear}
% \qty{66}{\millionyearago}


\section{Introduction}\label{sec:intro}

\ijk{I've focused on the figures, results, and discussion first!}

\ijk{I like how some of RZ's paper's start with an ancient quote}

``Cool quote that addresses how crucial good dating is''.
Maybe something from \citeA{Berger1980} review? Or perhaps \citeA{Berger1989}.


As we move to older and older time-periods, the available records will get further distorted and \ijk{TODO}.

\ijk{@REZ: I just added \cite{Westerhold2017} here. They make the comparison between data and several solutions and comment on implications, but it's not so explicit.}

In this work, we use geological data to constrain astronomical solutions that most accurately reflect the orbit of the Earth.
Ultimately, this will result in an astronomical time scale for applications in geology and paleoclimate.
For previous on the Paleocene and Eocene, see \citeA{Westerhold2017,ZeebeLourens2019,ZeebeLourens2022EPSL}.
The Walvis Ridge records are exceptional in their quality, so finding comparable records that capture precession-scale alternations and patterns of amplitude modulation for older time periods is difficult at this time.

Here we use the best available records immediately prior to the \gls{KT} boundary (informal use)
for the Maastrichtian (data falls between \qtyrange{71.1}{65.9}{\millionyearago}).
We use the Zumaia and Sopelana terrestrial composite records, as generated by \citeA{Batenburg2012,Batenburg2014}.

\ijk{Something to introduce long eccentricity (\qty{405}{\kiloyear} period) and short eccentricity (\qty{100}{\kiloyear} period).}

\ijk{Also introduce very long eccentricity nodes!}


\section{Material and Methods}\label{sec:mm}

\ijk{\textbf{TODO:} Decide on consistent naming for scaled vs.\ normalized vs.\ standardized: subtract mean, divide by standard deviation. I like scaled?}

\subsection{Datasets Used}\label{sec:data}

In order to find the most suitable Maastrichtian proxy record that best reflects amplitude variations in long- and short eccentricity,
we have considered several continental sections and sea floor drill core sites.
We have looked at records from
Gubbio~\cite{Voigt2012,Sinnesael2016},
ODP Leg 208 Site 1267~\cite{Westerhold2007,Westerhold2008,Husson2011},
IODP Leg 342 Site U1403~\cite{Batenburg2018},
Zumaia and Sopelana~\cite{tenKateSprenger1993,Batenburg2012,Batenburg2014},
ODP Leg 198 Site 1210B~\cite{Jung2012,Kim2022},
ODP Leg 74 Site 525A~\cite{Husson2011}
ODP Leg 122 Site 762C~\citeA{Husson2011,Thibault2012},
and ODP Leg 207 Site 1258A~\citeA{Husson2011}.
For the present study, we selected the Zumaia and Sopelana field sections,
which show the clearest orbitally-forced patterns for the Maastrichtian,
with continuous high-resolution records and an established astronomically forced hierarchy of bundling patterns~\cite{Batenburg2012,Batenburg2014}.

The coastal cliffs of Zumaia and Sopelana are located in the Bay of Biscay in the Basque Country, Northern Spain~\cite{Herm1965,Pujalte1993}.
%\ijk{NOTE: I tried to go through the literature trail thoroughly to arrive at the first study on this site, but most of these older papers are very hard to access!}
and comprise quasi-periodic alternations between limestones and marls~\cite{tenKateSprenger1993,Pujalte1999}. % this is cited as Pujalte 1998 in Batenburg2012, year off-by-one!
See \citeA{Batenburg2014} and references therein for the historical context of the study sites. % this has better refs than their 2012 paper I think.
% quote from Batenburg2014: (Herm 1965; Percival & Fischer 1977; Lamolda 1990; Ward et al. 1991; Ten Kate & Sprenger 1993; Ward & Kennedy, 1993; elorza & García-Garmilla 1998; Pujalte et al. 1998; Dinarès-Turell et al. 2003, 2013; Gómez-Alday et al. 2008; Kuiper et al. 2008) and a reference section for the K–Pg boundary (Molina et al. 2009).
% they also say ten Kate & Sprenger 1993 first cyclostrat paper of the site.

Care must be taken when considering the lower Maastrichtian at the Zumaia site due to frequent turbidite successions,
while the upper Maastrichtian was tectonically more stable~\cite{Pujalte1998}.
Therefore, \citeA{Batenburg2014} expanded their Maastrichtian Zumaia record~\cite{Batenburg2012} with the Sopelana site, which shows no turbidites.
\rez{[let's make sure the following text is not too close to
the original refs]}
\ijk{I've looked at the original refs again. It was obviously never copy-pasted, but looking at it again it was mostly reshuffled text. I am not sure how close I can be to the original text that I cite/if I need to incorporate other stuff...}
The upper Maastrichtian section of Zumaia still contains many thin turbidites,
whose occurrence shows many random spectral peaks but may also include an eccentricity and obliquity signal for cycles Ma\(_{405}1--2\)~\cite<the first two long eccentricity bundles prior to the \gls{KT}-boundary>{tenKateSprenger1993}. % this part is not mentioned in Batenburg
However, the pattern of limestone--marl alternations in the Zumaia succession that has been associated with astronomical forcing has not been disrupted by the turbidites in the upper Maastrichtian~\cite{Batenburg2012}.
The alternations in color and bed resistance at smallest-scale of \appr\qty{70}{\cm} were logged in detail~\cite{Batenburg2012}
and have been associated with climatic precession forcing (\appr\qty{21}{\kiloyear}).
Bundles of 5 of these couplets with darker, redder, thicker beds, especially in the marls, were related to short eccentricity amplitude modulation (\appr\qty{4}{\metre}, \appr\qty{100}{\kiloyear}),
which in turn were grouped in 4 bundles that were associated with the long eccentricity cycle (\appr\qty{16}{\metre}, \appr\qty{405}{\kiloyear}) and expressed as limited thinner marlier parts~\cite{Batenburg2012,Batenburg2014}.
Eccentricity modulates the amplitude of climatic precession, with eccentricity maxima and strong precession minima linked to enhanced seasonal contrast in the Northern hemisphere.
This intensifies the hydrological cycle in the region and results in the deposition of thicker, darker marls~\cite{Batenburg2014}.
The Sopelana section covers the lower part of the upper Maastrichtian, and shows similar lithology but has no indications for turbidites, likely because it was located farther offshore~\cite{Batenburg2012}.

\citeA{Batenburg2012,Batenburg2014} measured the \gls{MS}, the \gls{d13C}, and the \gls{L*} of rock samples (among others), but we focus on \gls{L*} in the main manuscript.
The \gls{d13C} proxy shows long eccentricity forcing clearly but the short eccentricity is less clearly expressed,
which follows from carbon isotopes being more sensitive to lower frequencies in orbital forcing due to its long residence time~\cite{Zeebe2017,Kocken2019loscar}.
It is unclear how \gls{MS} relates to the climate forcing and the original study reports that the values show scatter,
potentially due to ``surface irregularities and the generally low values''~\cite{Batenburg2012}. % Richard doesn't trust/understand MS.
The \gls{L*} proxy, on the other hand, likely corresponds to changes in the lithology that have been driven by carbonate--clay alternations~\cite{MountWard1986,Batenburg2012},
which are in turn driven by changes in wetness and climate.
% I wanted to add something like the below:
% Furthermore, the \gls{L*} proxy shows similar patterns to \ce{CaCO3}-data for the uppormost Maastrichtian~\cite{tenKateSprenger1993}.
% BUT it seems that this record has a _--_ pattern for the amplitudes of the short ecc in the first 4 long ecc cycles prior to the K/T
We therefore argue that the color reflectance record most directly reflects the orbital forcing.


% commented out now, just put the relevant refs up above!
% \ijk{Just went through the table again, 1267 still looks promising and so does U1403.}
% \begin{table}
%     \centering
%     \caption{\label{tab:sites}
%         Overview of sites considered for study, sorted from younger to older, shorter to longer.
%         Gubbio consists of Contessa Highway (Danian) and Bottaccione (Maastrichtian).
%         \ijk{Rough first draft of table! See \TeX{} file for notes in comments.}
%     }
%     \begin{tabular}{llccl}
%         Site & Proxy & Ma\textsubscript{405}x & Age (Ma) & Reference \\
%         % the ages and Ma405 cycles are pretty rough estimates from figures/raw data
%         \hline
%         %208-1262 & a*, d13C, d18O & - & 58--53 & \citeA{ZeebeLourens2019} \\
%         % should probably cite original study but won't show in final table here anyway
%         %198-1209 & a*/b* & - & 66--56 Ma & \citeA{ZeebeLourens2022EPSL}\\
%         %Hendaye & section photos & - & 66--64 & Hilgen, pers. comm.\\
%         % Frits says it has a 200 kyr cycle that is unique to this interval!
%         Gubbio & \gls{d13C}, MS & 1--12 & 76--66 & \citeA{Voigt2012}\\
%         % too low-res? Contrasts to La10d
%         % Gubbio & Ca isotopes & - & OAE-2 & \citeA{Kitch2022}\\ % low-res, old, wrong proxy
%         Gubbio & MS, \ce{CaCO3}, d13C, d18O & - & 67--62.5 & \citeA{Sinnesael2016} \\
%         % bottaccione and contessa highway
%         % short, but Zumaia doesn't have much signal here
%         % may be worth another look! Contrasts to La11
%         208-1267 & color & 1--6/7 & 69.1--56.0 & \citeA{Westerhold2007} \\
%         % a* in \citeA{Westerhold2007}, mostly focused on PETM--Elmo
%         % extended in \cite{Batenburg2018} (56.042 Ma to 69.070 Ma). --> can no longer find this! :O
%         % also related: [cite:@Zachos2004]
%         208-1267 & ln(Fe) & 1--6/7 & 69.1--56.0 & \citeA{Westerhold2008} \\ % very very strong short cycle
%         208-1267 & MS & 1--6/7 & 68.6--66 & \citeA{Husson2011} fig. 3 and 4. \\
%         % raw MS data from Blum2005 https://doi.pangaea.de/10.1594/PANGAEA.266605
%         % Husson2011 uses MBSF but need to use RMCD -> core gaps etc.!
%         % splice table from \cite{Westerhold2008} PDF https://doi.pangaea.de/10.1594/PANGAEA.592301
%         % I spent some time reworking this!
%         %% 208-1267 & \ce{CaCO3} & - & 50-47.8 & \citeA{Sexton2011}\\ % too young
%         342-U1403 & MS, ln(Fe/Ca) & 1--7 & 68.8--66 & \citeA{Batenburg2018}\\
%         % not sure why I didn't pick the one above
%         Zumaia & \ce{CaCO3} & \(-\)3--3 & 67.5--65.5 & \citeA{tenKateSprenger1993}\\ % too short?
%         % also shown in \cite{Westerhold2008}, next to 1267
%         Zumaia & L*, MS, d13C & 1--9 & 70--66 & \citeA{Batenburg2012} \\
%         Sopelana & L*, MS & 9--13 & 71.1--69.6 & \citeA{Batenburg2014}\\
%         198-1210B & d13C, d18O, XRF Ba & 2--12 & 71.5--66.25 & \citeA{Jung2012,Kim2022}\\
%         % some low-res intervals in crucial parts?
%         % higher res around 68 Ma, Kim2022 pretty high res but only d13C
%         74-525A & grayscale & 6--8.5 & 69--67.8 & \citeA{Husson2011}\\
%         % have emailed requesting data, but has left academia
%         122-762C & grayscale & 8--17+ & 77.5--69 & \citeA{Husson2011,Thibault2012}\\
%         % Thibault has published most data and sent me some upon request!
%         % the above may be useful to look at:
%         % some core gaps but high amplitude short ecc in Ma405_8 and Ma405_10!
%         207-1258A & MS & 8--14 & - & \citeA{Husson2011}\\
%         % raw data not available, but can reproduce by getting data from JANUS
%         % database
%         % taner filters not so nice, some big core gaps
%         % Gubbio & d13C, d18O & - & 84.2--72.1 & \citeA{Voigt2012}\\
%         % Gubbio & d13C & - & 84.2--72.1 & \citeA{Sabatino2018}\\ % high-res
%         % Furlo & color, d13C, d18O & - & 96--90 & \citeA{Batenburg2016}\\ % too old for us, but cool for future?
%         % Levant Platform & TOC, d13C, d18O & - & 96.2--90.9 & \citeA{Wendler2014}\\
%     \end{tabular}
% \end{table}

\subsection{Age Model}\label{sec:agemodel}

\ijk{re: past/present tense: actually I'm used to past tense for methods, but I guess I wasn't being very consistent. Rewrote everything to present tense now. Maybe rewrite again later?}

The age model for the Zumaia and Sopelana sites is based on the identification of long eccentricity minima in the field \citeA{Batenburg2012,Batenburg2014}.
The original study relied on foraminifera biostratigraphy, magnetostratigraphy, and astrochronology by tuning these long eccentricity minima to the long eccentricity filter of the La11 solution~\cite{Laskar2011a}. % not sure if sentence is needed
Here we assume durations of \qty{405}{\kiloyear} for each of these long eccentricity cycles to arrive at an initial floating age model.
A previous best-estimate of the age of the \gls{KT} boundary \cite<\qty{65.9}{\millionyearago},>[]{ZeebeLourens2022EPSL} anchors the floating age model to absolute time.
We then perform optimizations for each astronomical solution that allow the age of the \gls{KT} boundary to shift by \qty{\pm200}{\kiloyear}
and to change the depth for each of the long eccentricity minima tie-points (see \cref{sec:algorithm}).

\subsection{Removing Long-Term Trends}\label{sec:detrend}

There are several rapid shifts in the lithology---from red, clay-rich marly intervals to whiter limestones---that occur in the Zumaia section on the order of every \qty{50}{\metre}.
This has previously been suggested to potentially indicate a \qty{1.2}{\millionyear} cycle~\citeA{Batenburg2014}.
These shifts in lithology are recorded in the proxy archives of Zumaia as very rapid transitions, likely caused by crossing some threshold value in a nonlinear climate system response.
This could hamper spectral analysis and filtering, and therefore we use several methods of detrending the records:
(1) We subtract a linear fit and scale (subtract the mean and divide by the standard deviation) the result;
(2) We fit a \gls{GAM} to the record, subtract the fit from each value, then scale the results;
(3) We apply a lowpass filter to the record with a frequency of \qty{0.025}{\metre\per cycle}, subtract it from each value, then scale the results;
and (4) We apply a piecewise linear fit based on depth intervals where the lithology changes dramatically, or we observe long-term trends that are at a larger scale than long eccentricity.
We subtract the linear fits from each value and scale the results.
We have tried many ways of detrending the records, for example with larger and smaller chunks in the depth domain (\cref{fig:detrend}), or with different frequencies for the lowpass filter, and report sensitivity to this in the appendix.
% The effects that these trends can have on our results were limited because we filter only in the time-domain, after applying the age model. % Not sure if true, commented out for now.
%TODO: add link to new appendix figs

\rez{you could pad more 5x, 10x, any difference?}
\ijk{I didn't read the docs correctly, \texttt{padfac} defaults to 5.
    I quickly tried out 5, 10, 15, 100 times and no diff for 405 and 100 kyr filters.
    Also tried different padfacs for taner, it just becomes way slower but other than that it's nearly identical once you have over 3x the data.}

\subsection{Finding the Best Fit}\label{sec:algorithm}

In order to test which astronomical solution provides the best match to the data
(and hence is more likely to reflect the true astronomical time scale)
we adopt and extend the approach of \citeA{ZeebeLourens2019,ZeebeLourens2022EPSL}.
The detrended data (\cref{sec:detrend}) is combined with the initial age model (\cref{sec:agemodel}) to obtain data in the time domain.
After linearly interpolating to a timestep that is a multiple of the timestep of the astronomical solution, we filter targeting the long and short eccentricity cycles.
We use a taner filter with a roll of \num{e10} targeting periods of \qtylist{405;110}{\kiloyear} (frequency \qty{\pm25}{\percent}) (\cref{fig:filter-windows}).
We chose these parameters after analyzing the sensitivity to different ways of filtering (rectangular, gaussian, different parameters for the taner filter in \cref{fig:full-RMSD-filter,fig:filter-windows}).
The filtered signals are added and scaled to construct an artificial ``eccentricity'' curve.
The scaled astronomical solutions are then interpolated/subset to the same timesteps as the data.
Then we calculate how well the filtered records match the astronomical solution via the \gls{RMSD}:
\begin{equation}\label{eqn:rmsd}
    \text{RMSD} = \sqrt{\frac{1}{n}\sum_{i=1}^{n}(e_{i} - f_{i})^{2}}
\end{equation}
where \(e\) is the scaled eccentricity of the astronomical solution and \(f\) is the scaled and filtered combination of long- and short eccentricity.

\ijk{TODO:\ make a new figure demonstrating this?}
To study the effect of the two eccentricity cycles on the match with the astronomical solution, we also looked at the effect of assigning different weights to the two filtered signals (relative weights of 1:0, 1:0.25, 1:0.5, 1:0.75, 1:1, 0.75:1, 0.5:1, 0.25:1, and 0:1 for the long and short eccentricity cycles respectively).
We limit the main manuscript to the simplest 1:1 combination because the relative amplitude of both components is then preserved from the original data.

We update the initial age model (see \cref{sec:agemodel}) after shifting the age of the \gls{KT}-boundary by up to \qty{\pm200}{\kiloyear} and finding the offset that gives the lowest \gls{RMSD}.
This means that a new estimate for the age of the \gls{KT}-boundary for each astronomical solution is an additional outcome of this study.
To account for potential errors in the tie-points---the long eccentricity minima depths as identified in the field---we iteratively shift each tie-point from the youngest to oldest by a range of values between \qtyrange[range-phrase=~to~]{-1.6}{1.6}{\metre} in \qty{20}{\centi\metre} increments (about \qty{10}{\percent} of the long eccentricity period).
After tweaking the tie-point depth, the updated age model is applied, the record is filtered to target the long and short eccentricity cycles, and an artificial eccentricity curve is created to calculate the \gls{RMSD} as before.
Once the optimal (lowest \gls{RMSD}) tie-point depth is found, we fix it and move on to the next-youngest tie-point.
This is repeated until all tie-points have been adjusted, resulting in the overall best fit.
\ijk{@REZ:\ I'm not sure where best to link to the appendix figures that show how the RMSD scores improve after this. Bringing it all up in the results might be repetitive?}
This analysis is performed separately for the Zumaia and Sopelana sites, so that potential errors in the depth correlation between sites can be corrected for.
Furthermore, because the Sopelana site was likely located farther offshore~\cite{Batenburg2014},
treating the sites separately will (due to the scaling) correct for differences in amplitude between the records that could be the result of this difference in paleogeographic setting.
We also calculate a square root of the cumulative sum of the squared differences starting at the \gls{KT}-boundary and moving to older data points.
This allows us to visualize where the time domain the data differs most from the astronomical solutions.

\section{Results}\label{sec:results}

\ijk{something about spectral analysis? \(\to{}\) No, just rely on original study that showed this. But do I need to mention this? And if so, where?}

% first talk about how the matched filtered records look!
The matched eccentricity construct shows generally good compatibility with the different astronomical solutions (\cref{fig:rolling-depth-age}).
Especially bundle \(\text{Ma}_{405}8\) shows a very large amplitude in the short eccentricity filter, which relates to the darker interval at around \qty{117}{\metre}.
% this weird thing is soooo influential! I tried filtering it out


% then describe simple "what is the best RMSD-scoring solution?"
\gls{RMSD}-scores are very similar between the different astronomical solutions.
The combined \gls{L*} record of the Zumaia and Sopelana section gives the best \gls{RMSD} score for
% La10b (\num{1.06}) and ZB18a (\num{1.06}) (\cref{fig:full-RMSD}).???

% but we can also look at how it changes through time
\ijk{TODO:\ update this older section where I looked at discrete chunks.}
This analysis can be improved by looking at specific depth intervals separately, however.
The full Zumaia record gives the lowest \gls{RMSD}-scores for the La10b (\num{1.055}) and ZB20b (\num{1.120}) solutions,
while the Sopelana site gives the best results for ZB20a (\num{0.872}) and ZB18a (\num{0.894}), and to a lesser extent ZB20c (\num{0.959}) and La10c (\num{0.968}).
For Zumaia we also analyzed the record separately for two depth intervals to account for a change in sedimentation rate at around \qty{109.26}{\metre}.
The older interval shows a better match with the La10c (\num{0.981}) and ZB20b (1.001) solutions.
The younger interval has a slightly better match for La10b (\num{1.104}), ZB20a (\num{1.105}), and ZB20b (\num{1.116}), but results show similar \gls{RMSD} scores.

% \subsection{Very long eccentricity nodes}
Looking beyond the \gls{RMSD} scores, additional patterns may be used as criteria for selecting appropriate astronomical solutions.
\cref{fig:rolling-depth-age} shows our filtered, normalized records tuned to the astronomical solutions, highlighting where the records match the solutions better or worse.
The key differentiating factors between astronomical solutions are the locations of the very long eccentricity nodes, where the amplitude of the short eccentricity cycle is very low and the long eccentricity cycle dominates.
Because of the filtering approach it is difficult to distinguish these very long eccentricity nodes in the filtered data directly, because there may be non-orbital reasons to get a low-amplitude in the filtered short eccentricity cycle.
We can, however, detect sections with a high amplitude in the short eccentricity cycle, indicating the absence of a very long eccentricity node.

\ijk{hmm already make this conclusion or first describe that it has low amplitude in the short ecc component here and put below in discussion/conclusion?}

The high amplitude of the short eccentricity filtered records indicate that Ma405\(_{8}\) at around \qty{68.9}{\millionyearago} was probably not a very long eccentricity node.
This is significant, because solutions La10b, ZB18a, and ZB20c have a low amplitude of short eccentricity here.
The same goes for Ma405\(_{10}\) at \qty{69.7}{\millionyearago}, which does not match La10c and ZB20b.

% FIGURE 1
\begin{figure}[htb]
  \centering
  \includegraphics[width=\textwidth]{Lstar-vs-solutions.png}
  \caption{\label{fig:rolling-depth-age}
    \textbf{Maastrichtian Zumaia (purple) and Sopelana (orange) inverted filtered/normalized \gls{L*} records tuned to astronomical solutions (black, first 7 panels).}
    % This uses short linear detrending to correct for changes in sediment composition (\cref{sec:detrend}).
    %We also show Zumaia above and below \qty{109.26}{\metre} to account for the change in the sedimentation rate and lithology.
    Bottom three panels show the record in the depth domain.
    Bottom panel shows the log of the two sites, adapted from \citeA{Batenburg2014}.
    Then \gls{L*} values are shown with different ways of detrending in coloured lines (\cref{sec:detrend}).
    The record after piecewise-linear detrendeding and normalization is shown after that.
    Vertical lines show long eccentricity minima as identified in the field,
    with adjustments of up to \qty{\pm1.6}{\metre} (horizontal segments)
    and how they were matched to each astronomical solution in the time domain to minimize \gls{RMSD} (see \cref{sec:algorithm}).
    % \ijk{TODO: I'd like to illustrate long ecc/short ecc filters separately,
    %      but they are different for each one (= too many lines).
    %      A nice place to do this would be in the 3rd panel from the bottom (detrended record),
    %      but I never filter in the depth domain for my analysis and that might be confusing?}
    % -> rez said to just leave it out/potentially put it in the supplement since this is done in Z&L papers already.
    \ijk{TODO:\ add Ma405 numbering to make referring to certain intervals easier? Did it but not sure if happy with it. @REZ?}
    }
\end{figure}


% FIGURE 2
\begin{figure}[htb]
    \centering
    \includegraphics[width=0.6\textwidth]{full_RMSD_scores_all.pdf}
    \caption{\label{fig:full-RMSD-all}
      \textbf{Best overall matches.}
        \gls{RMSD} scores of compilation of Zumaia and Sopelana proxy records against several orbital solutions.
        % Tie-point depths were adjusted to arrive at the best match with each solution (\cref{sec:algorithm}). % not needed?
    }
\end{figure}

% FIGURE 3
\begin{figure}[htb]
  \centering
  \includegraphics[width=0.6\textwidth]{cumulative_rRMSD_all.png}
  \caption{\label{fig:cum-RMSD-all}
    \textbf{Best matches through time.}
    Square root cumulative sum squared difference scores of
    different astronomical solutions versus the Zumaia and Sopelana Maastrichtian composite record
    for \gls{d13C}, \gls{L*}, and \gls{MS}.
    \gls{L*} scores (middle panel) diverge most from solutions at around \ijk{TODO}.
    The root cumulative sum of squared differences is calculated as \(\text{RCSD}_{k} = \sqrt{\sum_{i=1}^{k}(e_{i} - f_{i})^{2}}\).
  }
\end{figure}


% \subsection{Sensitivity Analysis}
% talk about all the nitty-gritty details of parameter perturbations?
% - proxy (show the same pattern?)
% - site (Zumaia vs. Sopelana) part of main text
% - depth_chunk (cut Zumaia into seperate sedimentation rate intervals) -> I stopped doing this, I think the lower part of Zumaia is too short for this.
% - detrend_type
% - comb (I only show 1:1, but 1:0.5 and 0.5:1 both outperform the rest)
% - window (gaussian/rectangular/taner)
% - roll (taner roll parameter)
% - frac (width of the bandpass filter) 0.25 = target frequency \pm 25% (so 1/405 - 0.25*1/405 and 1/405 + 0.25*1/405)

% nitty-gritty:
\ijk{This is quite boring to read, what do you think?}
The different proxies match against the astronomical solutions in the time domain (\cref{fig:rolling-age-MS,fig:rolling-age-d13C}).
Different ways of detrending of the proxy records to subtract non-cyclical patterns lead to different results (\cref{fig:detrend,fig:full-RMSD-detrend,fig:Lstar-detrend}).
Furthermore, changing the parameters of the taner filter has an effect: when using a more rectangular window (roll of \num{e10} or a smoother window (roll of \(10^{3}\)) results in different ``best'' solutions (\cref{fig:full-RMSD-filter,fig:filter-windows}).
These differences can also be achieved when the width of the filters is changed (not shown). % TODO: add appendix fig for this
This effectively occurs when the optimization routine shifts one or more of the long eccentricity minima tie-points so much that an offset of about one short eccentricity cycle occurs.
See \cref{fig:full-RMSD-comb} for an example of the sensitivity to the different combinations of the long- and short eccentricity related filtered data.
See \cref{fig:full-RMSD-tie} for how \gls{RMSD}-scores change after optimizing tie-point depths.
For an idea of uncertainty around the \gls{RMSD}-scores, see \cref{fig:full-RMSD-boot} for bootstrapped values.
% This study shows some of the sensitivity of the different detrending/filtering techniques for astronomical calibration.
% didn't introduce bootstrapped values in M&M yet

\section{Discussion}\label{sec:discussion}

This study aims to find out which, if any, astronomical solutions best match the highest-quality Maastrichtian proxy records of the Zumaia and Sopelana sites, in order to arrive at an astronomically calibrated time scale.

The youngest interval for Zumaia shows a slight preference for solutions La10b, ZB20a, and ZB20b (\cref{fig:cum-RMSD}).
For the deeper/older part of the Zumaia record, only La10c (and to a lesser extent La10b) and ZB20b have good scores because ZB20a does not have a very long eccentricity node at Ma405\(_{8}\) (\appr\qty{68.9}{\millionyearago}), whereas the record clearly does.
The record shows an exceptional peak within this bundle though.

% also observe a high amplitude of the short-eccentricity cycle in Ma\(_{405}8\) in ODP Site 1262 and to a lesser extent in 1209.
Contrast to: Zumaia, ODP 1262, and 1267 Ma\(_{405}-1\) and Ma\(_{405}-2\)~\cite{Westerhold2008} figure 4.
% this is all Husson2011:
525A gray level bundle 1 (fig 4)
1267B MS 1--6.5 (fig 4) but note that they did not use the splice table from~\cite{Westerhold2008} to convert from MBSF to RMCD.
762C bundles 1 to 5.5 (fig 4)
525A gray level smoothed 6--9 (fig 3) hard to tell...
762C gray level smoothed bundles 8 to 16 (fig 3) indicates potential hiatus within this bundle but also high change
1258A MS Ma405-8--14ish (fig 3) no recovery in 10
of~\citeA{Husson2011}.

Can we look at multiple sites?


Solutions La10b and La10c showed a poorer match already in the late Cretaceous, where they did not show a very long eccentricity node at around \qty{61}{\millionyearago} whereas the Walvis Ridge data did \cite{ZeebeLourens2022EPSL}.
Then as we move to the older section the Sopelana data results in a best score for ZB20a because ZB20b has a very long eccentricity node at Ma405\(_{10}\) (\appr\qty{69.7}{\millionyearago}), whereas the amplitude of the record is large there.
Therefore, we prefer the ZB20b solution up to \appr\qty{69.2}{\millionyearago}.
For studies working with data older than this we must conclude that the data do not support any of the currently available solutions that are also consistent with data from the late Cretaceous.

We suggest that future studies generate new astronomical solutions with similar initial settings to the ZB20b solution and, using appropriate parameter variations, attempt to reproduce the patterns in the \gls{L*} record of the Sopelana site.





%%

%  Numbered lines in equations:
%  To add line numbers to lines in equations,
%  \begin{linenomath*}
%  \begin{equation}
%  \end{equation}
%  \end{linenomath*}



%% Enter Figures and Tables near as possible to where they are first mentioned:
%
% DO NOT USE \psfrag or \subfigure commands.
%
% Figure captions go below the figure.
% Table titles go above tables;  other caption information
%  should be placed in last line of the table, using
% \multicolumn2l{$^a$ This is a table note.}
%
%----------------
% EXAMPLE FIGURES
%
% \begin{figure}
% \includegraphics{example.png}
% \caption{caption}
% \end{figure}
%
% Giving latex a width will help it to scale the figure properly. A simple trick is to use \textwidth. Try this if large figures run off the side of the page.
% \begin{figure}
% \noindent\includegraphics[width=\textwidth]{anothersample.png}
%\caption{caption}
%\label{pngfiguresample}
%\end{figure}
%
%
% If you get an error about an unknown bounding box, try specifying the width and height of the figure with the natwidth and natheight options. This is common when trying to add a PDF figure without pdflatex.
% \begin{figure}
% \noindent\includegraphics[natwidth=800px,natheight=600px]{samplefigure.pdf}
%\caption{caption}
%\label{pdffiguresample}
%\end{figure}
%
%
% PDFLatex does not seem to be able to process EPS figures. You may want to try the epstopdf package.
%

%
% ---------------
% EXAMPLE TABLE
% Please do NOT include vertical lines in tables
% if the paper is accepted, Wiley will replace vertical lines with white space
% the CLS file modifies table padding and vertical lines may not display well
%
 % \begin{table}
 % \caption{Time of the Transition Between Phase 1 and Phase 2$^{a}$}
 % \centering
 % \begin{tabular}{l c}
 % \hline
 %  Run  & Time (min)  \\
 % \hline
 %   $l1$  & 260   \\
 %   $l2$  & 300   \\
 %   $l3$  & 340   \\
 %   $h1$  & 270   \\
 %   $h2$  & 250   \\
 %   $h3$  & 380   \\
 %   $r1$  & 370   \\
 %   $r2$  & 390   \\
 % \hline
 % \multicolumn{2}{l}{$^{a}$Footnote text here.}
 % \end{tabular}
 % \end{table}

%% SIDEWAYS FIGURE and TABLE
% AGU prefers the use of {sidewaystable} over {landscapetable} as it causes fewer problems.
%
% \begin{sidewaysfigure}
% \includegraphics[width=20pc]{figsamp}
% \caption{caption here}
% \label{newfig}
% \end{sidewaysfigure}
%
%  \begin{sidewaystable}
%  \caption{Caption here}
% \label{tab:signif_gap_clos}
%  \begin{tabular}{ccc}
% one&two&three\\
% four&five&six
%  \end{tabular}
%  \end{sidewaystable}

%% If using numbered lines, please surround equations with \begin{linenomath*}...\end{linenomath*}
%\begin{linenomath*}
%\begin{equation}
%y|{f} \sim g(m, \sigma),
%\end{equation}
%\end{linenomath*}

%%% End of body of article

%%%%%%%%%%%%%%%%%%%%%%%%%%%%%%%%
%% Optional Appendix goes here
%
% The \appendix command resets counters and redefines section heads
%
% After typing \appendix
%
%\section{Here Is Appendix Title}
% will show
% A: Here Is Appendix Title
%
\appendix
% this template also turns the open research section into an appendix...

% We show some more combinations of \cref{fig:full-RMSD-all} in \cref{fig:full-RMSD-detrend}.
% We show an adaptation of \cref{fig:rolling-depth-age} for \gls{MS} (\cref{fig:rolling-age-MS}) and for \gls{d13C} (\cref{fig:rolling-age-d13C}).

% Appendix Figure: MS filters vs solutions
\begin{figure}[htb]
  \centering
  \includegraphics[width=\textwidth]{MS-vs-solutions.png}
  \caption{\label{fig:rolling-age-MS}
    \textbf{Maastrichtian Zumaia (purple) and Sopelana (orange) filtered/normalized \gls{MS} records tuned to astronomical solutions (black, first 7 panels).}
    % This uses short linear detrending to correct for changes in sediment composition (\cref{sec:detrend}).
    %We also show Zumaia above and below \qty{109.26}{\metre} to account for the change in the sedimentation rate and lithology.
    Bottom three panels show the record in the depth domain.
    Bottom panel shows the log of the two sites, adapted from \citeA{Batenburg2014}.
    Then \gls{MS} values are shown with different ways of detrending in coloured lines (\cref{sec:detrend}).
    The record after piecewise-linear detrendeding and normalization is shown after that.
    Vertical lines show long eccentricity minima as identified in the field,
    with adjustments of up to \qty{\pm1.6}{\metre} (horizontal segments)
    and how they were matched to each astronomical solution in the time domain to minimize their \gls{RMSD} (see \cref{sec:algorithm}).
    }
\end{figure}

% Appendix Figure: d13C filters vs solutions
\begin{figure}[htb]
  \centering
  \includegraphics[width=\textwidth]{d13C-vs-solutions.png}
  \caption{\label{fig:rolling-age-d13C}
    \textbf{Maastrichtian Zumaia (purple) inverted filtered/normalized \gls{d13C} records tuned to astronomical solutions (black, first 7 panels).}
    % This uses short linear detrending to correct for changes in sediment composition (\cref{sec:detrend}).
    %We also show Zumaia above and below \qty{109.26}{\metre} to account for the change in the sedimentation rate and lithology.
    Bottom three panels show the record in the depth domain.
    Bottom panel shows the log of the two sites, adapted from \citeA{Batenburg2014}.
    Then \gls{d13C} values are shown with different ways of detrending in coloured lines (\cref{sec:detrend}).
    The record after piecewise-linear detrendeding and normalization is shown after that.
    Vertical lines show long eccentricity minima as identified in the field,
    with adjustments of up to \qty{\pm1.6}{\metre} (horizontal segments)
    and how they were matched to each astronomical solution in the time domain to minimize their \gls{RMSD} (see \cref{sec:algorithm}).
    \ijk{TODO: comment on overfitting detrending and lower resolution of data.}
    \ijk{@REZ: do you see the \appr\qty{1.6}{\millionyear} cycles here before detrending? Might be something to look at?}
    }
\end{figure}

% Appendix Figure: detrend fits compared to raw data vs. depth
% TODO: remove some of these?
\begin{figure}[htbp]
  \centering
  \includegraphics[width=.9\linewidth]{depth_detrend.png}
  \caption{\label{fig:detrend}
    \textbf{Zumaia (purple) and Sopelana (yellow) trend removal strategies.}
    The raw data and the lines that were fit (other colours), which were subtracted from the record prior to filtering.
    The piecewise linear fit that we subtract in the main manuscript corresponds to \texttt{lin\_pred\_med}.
    % The detrended resultant record is callend \texttt{lin\_scl\_med}.
  }
\end{figure}

% Appendix Figure: effects of detrending in time domain
\begin{figure}[htb]
  \centering \includegraphics[width=\textwidth]{sol_SD_detrend.pdf}
  \caption{\label{fig:Lstar-detrend}
  \textbf{Effects of different detrending strategies in the time domain.}
    Illustration of how simple scaling of \gls{L*} with subsequent linear detrending (lower opacity lines)
    compares to the piecewize linear detrending strategy used in the main manuscript.
    Both use the same taner filter to extract long and short eccentricity components.
    The large changes in lithology result in a very large amplitude filter in the simple detrending approach.
    For most solutions, this simply results in a worse fit, but for ZB20a and La10b it makes our algorithm shift the long-eccentricity tiepoint depths enough that it gets offset by about one short-eccentricty cycle, resulting in an even worse fit.
    }
\end{figure}

% Appendix Figure: Bootstrapped RMSD
\begin{figure}[htb]
    \centering
    \includegraphics[width=0.6\textwidth]{full_RMSD_boot.png}
    \caption{\label{fig:full-RMSD-boot}
        \textbf{\gls{RMSD} scores of compilation of Zumaia and Sopelana proxy records against several orbital solutions.}
        Shaded intervals represent the bootstrapped (\(N = 10^{5}\)) \qtyrange{5}{95}{\percent} confidence intervals of the \gls{RMSD} (in increments of \qty{5}{\percent}).
        % Tie-point depths were adjusted to arrive at the best match with each solution (\cref{sec:algorithm}).
        This illustrates that while differences between astronomical solutions are not large, they may be significant.
    }
\end{figure}

% Appendix Figure: RMSD scores before and after tie-point adjustments
\begin{figure}[htb]
  \centering \includegraphics[width=\textwidth]{sol_SD_tie.png}
  \caption{\label{fig:full-RMSD-tie}
    \textbf{Illustration of how RMSD scores change after tie-point depth optimization.}
    The change in \gls{RMSD} scores for the
    %piecewize linearly detrended
    Zumaia (not the Zumaia/Sopelana composite!) proxy records (column panels)
    % where the long and short eccentricity components where filtered with a taner filter
    when we allow tie-point depths identified in the field to vary by \qty{\pm1.6}{\metre}.
    This uses the taner filter from the main text on the piecewise linearly detrended record.
    }
\end{figure}

% Appendix Figure: RMSD scores for different ways of detrending
\begin{figure}[htb]
  \centering \includegraphics[width=\textwidth]{full_RMSD_detrend_comparison.png}
  \caption{\label{fig:full-RMSD-detrend}
    \textbf{Sensitivity analysis of various ways of detrending the data.}
    This uses optimized tie-point depths and limits the results to a rectangular filter.
    We show main-paper taner filters for \texttt{lin\_scl\_med} with different colours for each proxy for context.
    Note that the \texttt{lin\_scl\_fine} for the taner filter is very similar to the one for \texttt{lin\_scl\_med}.
    }
\end{figure}

% Appendix Figure:  RMSD scores for different filter strategies
\begin{figure}[htb]
  \centering \includegraphics[width=\textwidth]{full_RMSD_filter_comparison.png}
  \caption{\label{fig:full-RMSD-filter}
    \textbf{Sensitivity analysis of different types of bandpass filtering.}
    %Note that gaussian window also narrows filter width (see \cref{fig:filter-windows}).
    This uses the \texttt{lin\_scl\_med} detrend type and tie-point depth optimization.
    The taner filter has the main-text roll parameter of \num{e10}.
    Note that we have reparametrized the gaussian filter so that the upper and lower boundaries correspond to \(\pm\sigma\).
    % We also show the original for \gls{L*} but this is effectively a narrower filter range.
    See \cref{fig:filter-windows} for illustrations of filter windows and alternative roll parameters.
    }
\end{figure}

% Appendix Figure:  Filter Windows + RMSD scores for different taner filters
\begin{figure}[htb]
  \centering \includegraphics[width=0.9\textwidth]{filter_windows.pdf}
  \caption{\label{fig:filter-windows}
    \textbf{Sensitivity analysis for taner filter roll parameters.}
    The rectangular (gray), gaussian (pink), and taner filters (blue fill)
    use the same filter boundary frequencies (\(\frac{1}{405}\) and \(\frac{1}{110}\)~\si[per-mode=power]{\per\kiloyear} \qty{\pm25}{\percent}).
    Note that the way \texttt{astrochron} parametrizes the gaussian window width (narrower gaussian, better RMSD scores) is not directly comparable to the taner and rectangular filters.
    Therefore, we also show a reparametrized gaussian filter, where the upper and lower boundaries correspond to \(\pm\sigma\).
    % We experimentally found that multiplying the fraction of the target frequency added/subtracted to determine the lower and upper bounds by three matched a true normal distribution best.
    Different roll parameters for the taner filters are shown,
    from the pointy \num{e3} (narrowest peak, widest shoulders) moving outwards from the peak to
    \num{e6}, \num{e8}, \num{e10}, \num{e12}, and \num{e100} (approaching the rectangular filter).
    We show MTM-spectral peaks of the astronomical solutions analyzed in this study in black for reference.
    The bottom panel contrasts \gls{RMSD} scores for the \texttt{lin\_scl\_med} detrended \gls{L*} data with tie-point depth optimization.
    }
\end{figure}


%%%%%%%%%%%%%%%%%%%%%%%%%%%%%%%%%%%%%%%%%%%%%%%%%%%%%%%%%%%%%%%%%%%%%%%%%%
%                   some spectral analysis figures                       %
%%%%%%%%%%%%%%%%%%%%%%%%%%%%%%%%%%%%%%%%%%%%%%%%%%%%%%%%%%%%%%%%%%%%%%%%%%

% \begin{figure}[htb]
%   \centering \includegraphics[width=\textwidth]{Zumaia-Sopelana_mtm_raw.pdf}
%   \caption{\label{fig:spectral-depth}
%     \textbf{Spectral analysis in the depth domain.}
%     % Do I need refs for all of these?
%     \ijk{In the end probably show analysis only in time-domain?}
%     BT = Blackman-Tukey,
%     FFT = Fast Fourier Transform,
%     LOWSPEC = Robust Locally-Weighted Regression Spectral Background Estimation \cite{Meyers2012},
%     LS = Lomb-Scargle,
%     MTLS = Multi-taper Averaged Lomb-Scargle periodogram of (un)evenly
% spaced data \cite{Springford2020},
%     MTM = Multitaper method \cite{Thomson1982}.
%     Shaded intervals for the MTM and LOWSPEC indicate AR1 fit and AR1-power and LOWSPEC fit and power confidence levels.
%   }
% \end{figure}
%
% \begin{figure}[htb]
%   \centering \includegraphics[width=\textwidth]{Zumaia-Sopelana_mtm.pdf}
%   \caption{\label{fig:spectral-depth}
%     \textbf{Spectral analysis in the depth domain.}
%     % Do I need refs for all of these?
%     \ijk{This is the same as above but after linear detrending with \texttt{lin\_scl\_fine}.}
%     BT = Blackman-Tukey,
%     FFT = Fast Fourier Transform,
%     LOWSPEC = Robust Locally-Weighted Regression Spectral Background Estimation \cite{Meyers2012},
%     LS = Lomb-Scargle,
%     MTLS = Multi-taper Averaged Lomb-Scargle periodogram of (un)evenly
% spaced data \cite{Springford2020},
%     MTM = Multitaper method \cite{Thomson1982}.
%     Shaded intervals for the MTM and LOWSPEC indicate AR1 fit and AR1-power and LOWSPEC fit and power confidence levels.
%   }
% \end{figure}
%
%
% \begin{figure}[htb]
%   \centering \includegraphics[width=\textwidth]{Zumaia_Sopelana_spectra_filters_raw.pdf}
%   \caption{\label{fig:spectral-age-raw}
%     \textbf{Spectral analysis in the time domain.}
%     % Do I need refs for all of these?
%     \ijk{This is raw values, only linear detrend}
%     % BT = Blackman-Tukey,
%     FFT = Fast Fourier Transform,
%     LOWSPEC = Robust Locally-Weighted Regression Spectral Background Estimation \cite{Meyers2012},
%     % LS = Lomb-Scargle,
%     MTLS = Multi-taper Averaged Lomb-Scargle periodogram of (un)evenly
% spaced data \cite{Springford2020},
%     MTM = Multitaper method \cite{Thomson1982}.
%     Shaded intervals for the MTM and LOWSPEC indicate AR1 fit and AR1-power and LOWSPEC fit and power confidence levels.
%   }
% \end{figure}
%
%
% \begin{figure}[htb]
%   \centering
%   \includegraphics[width=1.2\textwidth]{Zumaia_MS_1-1_solutions_simple_with_log.pdf}
%   \caption{\label{fig:rolling-age-MS}
%     Same as \cref{fig:rolling-depth-age} but for \gls{MS}.}
% \end{figure}

% \begin{figure}[htb]
%   \centering
%   \includegraphics[width=0.9\textwidth]{Zumaia_d13C_1-1_solutions_simple_with_log.pdf}
%   \caption{\label{fig:rolling-age-d13C}
%     Same as \cref{fig:rolling-depth-age} but for \gls{d13C}.}
% \end{figure}


%%%%%%%%%%%%%%%%%%%%%%%%%%%%%%%%%%%%%%%%%%%%%%%%%%%%%%%%%%%%%%%%
%
% Optional Glossary, Notation or Acronym section goes here:
%
%%%%%%%%%%%%%%
% Glossary is only allowed in Reviews of Geophysics
%  \begin{glossary}
%  \term{Term}
%   Term Definition here
%  \term{Term}
%   Term Definition here
%  \term{Term}
%   Term Definition here
%  \end{glossary}

%
%%%%%%%%%%%%%%
% Acronyms
%   \begin{acronyms}
%   \acro{Acronym}
%   Definition here
%   \acro{EMOS}
%   Ensemble model output statistics
%   \acro{ECMWF}
%   Centre for Medium-Range Weather Forecasts
%   \end{acronyms}

%
%%%%%%%%%%%%%%
% Notation
%   \begin{notation}
%   \notation{$a+b$} Notation Definition here
%   \notation{$e=mc^2$}
%   Equation in German-born physicist Albert Einstein's theory of special
%  relativity that showed that the increased relativistic mass ($m$) of a
%  body comes from the energy of motion of the body—that is, its kinetic
%  energy ($E$)—divided by the speed of light squared ($c^2$).
%   \end{notation}



\section*{Open Research}

\Gls{MS}, \gls{L*}, and \gls{d13C} data used in this study are from \citeA{Batenburg2012,Batenburg2012}.

Analysis was performed using the R programming language~\cite{RCoreTeam2020} and made use of \texttt{astrochron} \citeA{Meyers2014} and the \texttt{tidyverse} \citeA{Wickham2019}.
The new R package \texttt{AstronomicalSolutions} will be made available on publication on \url{https://github.com/japhir/AstronomicalSolutions} \citeA{Kocken2024}.

\ijk{\textbf{TODO:} come up with nice package name (working title: \texttt{AstronomicalSolutions} so it's broad enough for future additions) and host on github/ archive on Zenodo.}

% AGU requires an Availability Statement for the underlying data needed to understand, evaluate, and build upon the reported research at the time of peer review and publication.

% Authors should include an Availability Statement for the software that has a significant impact on the research. Details and templates are in the Availability Statement section of the Data and Software for Authors Guidance: \url{https://www.agu.org/Publish-with-AGU/Publish/Author-Resources/Data-and-Software-for-Authors#availability}

% It is important to cite individual datasets in this section and, and they must be included in your bibliography. Please use the type field in your bibtex file to specify the type of data cited. Some options include Dataset, Software, Collection, ComputationalNotebook. Ex:
% \\
% \begin{verbatim}

% @misc{https://doi.org/10.7283/633e-1497,
%   doi = {10.7283/633E-1497},
%   url = {https://www.unavco.org/data/doi/10.7283/633E-1497},
%   author = {de Zeeuw-van Dalfsen, Elske and Sleeman, Reinoud},
%   title = {KNMI Dutch Antilles GPS Network - SAB1-St_Johns_Saba_NA P.S.},
%   publisher = {UNAVCO, Inc.},
%   year = {2019},
%   type = {dataset}
% }

% \end{verbatim}

% For physical samples, use the IGSN persistent identifier, see the International Geo Sample Numbers section:
% \url{https://www.agu.org/Publish-with-AGU/Publish/Author-Resources/Data-and-Software-for-Authors#IGSN}
%%%%%%%%%%%%%%%%%%%%%%%%%%%%%%%%%%%%%%%%%%%%%%%

\acknowledgments
% This section is optional. Include any Acknowledgments here.
% The acknowledgments should list:\\
% All funding sources related to this work from all authors\\
% Any real or perceived financial conflicts of interests for any author\\
% Other affiliations for any author that may be perceived as having a conflict of interest with respect to the results of this paper.\\
% It is also the appropriate place to thank colleagues and other contributors. AGU does not normally allow dedications.

This work was supported by the Heising-Simons Foundation, under the CycloAstro
Cohort project 3.

%% ------------------------------------------------------------------------ %%
%% References and Citations

%%%%%%%%%%%%%%%%%%%%%%%%%%%%%%%%%%%%%%%%%%%%%%%
%
% \bibliography{<name of your .bib file>} don't specify the file extension
%
% don't specify bibliographystyle

% In the References section, cite the data/software described in the Availability Statement (this includes primary and processed data used for your research). For details on data/software citation as well as examples, see the Data & Software Citation section of the Data & Software for Authors guidance
% https://www.agu.org/Publish-with-AGU/Publish/Author-Resources/Data-and-Software-for-Authors#citation

%%%%%%%%%%%%%%%%%%%%%%%%%%%%%%%%%%%%%%%%%%%%%%%

\bibliography{references}


%Reference citation instructions and examples:
%
% Please use ONLY \cite and \citeA for reference citations.
% \cite for parenthetical references
% ...as shown in recent studies (Simpson et al., 2019)
% \citeA for in-text citations
% ...Simpson et al. (2019) have shown...
%
%
%...as shown by \citeA{jskilby}.
%...as shown by \citeA{lewin76}, \citeA{carson86}, \citeA{bartoldy02}, and \citeA{rinaldi03}.
%...has been shown \cite{jskilbye}.
%...has been shown \cite{lewin76,carson86,bartoldy02,rinaldi03}.
%... \cite <i.e.>[]{lewin76,carson86,bartoldy02,rinaldi03}.
%...has been shown by \cite <e.g.,>[and others]{lewin76}.
%
% apacite uses < > for prenotes and [ ] for postnotes
% DO NOT use other cite commands (e.g., \citet, \citep, \citeyear, \citealp, etc.).
% \nocite is okay to use to add references from your Supporting Information
%


\end{document}



% More Information and Advice:

%% ------------------------------------------------------------------------ %%
%
%  SECTION HEADS
%
%% ------------------------------------------------------------------------ %%

% Capitalize the first letter of each word (except for
% prepositions, conjunctions, and articles that are
% three or fewer letters).

% AGU follows standard outline style; therefore, there cannot be a section 1 without
% a section 2, or a section 2.3.1 without a section 2.3.2.
% Please make sure your section numbers are balanced.
% ---------------
% Level 1 head
%
% Use the \section{} command to identify level 1 heads;
% type the appropriate head wording between the curly
% brackets, as shown below.
%
%An example:
%\section{Level 1 Head: Introduction}
%
% ---------------
% Level 2 head
%
% Use the \subsection{} command to identify level 2 heads.
%An example:
%\subsection{Level 2 Head}
%
% ---------------
% Level 3 head
%
% Use the \subsubsection{} command to identify level 3 heads
%An example:
%\subsubsection{Level 3 Head}
%
%---------------
% Level 4 head
%
% Use the \subsubsubsection{} command to identify level 3 heads
% An example:
%\subsubsubsection{Level 4 Head} An example.
%
%% ------------------------------------------------------------------------ %%
%
%  IN-TEXT LISTS
%
%% ------------------------------------------------------------------------ %%
%
% Do not use bulleted lists; enumerated lists are okay.
% \begin{enumerate}
% \item
% \item
% \item
% \end{enumerate}
%
%% ------------------------------------------------------------------------ %%
%
%  EQUATIONS
%
%% ------------------------------------------------------------------------ %%

% Single-line equations are centered.
% Equation arrays will appear left-aligned.

% Math coded inside display math mode \[ ...\]
%  will not be numbered, e.g.,:
%  \[ x^2=y^2 + z^2\]

%  Math coded inside \begin{equation} and \end{equation} will
%  be automatically numbered, e.g.,:
%  \begin{equation}
%  x^2=y^2 + z^2
%  \end{equation}


% % To create multiline equations, use the
% % \begin{eqnarray} and \end{eqnarray} environment
% % as demonstrated below.
% \begin{eqnarray}
%   x_{1} & = & (x - x_{0}) \cos \Theta \nonumber \\
%         && + (y - y_{0}) \sin \Theta  \nonumber \\
%   y_{1} & = & -(x - x_{0}) \sin \Theta \nonumber \\
%         && + (y - y_{0}) \cos \Theta.
% \end{eqnarray}

%If you don't want an equation number, use the star form:
%\begin{eqnarray*}...\end{eqnarray*}

% Break each line at a sign of operation
% (+, -, etc.) if possible, with the sign of operation
% on the new line.

% Indent second and subsequent lines to align with
% the first character following the equal sign on the
% first line.

% Use an \hspace{} command to insert horizontal space
% into your equation if necessary. Place an appropriate
% unit of measure between the curly braces, e.g.
% \hspace{1in}; you may have to experiment to achieve
% the correct amount of space.


%% ------------------------------------------------------------------------ %%
%
%  EQUATION NUMBERING: COUNTER
%
%% ------------------------------------------------------------------------ %%

% You may change equation numbering by resetting
% the equation counter or by explicitly numbering
% an equation.

% To explicitly number an equation, type \eqnum{}
% (with the desired number between the brackets)
% after the \begin{equation} or \begin{eqnarray}
% command.  The \eqnum{} command will affect only
% the equation it appears with; LaTeX will number
% any equations appearing later in the manuscript
% according to the equation counter.
%

% If you have a multiline equation that needs only
% one equation number, use a \nonumber command in
% front of the double backslashes (\\) as shown in
% the multiline equation above.

% If you are using line numbers, remember to surround
% equations with \begin{linenomath*}...\end{linenomath*}

%  To add line numbers to lines in equations:
%  \begin{linenomath*}
%  \begin{equation}
%  \end{equation}
%  \end{linenomath*}
