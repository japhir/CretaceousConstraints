%%%%%%%%%%%%%%%%%%%%%%%%%%%%%%%%%%%%%%%%%%%%%%%%%%%%%%%%%%%%%%%%%%%%%%%%%%%%
% adapted from AGUJournalTemplate.tex: this template file is for articles formatted with LaTeX
%
%% To submit your paper:
\documentclass[draft]{agujournal2019}
% packages from the template
\usepackage{url} %this package should fix any errors with URLs in refs.
\usepackage{lineno}
\usepackage[inline]{trackchanges} %for better track changes. finalnew option will compile document with changes incorporated.
\usepackage{soul}

% IJK added packages
% for units, type \qty{x}{\unit}
\usepackage{siunitx}
% for chemical equations or, in my case, d13C
\usepackage[version=4]{mhchem}
%\usepackage{chemformula} % alternative for above
% \usepackage[utf8]{luainputenc} % allow UTF-8 input? δ13C?
% \usepackage{hyperref} % links between references etc % doesn't seem to work
% with this template?
\usepackage[capitalise,nameinlink,noabbrev]{cleveref}
% use same macro for acronym, this writes out the first use and then
% abbreviates after (use \gls{key}).
\usepackage{glossaries}
%\makeglossaries % not needed if we don't want to print them at the end


\newcommand{\appr}{\raise.17ex\hbox{$\scriptstyle\sim$}} % approximately symbol

\newcommand{\rez}{\textcolor{magenta}}
% I'll use this to highlight sections that I want to focus your attention on!
% I liked violet best, but might be hard to distinguish quickly from magenta...
\newcommand{\ijk}{\textcolor{violet}}


% normally I strongly prefer biblatex, but the AGU template uses only \cite,
% \citeA, and \nocite from apacite :(
% \usepackage[giveninits=true,uniquename=false,uniquelist=false,date=year,hyperref=true,mincitenames=1,maxcitenames=2,backend=biber,backref,doi=true,url=false,isbn=false]{biblatex}
% \addbibresource{references.bib}

% IJK package config
\sisetup{%
detect-all,% Detect surrounding font context, like weight, italics etc.
%
% Alternative range-phrase:
% en-dash via '--', but inside \text{}, so it's not 'two minus signs'
range-phrase={\,\text{--}\,},
separate-uncertainty=true,
multi-part-units=single,
list-units=single,% single: Print unit only once, at end
range-units=single,% single: Print unit only once, at end
per-mode=symbol,
}%
\DeclareSIUnit\annum{a} % a year, used in years before present
\DeclareSIUnit\year{yr} % a year as a duration
\DeclareSIUnit\millionyearago{\mega\annum} % a time, e.g. 34 million years before present
\DeclareSIUnit\millionyear{\mega\year} % a duration, e.g. the event lasted for 2 million years
\DeclareSIUnit\kiloyearago{\kilo\annum} % a time, e.g. 14 thousand years ago (before present)
\DeclareSIUnit\kiloyear{\kilo\year} % a duration, e.g. the event lasted 200 thousand years

% this makes it so the first use spells it out, the next uses the abbreviation!
\newacronym{d13C}{\ensuremath{\delta}\ce{^13C}}{carbon isotope ratio}
\newacronym{MS}{MS}{magnetic susceptibility}
\newacronym{L*}{L*}{color reflectance}

\newacronym{RMSD}{RMSD}{root mean square deviation}
\newacronym{KT}{K/T}{Cretaceous--Tertiary}
\newacronym{PETM}{PETM}{Paleocene--Eocene thermal maximum}

\newacronym{GAM}{GAM}{generalized additive model}
\newacronym{FFT}{FFT}{fast Fourier transform}
\newacronym{MTM}{MTM}{multi-taper method}
\newacronym{MTLS}{MTLS}{multi-taper averaged Lomb-Scargle periodogram of (un)evenly spaced data}


\linenumbers
%%%%%%%
% As of 2018 we recommend use of the TrackChanges package to mark revisions.
% The trackchanges package adds five new LaTeX commands:
%
%  \note[editor]{The note}
%  \annote[editor]{Text to annotate}{The note}
%  \add[editor]{Text to add}
%  \remove[editor]{Text to remove}
%  \change[editor]{Text to remove}{Text to add}
%
% complete documentation is here: http://trackchanges.sourceforge.net/
%%%%%%%

% \draftfalse

%% Enter journal name below.
%% Choose from this list of Journals:
%
% JGR: Atmospheres
% JGR: Biogeosciences
% JGR: Earth Surface
% JGR: Oceans
% JGR: Planets
% JGR: Solid Earth
% JGR: Space Physics
% Global Biogeochemical Cycles
% Geophysical Research Letters
% Paleoceanography and Paleoclimatology
% Radio Science
% Reviews of Geophysics
% Tectonics
% Space Weather
% Water Resources Research
% Geochemistry, Geophysics, Geosystems
% Journal of Advances in Modeling Earth Systems (JAMES)
% Earth's Future
% Earth and Space Science
% Geohealth
%
% ie, \journalname{Water Resources Research}

\journalname{Paleoceanography and Paleoclimatology}


\begin{document}

%% ------------------------------------------------------------------------ %%
%  Title
%
% (A title should be specific, informative, and brief. Use
% abbreviations only if they are defined in the abstract. Titles that
% start with general keywords then specific terms are optimized in
% searches)
%
%% ------------------------------------------------------------------------ %%

\title{Cretaceous Constraints on Astronomical Solutions}
% technically, Maastrichtian would be more appropriate but that doesn't alliterate ;-)

%% ------------------------------------------------------------------------ %%
%
%  AUTHORS AND AFFILIATIONS
%
%% ------------------------------------------------------------------------ %%

% List authors by first name or initial followed by last name and
% separated by commas. Use \affil{} to number affiliations, and
% \thanks{} for author notes.
% Additional author notes should be indicated with \thanks{} (for
% example, for current addresses).

% Example: \authors{A. B. Author\affil{1}\thanks{Current address, Antartica}, B. C. Author\affil{2,3}, and D. E.
% Author\affil{3,4}\thanks{Also funded by Monsanto.}}

\authors{I. J. Kocken\affil{1} and R. E. Zeebe\affil{1}}
\affiliation{1}{
School of Ocean and Earth Science and Technology,
University of Hawaii at Manoa,
1000 Pope Road, MSB 629, Honolulu, HI 96822, USA}

% \affiliation{2}{Second Affiliation}
%(repeat as many times as is necessary)

%% Corresponding Author:
% Corresponding author mailing address and e-mail address:

% (include name and email addresses of the corresponding author.  More
% than one corresponding author is allowed in this LaTeX file and for
% publication; but only one corresponding author is allowed in our
% editorial system.)

\correspondingauthor{Ilja Kocken}{ikocken@hawaii.edu}

%% Keypoints, final entry on title page.

%  List up to three key points (at least one is required)
%  Key Points summarize the main points and conclusions of the article
%  Each must be 140 characters or fewer with no special characters or punctuation and must be complete sentences

% Example:
% \begin{keypoints}
% \item	List up to three key points (at least one is required)
% \item	Key Points summarize the main points and conclusions of the article
% \item	Each must be 140 characters or fewer with no special characters or punctuation and must be complete sentences
% \end{keypoints}

\begin{keypoints}
\item enter point 1 here
\item enter point 2 here
\item enter point 3 here
\end{keypoints}

%% ------------------------------------------------------------------------ %%
%
%  ABSTRACT and PLAIN LANGUAGE SUMMARY
%
% A good Abstract will begin with a short description of the problem
% being addressed, briefly describe the new data or analyses, then
% briefly states the main conclusion(s) and how they are supported and
% uncertainties.

% The Plain Language Summary should be written for a broad audience,
% including journalists and the science-interested public, that will not have
% a background in your field.
%
% A Plain Language Summary is required in GRL, JGR: Planets, JGR: Biogeosciences,
% JGR: Oceans, G-Cubed, Reviews of Geophysics, and JAMES.
% see http://sharingscience.agu.org/creating-plain-language-summary/)
%
%% ------------------------------------------------------------------------ %%

%% \begin{abstract} starts the second page

\begin{abstract}
[ enter your Abstract here ]
\end{abstract}

\section*{Plain Language Summary}
[ enter your Plain Language Summary here or delete this section]


%%% Suggested section heads:
% \section{Introduction}
%
% The main text should start with an introduction. Except for short
% manuscripts (such as comments and replies), the text should be divided
% into sections, each with its own heading.

% Headings should be sentence fragments and do not begin with a
% lowercase letter or number. Examples of good headings are:

% \section{Materials and Methods}
% Here is text on Materials and Methods.
%
% \subsection{A descriptive heading about methods}
% More about Methods.
%
% \section{Data} (Or section title might be a descriptive heading about data)
%
% \section{Results} (Or section title might be a descriptive heading about the
% results)
%
% \section{Conclusions}

%%% IJK some notes on how I write special symbols and units here:

% how do I want to write d13C?
% \(\delta^{13}\)C % plain
% \ce{\delta^13C} % mhchem
% \(\delta\)\ch{^{13}C} % chemformula
% \gls{d13C} % using glossaries and mhchem  <-----

% test my acronyms
% \Gls{MS}
% \Gls{d13C}
% \Gls{L*}
% \Gls{KT} boundary

% test my units. Define them the first time we use them too.
% \qty{5}{\kiloyear}
% \qtylist{2;3;5}{\kiloyear}
% \qtyrange{2}{35}{\kiloyear}
% \qty{66}{\millionyearago}

\section{Introduction}\label{sec:intro}

``Cool quote that addresses how crucial good dating is''.
\ijk{I like how some of RZ's paper's start with an ancient quote}
Maybe something from \citeA{Berger1980} review?

As we move to older and older time-periods, the available records will get further distorted and TODO.

In this work, we use geological data to constrain astronomical solution in order to most accurately reflect the orbit of the Earth; for previous work on the Paleocene and Eocene, see\citeA{ZeebeLourens2019,ZeebeLourens2022EPSL}.
The Walvis Ridge record is exceptional in its quality, so finding comparable records that capture precession-scale alternations and patterns of amplitude modulation for older time periods is difficult at this time.

We use the Zumaia and Sopelana terrestrial composite records, as generated by \citeA{Batenburg2012,Batenburg2014}.

Here we aim to use the best available records older than the \gls{KT} boundary to constrain which astronomical solution best reflects this time period.

Something to introduce long eccentricity (\qty{405}{\kiloyear} period) and short eccentricity (\qty{100}{\kiloyear} period).
Also introduce very long eccentricity nodes!



\section{Material and Methods}\label{sec:mm}

\ijk{\textbf{TODO:} Decide on consistent naming for scaled vs.\ normalized: subtract mean, divide by standard deviation. I like scaled?}

\subsection{Datasets Used}\label{sec:data}

Site descriptions for Zumaia and Sopelana.
We use proxy records \gls{MS}, \gls{d13C}, and \gls{L*} measurements~\cite{Batenburg2012,Batenburg2014}.

\subsection{Age Model}\label{sec:agemodel}

The age model for the Zumaia and Sopelana sites was based on the identification of long eccentricity minima in the field \citeA{Batenburg2012,Batenburg2014}.
We assumed a duration of \qty{405}{\kiloyear} for each of these long eccentricity cycles to get a floating age model.
A previous best-estimate of the age of the \gls{KT} boundary (\qty{65.9}{\millionyearago}) was used to anchor the floating age model to absolute time.
We later perform optimizations for each astronomical solution that allow the age of the \gls{KT} boundary to shift by \qty{\pm200}{\kiloyear} (see \cref{sec:algorithm}).

\subsection{Removing Long-Term Trends}\label{sec:detrend}

We removed long-term trends in the record using various strategies because there are some significant shifts in the lithology (red, clay-rich intervals and whiter marl-rich intervals) that could hamper spectral analysis and filtering.
Below is an overview of several of the strategies employed to remove these long-term non-cyclic trends.

We limit the effect these trends can have on our results by applying bandpass filtering only in the time-domain, after applying the age model from the field (\cref{sec:algorithm}).

\textbf{TODO:} I'll remove the ones we think are stupid after, and revise this table into a simple paragraph. For now this summarizes what I've looked at!

\begin{table}[htbp]
  \caption{\label{tab:detrend_types}
    The different methods to remove long-term trends.}
\centering
\begin{tabular}{ll}
  name & method\\
  \hline
  \texttt{value} & The raw value. Most spectral analysis/bandpass filtering remove a single linear trend.\\
  \texttt{scl} & The scaled raw value (subtract mean, divide by standard deviation). \\
  \texttt{lin\_scl\_rw} & Remove linear trends from red/white intervals separately, then scale the result.\\ % I think I chose wrong intervals for this so have removed it.
  \texttt{lin\_scl\_coarse} & Chop the record into three (33--59 m) sections and detrend linearly.\\
  \texttt{lin\_scl\_fine} & Chop the record into smaller (\appr\qtyrange{9}{21}{\metre}) sections and detrend linearly.\\
  \texttt{lin\_scl\_med} & Combination of fine and coarse, where we make the deeper part a single chunk.\\
  \texttt{gam\_scl} & Subtract a \gls{GAM} from the value, then scale the result.\\
  \texttt{loess\_scl} & Subtract a LOESS trend from the value, then scale the result.\\
  \texttt{lin\_gam\_scl} & First do \texttt{lin\_scl} from above, then do \texttt{gam\_det}.\\ % This is excessive so I removed it from the analysis.
  \texttt{lp\_scl} & Subtract a lowpass filter from the value, then scale the result.\\
\end{tabular}
\end{table}

\subsection{Spectral Analysis and Filtering}\label{sec:spectral}

We perform spectral analysis to verify the presence of Milankovitch cycles in the proxy records.
We compute and show the \gls{FFT}, % Blackman-Tukey (BT), % Lomb-Scargle (LS),
\gls{MTM}, and the \gls{MTLS} to illustrate the sensitivity of the spectral analysis to the analysis tool used.

We use bandpass filtering to extract the long and short eccentricity cycles (rectangular and Gaussian windows targeting \qtylist{405;110}{\kiloyear} periods, frequencies of \qtylist{0.00247;0.00909}{cycles\per\kiloyear}\textpm\qty{30}{\percent}, zero-padded with twice the number of datapoints).


\subsection{Finding the Best Fit}\label{sec:algorithm}

In order to test which astronomical solution best matches the data we adopt and extend the approach of \citeA{ZeebeLourens2019,ZeebeLourens2022EPSL}.
We started with the detrended data (see \cref{sec:detrend}) and the depths of the long eccentricity minima as identified in the field (\cref{sec:agemodel}) which resulted in an initial age model.
We then applied these agemodels and, after linearly interpolating to a timestep that is a multiple of the timestep of the astronomical solution, performed bandpass filtering targeting the long and short eccentricity cycles in the time domain.
The filtered signals were added and scaled (or normalized: we subtracted the mean and divided by the standard deviation) to construct an artificial ``eccentricity'' curve.
%% To study the effect of the two cycles on the match with the astronomical solution, we also looked at the effect of assigning different weights to the two filtered signals (weights of 1:0, 1:0.25, 1:0.5, 1:0.75, 1:1, 0.75:1, 0.5:1, 0.25:1, and 0:1 for the long and short eccentricity cycles respectively).
The scaled astronomical solutions were then interpolated to the same timesteps as the data, and the squared differences were summarized by calculating the \gls{RMSD}.

An age offset from the initial age model was found minimizing the \gls{RMSD} after offsetting the records by \qty{\pm200}{\kiloyear}.
In order to optimize the tiepoints---the long eccentricity minima depths as identified in the field (see \cref{sec:agemodel})---we iteratively shifted each tiepoint from the youngest to oldest by a range of values between \qtyrange[range-phrase=~to~]{-1.6}{1.6}{\metre} in \qty{20}{\centi\metre} increments (about \qty{10}{\percent} of the \qty{405}{\kiloyear} period).
After tweaking the tiepoint depth, the altered agemodel is applied, the record is filtered to target the long and short eccentricity cycles (see \cref{sec:spectral}), and an artificial eccentricity curve was created to calculate the \gls{RMSD} as before.
Once the optimal (lowest \gls{RMSD}) tiepoint depth was found, we fixed it and moved on to the next-youngest tiepoint.
This was repeated until all tiepoints had been adjusted, resulting in the overall best fit.

\section{Results}\label{sec:results}

\ijk{@REZ: i'm struggling with the order in which to put this, first all the details of sensitivity analysis and then main results, so that conclusions naturally flow from the results or first the main findings and then the nitty-gritty stuff (but then we're not able to describe the detrend types and such as nicely)}

\ijk{something about spectral analysis?}

% talk about all the nitty-gritty details of parameter perturbations?
% - proxy (show the same pattern?)
% - site (Zumaia vs. Sopelana) part of main text
% - depth_chunk (cut Zumaia into seperate sedimentation rate intervals
% - detrend_type
% - window (gaussian/rectangular)
% - comb (I only show 1:1, but 1:0.5 and 0.5:1 both outperform the rest)
We performed our analysis for the proxies \gls{L*}, \gls{d13C} and \gls{MS}, but  focus on \gls{L*} in the main paper.
The \gls{d13C} proxy captures lower frequency orbital forcing more strongly~\cite{Zeebe2017,Kocken2019loscar}.
For \gls{MS} it is unclear how it relates to the climate forcing. %Richard doesn't trust/understand it.
\gls{L*} likely corresponds to changes in the lithology that have been driven by carbonate--clay alternations, which are in turn driven by changes in wetness and climate.
We argue that the colour reflectance record most directly reflects the orbital forcing.

Different ways of detrending the proxy records in order to subtract non-cyclical patterns (\cref{fig:detrend}) lead to slightly different results (\cref{fig:rolling-rmsd-improvement}).
Furthermore, changing the parameters of the bandpass filter has a strong effect: when using a rectangular window or a gaussian window (both with the same upper- and lower boundaries, which means that a gaussian window effectively shrinks the filter range) results in different ``best'' solutions.
See \cref{fig:rolling-rmsd-improvement} for a comparison of \gls{RMSD} scores across all detrend types, bandpass filter windows, different proxies, and whether or not we performed our tiepoint-adjustments (\cref{sec:algorithm}).
See also \cref{fig:rolling-age-MS,fig:rolling-age-d13C} for the fits against the astronomical solutions in the time domain.

\begin{figure}[htbp]
  \centering
  \includegraphics[width=.9\linewidth]{Zumaia-Sopelana_detrend_types_all.pdf}
  \caption{\label{fig:detrend}
    \textbf{Zumaia and Sopelana trend removal strategies.}
    The raw data (gray) and the lines that were fit through (coloured lines), which were subtracted from the record. See \cref{tab:detrend_types} for descriptions.
    }
\end{figure}

% first describe simple "what is the best RMSD-scoring solution?"
\subsection{Comparison of RMSD scores}

% figure 2 from top to bottom (even though in the duscussion/conclusion I'll go young to old, since that makes more sense in selecting a solution
We show the best scores across all analyses in \cref{fig:rolling-rmsd}.
The full Zumaia record gives the lowest \gls{RMSD}-scores for the La10b (\num{1.055}) and ZB20b (\num{1.120}) solutions,
while the Sopelana site gives the best results for ZB20a (\num{0.872}) and ZB18a (\num{0.894}), and to a lesser extent ZB20c (\num{0.959}) and La10c (\num{0.968}).
For Zumaia we also analysed the record separately for two depth intervals to account for a change in sedimentation rate at around \qty{109.26}{\metre}.
The older interval shows a better match with the La10c (\num{0.981}) and ZB20b (1.001) solutions.
The younger interval has a slightly better match for La10b (\num{1.104}), ZB20a (\num{1.105}), and ZB20b (\num{1.116}), but results show similar \gls{RMSD} scores.

\begin{figure}[htb]
  \centering
  \includegraphics[width=0.7\textwidth]{rolling_age_overview_best.pdf}
  \caption{\label{fig:rolling-rmsd}
    \textbf{Best \gls{RMSD} scores of different astronomical solutions versus the Zumaia and Sopelana Maastrichtian records for \gls{L*}}
    Colours show which detrending strategy was used (see \cref{tab:detrend_types}), while circles indicate a gaussian filter and squares show a rectangular filter.
    All scores have adjusted the tie-point depths to arrive at the best match with each solution (\cref{sec:algorithm}).
    The two bottom panels show results for the upper and lower parts of the Zumaia record analyzed separately because of the change in sedimentation rate and lithology.
    % TODO: link to full figure
}
\end{figure}

\subsection{Very long eccentricity nodes}

When we look beyond just the \gls{RMSD} scores, more illuminating patterns emerge.
\cref{fig:rolling-depth-age} shows our filtered, normalized records tuned to the astronomical solutions, highlighting where the records match the solutions better or worse.
The key differentiating factors between astronomical solutions are the locations of the very long eccentricity nodes, where the amplitude of the short eccentricity cycle is very low and the long eccentricity cycle dominates.
Because of the filtering approach it is difficult to distinguish these very long eccentricity nodes in the filtered data directly.
There may be non-orbital reasons to get a low-amplitude in the filtered short eccentricity cycle.
We can, however, detect sections with a high amplitude in the short eccentricity cycle, indicating the absence of a very long eccentricity node.

\ijk{hmm already make this conclusion or first describe that it has low amplitude in the short ecc component here and put below in discussion/conclusion?}
The filtered records indicate that Ma405\(_{8}\) at around \qty{68.9}{\millionyearago} was probably not very long eccentricity node.
This is significant, because solutions La10b, ZB18a, and ZB20c have a low amplitude of short eccentricity here.
The same goes for Ma405\(_{10}\) at \qty{69.7}{\millionyearago}, which does not match La10c and ZB20b.


\begin{figure}[htb]
  \centering
  \includegraphics[width=1.1\textwidth]{Zumaia_Lstar_1-1_solutions_simple_with_log.pdf}
  \caption{\label{fig:rolling-depth-age}
    \textbf{Maastrichtian Zumaia and Sopelana filtered/normalized \gls{L*} records tuned to various astronomical solutions.}
    % This uses short linear detrending to correct for changes in sediment composition (\cref{sec:detrend}).
    We also show Zumaia above and below \qty{109.26}{\metre} to account for the change in the sedimentation rate and lithology.
    Colours represent the way in which the record was detrended (\cref{tab:detrend_types}).
    Numbers, opacity, and line thickness indicate \gls{RMSD} scores after optimizing the tiepoints (see \cref{sec:algorithm}) for the best fit.
    All other fits are shown as thin lines to illustrate the sensitivity of the analysis.
    The second-to-lowest panel shows detrended proxy records in the depth domain.
    Points show long eccentricity minima as identified in the field, with adjustments of up to \qty{\pm1.6}{\metre} (see \cref{sec:algorithm}).
    The bottom panel's log is adapted from \citeA{Batenburg2014}.
    }
\end{figure}




\section{Discussion}\label{sec:discussion}

This study shows some of the sensitivity of the different detrending/filtering techniques for astronomical calibration.
\ijk{some more stuff here before reaching the next bit:}

So which astronomical solution best matches the Zumaia and Sopelana records?
The youngest interval for Zumaia shows a slight preference for solutions La10b, ZB20a, and ZB20b \cref{fig:rolling-rmsd}.

For the deeper/older part of the Zumaia record, only La10c (and to a lesser extent La10b) and ZB20b have good scores because ZB20a does not have a very long eccentricity node at Ma405\(_{8}\) (\appr\qty{68.9}{\millionyearago}), whereas the record clearly does.
Solutions La10b and La10c showed a poorer match already in the late Cretaceous, where they did not show a very long eccentricity node at around \qty{61}{\millionyearago} whereas the Walvis Ridge data did \cite{ZeebeLourens2022EPSL}.
Then as we move to the older section the Sopelana data results in a best score for ZB20a because ZB20b has a very long eccentricity node at Ma405\(_{10}\) (\appr\qty{69.7}{\millionyearago}), whereas the amplitude of the record is large there.
Therefore, we prefer the ZB20b solution up to \appr\qty{69.2}{\millionyearago}.
For studies working with data older than this we must conclude that the data do not support any of the currently available solutions that are also consistent with data from the late Cretaceous.

Future studies could generate new astronomical solutions with similar initial settings to the ZB20b solution to see if any of them can reproduce the patterns in the \gls{L*} record of the Sopelana site.

%%

%  Numbered lines in equations:
%  To add line numbers to lines in equations,
%  \begin{linenomath*}
%  \begin{equation}
%  \end{equation}
%  \end{linenomath*}



%% Enter Figures and Tables near as possible to where they are first mentioned:
%
% DO NOT USE \psfrag or \subfigure commands.
%
% Figure captions go below the figure.
% Table titles go above tables;  other caption information
%  should be placed in last line of the table, using
% \multicolumn2l{$^a$ This is a table note.}
%
%----------------
% EXAMPLE FIGURES
%
% \begin{figure}
% \includegraphics{example.png}
% \caption{caption}
% \end{figure}
%
% Giving latex a width will help it to scale the figure properly. A simple trick is to use \textwidth. Try this if large figures run off the side of the page.
% \begin{figure}
% \noindent\includegraphics[width=\textwidth]{anothersample.png}
%\caption{caption}
%\label{pngfiguresample}
%\end{figure}
%
%
% If you get an error about an unknown bounding box, try specifying the width and height of the figure with the natwidth and natheight options. This is common when trying to add a PDF figure without pdflatex.
% \begin{figure}
% \noindent\includegraphics[natwidth=800px,natheight=600px]{samplefigure.pdf}
%\caption{caption}
%\label{pdffiguresample}
%\end{figure}
%
%
% PDFLatex does not seem to be able to process EPS figures. You may want to try the epstopdf package.
%

%
% ---------------
% EXAMPLE TABLE
% Please do NOT include vertical lines in tables
% if the paper is accepted, Wiley will replace vertical lines with white space
% the CLS file modifies table padding and vertical lines may not display well
%
 % \begin{table}
 % \caption{Time of the Transition Between Phase 1 and Phase 2$^{a}$}
 % \centering
 % \begin{tabular}{l c}
 % \hline
 %  Run  & Time (min)  \\
 % \hline
 %   $l1$  & 260   \\
 %   $l2$  & 300   \\
 %   $l3$  & 340   \\
 %   $h1$  & 270   \\
 %   $h2$  & 250   \\
 %   $h3$  & 380   \\
 %   $r1$  & 370   \\
 %   $r2$  & 390   \\
 % \hline
 % \multicolumn{2}{l}{$^{a}$Footnote text here.}
 % \end{tabular}
 % \end{table}

%% SIDEWAYS FIGURE and TABLE
% AGU prefers the use of {sidewaystable} over {landscapetable} as it causes fewer problems.
%
% \begin{sidewaysfigure}
% \includegraphics[width=20pc]{figsamp}
% \caption{caption here}
% \label{newfig}
% \end{sidewaysfigure}
%
%  \begin{sidewaystable}
%  \caption{Caption here}
% \label{tab:signif_gap_clos}
%  \begin{tabular}{ccc}
% one&two&three\\
% four&five&six
%  \end{tabular}
%  \end{sidewaystable}

%% If using numbered lines, please surround equations with \begin{linenomath*}...\end{linenomath*}
%\begin{linenomath*}
%\begin{equation}
%y|{f} \sim g(m, \sigma),
%\end{equation}
%\end{linenomath*}

%%% End of body of article

%%%%%%%%%%%%%%%%%%%%%%%%%%%%%%%%
%% Optional Appendix goes here
%
% The \appendix command resets counters and redefines section heads
%
% After typing \appendix
%
%\section{Here Is Appendix Title}
% will show
% A: Here Is Appendix Title
%
\appendix
% this template also turns the open research section into an appendix...

\section{Alternative Combinations and Proxies}

We show some more combinations of \cref{fig:rolling-rmsd-improvement} in \cref{fig:rolling-rmsd-improvement}.
We show an adaptation of \cref{fig:rolling-depth-age} for \gls{MS} (\cref{fig:rolling-age-MS}) and for \gls{d13C} (\cref{fig:rolling-age-d13C}).

\begin{figure}[htb]
  \centering \includegraphics[width=\textwidth]{Zumaia-Sopelana_mtm_raw.pdf}
  \caption{\label{fig:spectral-depth}
    \textbf{Spectral analysis in the depth domain.}
    % Do I need refs for all of these?
    \ijk{In the end probably show analysis only in time-domain?}
    BT = Blackman-Tukey,
    FFT = Fast Fourier Transform,
    LOWSPEC = Robust Locally-Weighted Regression Spectral Background Estimation \cite{Meyers2012},
    LS = Lomb-Scargle,
    MTLS = Multi-taper Averaged Lomb-Scargle periodogram of (un)evenly
spaced data \cite{Springford2020},
    MTM = Multitaper method \cite{Thomson1982}.
    Shaded intervals for the MTM and LOWSPEC indicate AR1 fit and AR1-power and LOWSPEC fit and power confidence levels.
  }
\end{figure}

\begin{figure}[htb]
  \centering \includegraphics[width=\textwidth]{Zumaia-Sopelana_mtm.pdf}
  \caption{\label{fig:spectral-depth}
    \textbf{Spectral analysis in the depth domain.}
    % Do I need refs for all of these?
    \ijk{This is the same as above but after linear detrending with \texttt{lin\_scl\_fine}.}
    BT = Blackman-Tukey,
    FFT = Fast Fourier Transform,
    LOWSPEC = Robust Locally-Weighted Regression Spectral Background Estimation \cite{Meyers2012},
    LS = Lomb-Scargle,
    MTLS = Multi-taper Averaged Lomb-Scargle periodogram of (un)evenly
spaced data \cite{Springford2020},
    MTM = Multitaper method \cite{Thomson1982}.
    Shaded intervals for the MTM and LOWSPEC indicate AR1 fit and AR1-power and LOWSPEC fit and power confidence levels.
  }
\end{figure}


\begin{figure}[htb]
  \centering \includegraphics[width=\textwidth]{Zumaia_Sopelana_spectra_filters_raw.pdf}
  \caption{\label{fig:spectral-age-raw}
    \textbf{Spectral analysis in the time domain.}
    % Do I need refs for all of these?
    \ijk{This is raw values, only linear detrend}
    % BT = Blackman-Tukey,
    FFT = Fast Fourier Transform,
    LOWSPEC = Robust Locally-Weighted Regression Spectral Background Estimation \cite{Meyers2012},
    % LS = Lomb-Scargle,
    MTLS = Multi-taper Averaged Lomb-Scargle periodogram of (un)evenly
spaced data \cite{Springford2020},
    MTM = Multitaper method \cite{Thomson1982}.
    Shaded intervals for the MTM and LOWSPEC indicate AR1 fit and AR1-power and LOWSPEC fit and power confidence levels.
  }
\end{figure}


\begin{figure}[htb]
  \centering \includegraphics[width=1.2\textwidth]{new_rolling_improvement.pdf}
  \caption{\label{fig:rolling-rmsd-improvement}
    More elaborate version of \cref{fig:rolling-rmsd} showing all proxies, different ways of detrending and filtering.}
\end{figure}


\begin{figure}[htb]
  \centering
  \includegraphics[width=1.2\textwidth]{Zumaia_MS_1-1_solutions_simple_with_log.pdf}
  \caption{\label{fig:rolling-age-MS}
    Same as \cref{fig:rolling-depth-age} but for \gls{MS}.}
\end{figure}

\begin{figure}[htb]
  \centering
  \includegraphics[width=0.9\textwidth]{Zumaia_d13C_1-1_solutions_simple_with_log.pdf}
  \caption{\label{fig:rolling-age-d13C}
    Same as \cref{fig:rolling-depth-age} but for \gls{d13C}.}
\end{figure}

%%%%%%%%%%%%%%%%%%%%%%%%%%%%%%%%%%%%%%%%%%%%%%%%%%%%%%%%%%%%%%%%
%
% Optional Glossary, Notation or Acronym section goes here:
%
%%%%%%%%%%%%%%
% Glossary is only allowed in Reviews of Geophysics
%  \begin{glossary}
%  \term{Term}
%   Term Definition here
%  \term{Term}
%   Term Definition here
%  \term{Term}
%   Term Definition here
%  \end{glossary}

%
%%%%%%%%%%%%%%
% Acronyms
%   \begin{acronyms}
%   \acro{Acronym}
%   Definition here
%   \acro{EMOS}
%   Ensemble model output statistics
%   \acro{ECMWF}
%   Centre for Medium-Range Weather Forecasts
%   \end{acronyms}

%
%%%%%%%%%%%%%%
% Notation
%   \begin{notation}
%   \notation{$a+b$} Notation Definition here
%   \notation{$e=mc^2$}
%   Equation in German-born physicist Albert Einstein's theory of special
%  relativity that showed that the increased relativistic mass ($m$) of a
%  body comes from the energy of motion of the body—that is, its kinetic
%  energy ($E$)—divided by the speed of light squared ($c^2$).
%   \end{notation}



\section*{Open Research}

\Gls{MS}, \gls{L*}, and \gls{d13C} data used in this study are from \citeA{Batenburg2012,Batenburg2012}.

Analysis was performed using the R programming language~\cite{RCoreTeam2020} and made use of \texttt{astrochron} \citeA{Meyers2014} and the \texttt{tidyverse} \citeA{Wickham2019}.
The new R package \texttt{AstronomicalSolutions} will be made available on publication on \url{https://github.com/japhir/AstronomicalSolutions} \citeA{Kocken2024}.

\textbf{TODO:} come up with nice package name (working title: \texttt{AstronomicalSolutions} so it's broad enough for future additions) and host on github/ archive on Zenodo.

% AGU requires an Availability Statement for the underlying data needed to understand, evaluate, and build upon the reported research at the time of peer review and publication.

% Authors should include an Availability Statement for the software that has a significant impact on the research. Details and templates are in the Availability Statement section of the Data and Software for Authors Guidance: \url{https://www.agu.org/Publish-with-AGU/Publish/Author-Resources/Data-and-Software-for-Authors#availability}

% It is important to cite individual datasets in this section and, and they must be included in your bibliography. Please use the type field in your bibtex file to specify the type of data cited. Some options include Dataset, Software, Collection, ComputationalNotebook. Ex:
% \\
% \begin{verbatim}

% @misc{https://doi.org/10.7283/633e-1497,
%   doi = {10.7283/633E-1497},
%   url = {https://www.unavco.org/data/doi/10.7283/633E-1497},
%   author = {de Zeeuw-van Dalfsen, Elske and Sleeman, Reinoud},
%   title = {KNMI Dutch Antilles GPS Network - SAB1-St_Johns_Saba_NA P.S.},
%   publisher = {UNAVCO, Inc.},
%   year = {2019},
%   type = {dataset}
% }

% \end{verbatim}

% For physical samples, use the IGSN persistent identifier, see the International Geo Sample Numbers section:
% \url{https://www.agu.org/Publish-with-AGU/Publish/Author-Resources/Data-and-Software-for-Authors#IGSN}
%%%%%%%%%%%%%%%%%%%%%%%%%%%%%%%%%%%%%%%%%%%%%%%

\acknowledgments
% This section is optional. Include any Acknowledgments here.
% The acknowledgments should list:\\
% All funding sources related to this work from all authors\\
% Any real or perceived financial conflicts of interests for any author\\
% Other affiliations for any author that may be perceived as having a conflict of interest with respect to the results of this paper.\\
% It is also the appropriate place to thank colleagues and other contributors. AGU does not normally allow dedications.

This work was supported by the Heising-Simons Foundation, under the CycloAstro
Cohort project 3.

%% ------------------------------------------------------------------------ %%
%% References and Citations

%%%%%%%%%%%%%%%%%%%%%%%%%%%%%%%%%%%%%%%%%%%%%%%
%
% \bibliography{<name of your .bib file>} don't specify the file extension
%
% don't specify bibliographystyle

% In the References section, cite the data/software described in the Availability Statement (this includes primary and processed data used for your research). For details on data/software citation as well as examples, see the Data & Software Citation section of the Data & Software for Authors guidance
% https://www.agu.org/Publish-with-AGU/Publish/Author-Resources/Data-and-Software-for-Authors#citation

%%%%%%%%%%%%%%%%%%%%%%%%%%%%%%%%%%%%%%%%%%%%%%%

\bibliography{references}


%Reference citation instructions and examples:
%
% Please use ONLY \cite and \citeA for reference citations.
% \cite for parenthetical references
% ...as shown in recent studies (Simpson et al., 2019)
% \citeA for in-text citations
% ...Simpson et al. (2019) have shown...
%
%
%...as shown by \citeA{jskilby}.
%...as shown by \citeA{lewin76}, \citeA{carson86}, \citeA{bartoldy02}, and \citeA{rinaldi03}.
%...has been shown \cite{jskilbye}.
%...has been shown \cite{lewin76,carson86,bartoldy02,rinaldi03}.
%... \cite <i.e.>[]{lewin76,carson86,bartoldy02,rinaldi03}.
%...has been shown by \cite <e.g.,>[and others]{lewin76}.
%
% apacite uses < > for prenotes and [ ] for postnotes
% DO NOT use other cite commands (e.g., \citet, \citep, \citeyear, \citealp, etc.).
% \nocite is okay to use to add references from your Supporting Information
%


\end{document}



% More Information and Advice:

%% ------------------------------------------------------------------------ %%
%
%  SECTION HEADS
%
%% ------------------------------------------------------------------------ %%

% Capitalize the first letter of each word (except for
% prepositions, conjunctions, and articles that are
% three or fewer letters).

% AGU follows standard outline style; therefore, there cannot be a section 1 without
% a section 2, or a section 2.3.1 without a section 2.3.2.
% Please make sure your section numbers are balanced.
% ---------------
% Level 1 head
%
% Use the \section{} command to identify level 1 heads;
% type the appropriate head wording between the curly
% brackets, as shown below.
%
%An example:
%\section{Level 1 Head: Introduction}
%
% ---------------
% Level 2 head
%
% Use the \subsection{} command to identify level 2 heads.
%An example:
%\subsection{Level 2 Head}
%
% ---------------
% Level 3 head
%
% Use the \subsubsection{} command to identify level 3 heads
%An example:
%\subsubsection{Level 3 Head}
%
%---------------
% Level 4 head
%
% Use the \subsubsubsection{} command to identify level 3 heads
% An example:
%\subsubsubsection{Level 4 Head} An example.
%
%% ------------------------------------------------------------------------ %%
%
%  IN-TEXT LISTS
%
%% ------------------------------------------------------------------------ %%
%
% Do not use bulleted lists; enumerated lists are okay.
% \begin{enumerate}
% \item
% \item
% \item
% \end{enumerate}
%
%% ------------------------------------------------------------------------ %%
%
%  EQUATIONS
%
%% ------------------------------------------------------------------------ %%

% Single-line equations are centered.
% Equation arrays will appear left-aligned.

% Math coded inside display math mode \[ ...\]
%  will not be numbered, e.g.,:
%  \[ x^2=y^2 + z^2\]

%  Math coded inside \begin{equation} and \end{equation} will
%  be automatically numbered, e.g.,:
%  \begin{equation}
%  x^2=y^2 + z^2
%  \end{equation}


% % To create multiline equations, use the
% % \begin{eqnarray} and \end{eqnarray} environment
% % as demonstrated below.
% \begin{eqnarray}
%   x_{1} & = & (x - x_{0}) \cos \Theta \nonumber \\
%         && + (y - y_{0}) \sin \Theta  \nonumber \\
%   y_{1} & = & -(x - x_{0}) \sin \Theta \nonumber \\
%         && + (y - y_{0}) \cos \Theta.
% \end{eqnarray}

%If you don't want an equation number, use the star form:
%\begin{eqnarray*}...\end{eqnarray*}

% Break each line at a sign of operation
% (+, -, etc.) if possible, with the sign of operation
% on the new line.

% Indent second and subsequent lines to align with
% the first character following the equal sign on the
% first line.

% Use an \hspace{} command to insert horizontal space
% into your equation if necessary. Place an appropriate
% unit of measure between the curly braces, e.g.
% \hspace{1in}; you may have to experiment to achieve
% the correct amount of space.


%% ------------------------------------------------------------------------ %%
%
%  EQUATION NUMBERING: COUNTER
%
%% ------------------------------------------------------------------------ %%

% You may change equation numbering by resetting
% the equation counter or by explicitly numbering
% an equation.

% To explicitly number an equation, type \eqnum{}
% (with the desired number between the brackets)
% after the \begin{equation} or \begin{eqnarray}
% command.  The \eqnum{} command will affect only
% the equation it appears with; LaTeX will number
% any equations appearing later in the manuscript
% according to the equation counter.
%

% If you have a multiline equation that needs only
% one equation number, use a \nonumber command in
% front of the double backslashes (\\) as shown in
% the multiline equation above.

% If you are using line numbers, remember to surround
% equations with \begin{linenomath*}...\end{linenomath*}

%  To add line numbers to lines in equations:
%  \begin{linenomath*}
%  \begin{equation}
%  \end{equation}
%  \end{linenomath*}
