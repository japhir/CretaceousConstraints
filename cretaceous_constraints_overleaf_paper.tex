%%%%%%%%%%%%%%%%%%%%%%%%%%%%%%%%%%%%%%%%%%%%%%%%%%%%%%%%%%%%%%%%%%%%%%%%%%%%
% adapted from AGUJournalTemplate.tex: this template file is for articles formatted with LaTeX
%
%% To submit your paper:
\documentclass[draft]{agujournal2019}
% packages from the template
\usepackage{url} %this package should fix any errors with URLs in refs.
\usepackage{lineno}
\usepackage[inline]{trackchanges} %for better track changes. finalnew option will compile document with changes incorporated.
\usepackage{soul}

% IJK added packages
% for units, type \qty{x}{\unit}
\usepackage{siunitx}
% for chemical equations or, in my case, d13C
\usepackage[version=4]{mhchem}
%\usepackage{chemformula} % alternative for above
% \usepackage[utf8]{luainputenc} % allow UTF-8 input? δ13C?
% \usepackage{hyperref} % links between references etc % doesn't seem to work
% with this template?
\usepackage[capitalise,nameinlink,noabbrev]{cleveref}
% use same macro for acronym, this writes out the first use and then
% abbreviates after (use \gls{key}).
\usepackage{glossaries}
%\makeglossaries % not needed if we don't want to print them at the end
\usepackage[section]{placeins} % figures were jumping to the end, I prefer them closer

\usepackage{threeparttable} % for table footnotes

\newcommand{\appr}{\raise.17ex\hbox{\(\scriptstyle\sim\)}} % approximately symbol
\newcommand{\ma}[1]{Ma\(_{405}\)#1} % Ma\(_{405}\)8 was a hassle to type

\newcommand{\rez}{\textcolor{magenta}}
% I'll use this to highlight sections that I want to focus your attention on!
\newcommand{\ijk}{\textcolor{blue}}

% write Hawaiʻi correctly
% https://tex.stackexchange.com/questions/424535/how-to-type-a-proper-hawai%CA%BBian-%CA%BBokina
\DeclareRobustCommand{\okina}{%
  \raisebox{\dimexpr\fontcharht\font`A-\height}{%
    \scalebox{0.8}{`}%
  }%
}

% commented this out b/c they don't use lualatex or xelatex, so can't use UTF-8 directly (?)
%\newunicodechar{ʻ}{\okina}

% normally I strongly prefer biblatex, but the AGU template allows us to use \cite,
% \citeA, and \nocite from apacite :(
% \usepackage[giveninits=true,uniquename=false,uniquelist=false,date=year,hyperref=true,mincitenames=1,maxcitenames=2,backend=biber,backref,doi=true,url=false,isbn=false]{biblatex}
% \addbibresource{references.bib}

% IJK package config
\sisetup{%
detect-all,% Detect surrounding font context, like weight, italics etc.
%
% Alternative range-phrase:
% en-dash via '--', but inside \text{}, so it's not 'two minus signs'
range-phrase={\,\text{--}\,},
separate-uncertainty=true,
multi-part-units=single,
list-units=single,% single: Print unit only once, at end
range-units=single,% single: Print unit only once, at end
per-mode=symbol,
}%
\DeclareSIUnit\annum{a} % a year, used in years before present
\DeclareSIUnit\year{yr} % a year as a duration
\DeclareSIUnit\millionyearago{\mega\annum} % a time, e.g. 34 million years before present
\DeclareSIUnit\millionyear{\mega\year} % a duration, e.g. the event lasted for 2 million years
\DeclareSIUnit\kiloyearago{\kilo\annum} % a time, e.g. 14 thousand years ago (before present)
\DeclareSIUnit\kiloyear{\kilo\year} % a duration, e.g. the event lasted 200 thousand years

% this makes it so the first use spells it out, the next uses the abbreviation!
\newacronym{d13C}{\ensuremath{\delta}\ce{^13C}}{carbon isotope ratio}
\newacronym{MS}{MS}{magnetic susceptibility}
\newacronym{L*}{\(L^*\)}{total light reflectance}

\newacronym{RMSD}{RMSD}{root mean square deviation}
% do we need RCSSD?
\newacronym{VLN}{VLN}{very long eccentricity node}
% TODO: we may need a SEC and LEC for short eccentricity and long eccentricity? Or maybe e_{l} and e_{s}

\newacronym{KT}{KTB}{Cretaceous--Paleogene boundary}
\newacronym{PETM}{PETM}{Paleocene--Eocene thermal maximum}

\newacronym{GAM}{GAM}{generalized additive model}
\newacronym{FFT}{FFT}{fast Fourier transform}
\newacronym{MTM}{MTM}{multi-taper method}
\newacronym{MTLS}{MTLS}{multi-taper averaged Lomb-Scargle periodogram of (un)evenly spaced data}

\linenumbers
%%%%%%%
% As of 2018 we recommend use of the TrackChanges package to mark revisions.
% The trackchanges package adds five new LaTeX commands:
%
%  \note[editor]{The note}
%  \annote[editor]{Text to annotate}{The note}
%  \add[editor]{Text to add}
%  \remove[editor]{Text to remove}
%  \change[editor]{Text to remove}{Text to add}
%
% complete documentation is here: http://trackchanges.sourceforge.net/
%%%%%%%

% \draftfalse

%% Enter journal name below.
%% Choose from this list of Journals:
%
% JGR: Atmospheres
% JGR: Biogeosciences
% JGR: Earth Surface
% JGR: Oceans
% JGR: Planets
% JGR: Solid Earth
% JGR: Space Physics
% Global Biogeochemical Cycles
% Geophysical Research Letters
% Paleoceanography and Paleoclimatology
% Radio Science
% Reviews of Geophysics
% Tectonics
% Space Weather
% Water Resources Research
% Geochemistry, Geophysics, Geosystems
% Journal of Advances in Modeling Earth Systems (JAMES)
% Earth's Future
% Earth and Space Science
% Geohealth
%
% ie, \journalname{Water Resources Research}

\journalname{Paleoceanography and Paleoclimatology}


\begin{document}

%% ------------------------------------------------------------------------ %%
%  Title
%
% (A title should be specific, informative, and brief. Use
% abbreviations only if they are defined in the abstract. Titles that
% start with general keywords then specific terms are optimized in
% searches)
%
%% ------------------------------------------------------------------------ %%

% previously: Cretaceous Constraints on Astronomical Solutions
\title{An Astronomically Tuned Time Scale for the Latest Cretaceous}
% technically, Maastrichtian would be more appropriate but that doesn't alliterate ;-)

%% ------------------------------------------------------------------------ %%
%
%  AUTHORS AND AFFILIATIONS
%
%% ------------------------------------------------------------------------ %%

% List authors by first name or initial followed by last name and
% separated by commas. Use \affil{} to number affiliations, and
% \thanks{} for author notes.
% Additional author notes should be indicated with \thanks{} (for
% example, for current addresses).

% Example: \authors{A. B. Author\affil{1}\thanks{Current address, Antartica}, B. C. Author\affil{2,3}, and D. E.
% Author\affil{3,4}\thanks{Also funded by Monsanto.}}

\authors{I. J. Kocken\affil{1} and R. E. Zeebe\affil{1}}
\affiliation{1}{
School of Ocean and Earth Science and Technology,
University of Hawai\okina{}i at M\=anoa,
1000 Pope Road, MSB 629, Honolulu, HI 96822, USA}

% \affiliation{2}{Second Affiliation}
%(repeat as many times as is necessary)

%% Corresponding Author:
% Corresponding author mailing address and e-mail address:

% (include name and email addresses of the corresponding author.  More
% than one corresponding author is allowed in this LaTeX file and for
% publication; but only one corresponding author is allowed in our
% editorial system.)

\correspondingauthor{Ilja Kocken}{ikocken@hawaii.edu}

%% Keypoints, final entry on title page.

%  List up to three key points (at least one is required)
%  Key Points summarize the main points and conclusions of the article
%  Each must be 140 characters or fewer with no special characters or punctuation and must be complete sentences

% Example:
% \begin{keypoints}
% \item	List up to three key points (at least one is required)
% \item	Key Points summarize the main points and conclusions of the article
% \item	Each must be 140 characters or fewer with no special characters or punctuation and must be complete sentences
% \end{keypoints}

\begin{keypoints}
% I don't particularly like these yet...
\item Geological data can be used to constrain the chaos of the solar system to understand Earth's orbital history.
\item The Zumaia and Sopelana sections are most compatible with solution ZB20a for the Maastrichtian.
\item More records of higher quality that capture variability in short- and long eccentricity are needed.
% meh, don't like 3.
\end{keypoints}

%% ------------------------------------------------------------------------ %%
%
%  ABSTRACT and PLAIN LANGUAGE SUMMARY
%
% A good Abstract will begin with a short description of the problem
% being addressed, briefly describe the new data or analyses, then
% briefly states the main conclusion(s) and how they are supported and
% uncertainties.

% The Plain Language Summary should be written for a broad audience,
% including journalists and the science-interested public, that will not have
% a background in your field.
%
% A Plain Language Summary is required in GRL, JGR: Planets, JGR: Biogeosciences,
% JGR: Oceans, G-Cubed, Reviews of Geophysics, and JAMES.
% see http://sharingscience.agu.org/creating-plain-language-summary/)
%
%% ------------------------------------------------------------------------ %%

%% \begin{abstract} starts the second page

\begin{abstract}
% this is a first draft!
%Problem
Astronomical solutions form the backbone of accurate dating for geology and paleoclimate studies.
Beyond \appr\qty{50}{\millionyearago}, however, the chaos inherent in the solar system makes it impossible to calculate a single unique astronomical solution.
Geological data have been used to constrain this chaos in order to arrive at a geologic time scale up to the end-Cretaceous.
%Action
Here, we adopt and extend this approach into the latest Cretaceous, by re-analyzing the Zumaia and Sopelana composite proxy records from the Maastrichtian.
%Results
We find that the filtered \gls{L*} record is most compatible with the astronomical solution ZB20a.
However, these results are sensitive to parameter choices in our algorithm, which we describe in detail.
Nevertheless, \acrlongpl{VLN} in the astronomical solutions that coincide with large amplitudes in the short eccentricity-related peaks in the filtered proxy record give a clear indication on why the ZB20a solution should be favored when performing cyclostratigraphy for the latest Cretaceous.
% decided to not focus on the \gls{KT} age for now.
\end{abstract}

\section*{Plain Language Summary}
[ enter your Plain Language Summary here or delete this section]

%%% Suggested section heads:
% \section{Introduction}
%
% The main text should start with an introduction. Except for short
% manuscripts (such as comments and replies), the text should be divided
% into sections, each with its own heading.

% Headings should be sentence fragments and do not begin with a
% lowercase letter or number. Examples of good headings are:

% \section{Materials and Methods}
% Here is text on Materials and Methods.
%
% \subsection{A descriptive heading about methods}
% More about Methods.
%
% \section{Data} (Or section title might be a descriptive heading about data)
%
% \section{Results} (Or section title might be a descriptive heading about the
% results)
%
% \section{Conclusions}

%%% IJK some notes on how I write special symbols and units here:

% how do I want to write d13C?
% \(\delta^{13}\)C % plain
% \ce{\delta^13C} % mhchem
% \(\delta\)\ch{^{13}C} % chemformula
% \gls{d13C} % using glossaries and mhchem  <-----

% test my acronyms
% \Gls{MS}
% \Gls{d13C}
% \Gls{L*}
% \Gls{KT} boundary

% test my units. Define them the first time we use them too.
% \qty{5}{\kiloyear}
% \qtylist{2;3;5}{\kiloyear}
% \qtyrange{2}{35}{\kiloyear}
% \qty{66}{\millionyearago}


\section{Introduction}\label{sec:intro}
%\ijk{I've focused on the figures, methods, results, and discussion first. In that order.}

% Astronomical solutions have improved dating, form the backbone of cyclostratigraphy,
Since the breakthrough that proved the astronomical theory of the ice ages~\cite{Hays1976},
cyclostratigraphy and astrochronology have become essential components of the youngest part of the Geologic Time Scale~\cite<e.g.,>[]{Hilgen2006,Laskar2020}.
The aim of astrochronology is to link quasi-periodic signals in geological data to astronomical solutions, providing absolute ages with high precision.
These precise ages allow us to understand the causality of climate events, which ultimately give insights into the mechanisms that drive changes in Earth's climate.

% but CHAOS
% @REZ: Wikipedia has Milankovitch forcing, but write his name like below.
While many of the prominent Milankovi\'{c} cycles affect Earth's climate in a repeating manner over the past tens of millions of years (\si{\millionyear}),
\ijk{[@REZ: can you help me with phrasing this sentence?]}
the calculations of the astronomical solutions result in different outcomes with minor perturbations in initial conditions or parameter values at around \num{50} million years ago (\si{\millionyearago})~\cite<e.g.,>[]{Laskar2004,Laskar2011,Zeebe2017}. % what should I cite here? this is just quick 'n dirty.
This chaos is inherent to the solar system, so this uncertainty cannot be resolved by improvements in the astronomical calculations alone.

% how does the chaos manifest in the orbital solutions?
How the chaos manifests in the different astronomical solutions, is expressed in the phasing of the different cycles.
The short eccentricity cycle (periods of \qtylist{95;99.7;124;132.5}{\kiloyear} for \qtyrange{56}{80}{\millionyearago})
\rez{should be ca. 95,99,124,131} \ijk{see email}
is variable between different astronomical solutions due to the influence of \(g_{4}\) and \(g_{3}\),
``which are loosely related to the perihelion precession of Earth's and Mars' orbits''~\cite{ZeebeLourens2022EPSL}.
% importance: different solutions have different phases of even 405 kyr cycle!
\ijk{[Do we need abbreviations for long and short eccentricity? Like \(e_{l}\) and \(e_{s}\) or SEC and LEC?]}
The long eccentricity cycle (period of \qty{405}{\kiloyear}), on the other hand, is thought to be stable over much longer time scales.

Many studies have suggested that because the long eccentricity cycle is stable in the geologic past, we can rely on it for tuning records older than \appr\qty{40}{\millionyearago}~\cite<i.e.>{Westerhold2012,Dinares-Turell2013}.
\ijk{[Hmm... they actually have a much more nuanced take...]}
\ijk{[@REZ: I think this next section feels a bit longwinded... Can you help me trim it down?]}
However, if we compare different astronomical solutions we observe that the phasing of this \qty{405}{\kiloyear} cycle does shift when we go further back in time.
% taner filter \qty{\frac{1}{405}}{\per\kiloyear} \pm 25%
% calculated this myself
For example, the long eccentricity minimum near \qty{65.9}{\millionyearago} differs by up to \qty{70}{\kiloyear} between solutions (largest difference between La11~\cite{Laskar2011,Laskar2011a} and ZB20a~\cite{ZeebeLourens2022EPSL}, second-largest difference of \qty{34}{\kiloyear} between La10c and ZB20a).
At the long eccentricity minimum around \qty{69.8}{\millionyearago}
%between \ma{13} and \ma{14} (numbers represent the number of the long eccentricity bundle older than the \gls{KT}),
the age of the \qty{405}{\kiloyear} minimum already differs by up to \qty{85.2}{\kiloyear} between different astronomical solutions---between astronomical solutions La10c~\cite{Laskar2011} and ZB20a~\cite{ZeebeLourens2022EPSL}.
This means that providing absolute ages of Maastrichtian records based on tuning to the long eccentricity cycle of any particular astronomical solution %without first constraining the chaos in this interval with geological records,
may underestimate the true uncertainty.

One further key differentiating factor between different astronomical solutions is the timing of \acrfullpl{VLN}.
When the individual components that make up the short eccentricity cycle interfere to cancel each other out, only the long eccentricity component remains clearly visible~\cite<i.e.>[]{ZeebeLourens2022EPSL}.
Chaotic resonance transitions can change the timing of these \acrshortpl{VLN}~\cite{ZeebeLourens2019}.
%The implications for this when comparing absolute astronomical ages to Ar/Ar and Ur/Pb ages \ijk{TODO}.

% constrain the chaos with geologic data
Previous workers have attempted to put geological constraints on this chaos for the Paleocene and Eocene~\cite<>[respectively]{ZeebeLourens2019,ZeebeLourens2022EPSL}.
These studies shed light on the occurrence of chaotic resonance transitions that shift the timing of \acrshortpl{VLN} and long eccentricity minima.
\ijk{[@REZ: I had added \cite{Westerhold2017} to the above. They make the comparison between data and several solutions and comment on implications, but it's not so explicit. Does it belong there?]}
\ijk{[Not sure if the next two comments about record quality are needed here.]}
The Walvis Ridge records used in these studies are of exceptional quality.
Thus finding comparable records that capture precession-scale alternations and patterns of amplitude modulation for older time periods is challenging.

In this work, we use the best available records immediately prior to the \gls{KT} (informal use) to constrain which astronomical solutions most accurately reflect Earth's orbit.
%for the Maastrichtian (data falls between \qtyrange{71.1}{65.9}{\millionyearago}).
We argue that these records are the Zumaia and Sopelana proxy records generated by \citeA{Batenburg2012,Batenburg2014} (see \cref{sec:data}).
These study sites have been previously recommended to serve as primary candidates to help improve astronomical solutions~\cite{Dinares-Turell2013}.
Ultimately, this study works towards providing an astronomical time scale for applications in geology and paleoclimate in the Maastrichtian.


\section{Material and Methods}\label{sec:mm}

%% \ijk{\textbf{TODO:} Decide on consistent naming for
%% scaled vs.\
%% normalized vs.\
%% standardized:
%% subtract mean, divide by standard deviation.
%% I like scaled}

\subsection{Datasets Used}\label{sec:data}

In order to find the most suitable Maastrichtian proxy record that best reflects amplitude variations in long- and short eccentricity,
we have considered several coastal sections of marine successions and sea floor drill core sites.
Proxy records from the following sites and studies were considered:
Gubbio~\cite{Voigt2012,Sinnesael2016}, % Fig. 12, d13C low-res
ODP Leg 208 Site 1267~\cite{Westerhold2008,Husson2011}, % log(Fe) only Ma405-1 and Ma405-2, MS up to Ma405-6 but no splice
IODP Leg 342 Site U1403~\cite{Batenburg2018}, % why didn't we look at this in more detail again? It is one of the studies listed in \cite{Smith2023}.
Zumaia and Sopelana~\cite{tenKateSprenger1993,Batenburg2012,Batenburg2014,Dinares-Turell2013}, % <3
ODP Leg 198 Site 1210B~\cite{Jung2012,Kim2022}, % longer record but lower resolution and d13C. Also cited in Voigt2012
ODP Leg 74 Site 525A~\cite{Husson2011}, % Fig. 3 grey levels smoothed, Ma405-6--9, Fig. 4 only Ma405-1
ODP Leg 207 Site 1258A~\citeA{Husson2011}, % Fig 3. MS Ma405-8 to 14, some core gaps (also at Ma10 unfortunately)
and ODP Leg 122 Site 762C~\citeA{Husson2011,Thibault2012}. % Fig 3. grey levels smoothed \ma{8--16}, Fig. 4 \m{1--6}
For the present study, we selected the Zumaia and Sopelana I field sections,
which show the clearest orbitally-forced patterns for the Maastrichtian~\cite{tenKateSprenger1993,Batenburg2012,Dinares-Turell2013},
and continuous high-resolution proxy records~\cite{Batenburg2012,Batenburg2014}.

The coastal cliffs of Zumaia and Sopelana are located in the Bay of Biscay in the Basque Country, Northern Spain~\cite{Herm1965,Pujalte1993}
%\ijk{NOTE: I tried to go through the literature trail thoroughly to arrive at the first study on this site, but most of these older papers are very hard to access!}
and comprise quasi-periodic alternations between limestones and marls, interbedded with turbidites~\cite{tenKateSprenger1993,Pujalte1999}. % this is cited as Pujalte 1998 in Batenburg2012, year off-by-one!
\citeA{Dinares-Turell2013} concluded that these successions could play ``a primary role as a geologic aid for critical developments of the astronomical solutions''.
% does the below add anything? unfortunately the quote is also quoted on wikipedia...
The Zumaia section is one of the first 100 geological heritage sites, which was described as ``One of the best exposed, most continuous and highly studied outcrops of deep marine sediments in the world'' \cite[pp.~71--72]{Hilario2022}.
See \citeA{Batenburg2014} and references therein for some historical context of the study sites. % this has better refs than their 2012 paper I think.
% quote from Batenburg2014: (Herm 1965; Percival & Fischer 1977; Lamolda 1990; Ward et al. 1991; Ten Kate & Sprenger 1993; Ward & Kennedy, 1993; elorza & García-Garmilla 1998; Pujalte et al. 1998; Dinarès-Turell et al. 2003, 2013; Gómez-Alday et al. 2008; Kuiper et al. 2008) and a reference section for the K–Pg boundary (Molina et al. 2009).
% they also say ten Kate & Sprenger 1993 first cyclostrat paper of the site.

Care must be taken when considering the lower Maastrichtian at the Zumaia site due to frequent turbidite successions,
while the upper Maastrichtian was tectonically more stable~\cite{Pujalte1998}.
\ijk{[@rez: can I delete these remarks? feel free to do so if you agree]}
\rez{[let's make sure the following text is not too close to
the original refs]}
\ijk{I've looked at the original refs again. It was obviously never copy-pasted, but looking at it again it \emph{was} mostly reshuffled text. I am not sure how close I can be to the original text that I cite/if I need to incorporate other stuff...}
The upper Maastrichtian section of Zumaia also contains many thin turbidites,
whose occurrence shows many random spectral peaks, but may also include an eccentricity and obliquity signal for cycles \ma{1--4}~\cite<the first four long eccentricity bundles prior to the \gls{KT},>{tenKateSprenger1993}. % this part is not mentioned in Batenburg
However, the pattern of limestone--marl alternations in the Zumaia succession that has been associated with astronomical forcing remained undisturbed by the turbidites in the upper Maastrichtian~\cite{Batenburg2012}.
The Sopelana section covers the lower part of the Maastrichtian, and has a similar lithology but without indications for turbidites, likely because it was located farther offshore~\cite{Batenburg2012}.
The proxy record from Zumaia~\cite{Batenburg2012} was extended to include Sopelana in \citeA{Batenburg2014}.

% this is basically a summary of Batenburg2012 table 1
The alternations in color and bed resistance that occurred at periods of \appr\qty{70}{\cm} were logged in detail~\cite{Batenburg2012,Dinares-Turell2013}
and have been associated with climatic precession forcing (periods of \appr\qtyrange{19}{23}{\kiloyear}).
Bundles of 5 of these couplets with darker, redder, thicker beds---especially in the marls---were related to short eccentricity amplitude modulation (\appr\qty{4}{\metre}, \appr\qty{100}{\kiloyear}),
which in turn were grouped in 4 bundles that were associated with the long eccentricity cycle (\appr\qty{16}{\metre}, \appr\qty{405}{\kiloyear}) and expressed as limited thinner marlier parts~\cite{Batenburg2012,Batenburg2014}.
\ijk{[TODO: I don't like the wording in the above]}
Eccentricity modulates the amplitude of climatic precession,
with eccentricity maxima and strong precession minima linked to enhanced seasonal contrast in the Northern hemisphere.
This is thought to intensify the hydrological cycle in the region, resulting in the deposition of thicker, darker marls~\cite{Batenburg2014}.
\ijk{[I'm citing Batenburg2012 and Batenburg2014 A LOT, maybe we need a simple acronym for them? B12 and B14?]}

\citeA{Batenburg2012,Batenburg2014} measured the \gls{MS}, the \gls{d13C}, and the \gls{L*} of rock samples (among others), but we focus on \gls{L*} in the main manuscript.
The \gls{d13C} proxy showed clear long eccentricity-related cycles but the short eccentricity was less clearly expressed,
which follows from carbon isotopes being more sensitive to lower frequencies in orbital forcing due to its long residence time in the ocean~\cite{Zeebe2017,Kocken2019loscar}.
% Richard doesn't trust/understand MS.
The \gls{MS} signal is affected by dilution with carbonates~\cite{tenKateSprenger1993},
so its link to climate forcing is less direct than, for example, the \ce{CaCO3} content itself.
% it seems that the \citeA{tenKateSprenger1993} record has a _--_ pattern for the amplitudes of the short ecc in the first 4 long ecc cycles prior to the K/T
% \citeA{Dinares-Turell2013} re-analyzed this record and did some additional spectral analysis/filtering.
The \gls{MS} proxy shows a very strong negative correlation with \ce{CaCO3} in bundles \ma{1--4}~\cite{tenKateSprenger1993,Gilabert2022}.
However, clear differences were found between the intercept of this relationship before, during, and after the \gls{KT}~\cite<supplementary figure S2 in>{Gilaberg2022}.
Unfortunately, the \ce{CaCO3} record from \citeA{tenKateSprenger1993} currently spans bundles \ma{1--4} and has not yet been extended to encompass the full Maastrichtian.
Furthermore, the original study reports that their \gls{MS} values show scatter, potentially due to ``surface irregularities and the generally low values''~\cite{Batenburg2012}.
The \gls{L*} proxy, on the other hand, likely corresponds to changes in the lithology that were driven by carbonate--clay alternations~\cite{MountWard1986,Batenburg2012},
which were in turn driven by changes in wetness and climate.
We therefore argue that the color reflectance \gls{L*} record most directly reflects the orbital forcing.

% commented out now, just put the relevant refs up above!
% \ijk{Just went through the table again, 1267 still looks promising and so does U1403.}
% \begin{table}
%     \centering
%     \caption{\label{tab:sites}
%         Overview of sites considered for study, sorted from younger to older, shorter to longer.
%         Gubbio consists of Contessa Highway (Danian) and Bottaccione (Maastrichtian).
%         \ijk{Rough first draft of table! See \TeX{} file for notes in comments.}
%     }
%     \begin{tabular}{llccl}
%         Site & Proxy & Ma\textsubscript{405}x & Age (Ma) & Reference \\
%         % the ages and Ma405 cycles are pretty rough estimates from figures/raw data
%         \hline
%         %208-1262 & a*, d13C, d18O & - & 58--53 & \citeA{ZeebeLourens2019} \\
%         % should probably cite original study but won't show in final table here anyway
%         %198-1209 & a*/b* & - & 66--56 Ma & \citeA{ZeebeLourens2022EPSL}\\
%         %Hendaye & section photos & - & 66--64 & Hilgen, pers. comm.\\
%         % Frits says it has a 200 kyr cycle that is unique to this interval!
%         Gubbio & \gls{d13C}, MS & 1--12 & 76--66 & \citeA{Voigt2012}\\
%         % too low-res? Contrasts to La10d
%         % Gubbio & Ca isotopes & - & OAE-2 & \citeA{Kitch2022}\\ % low-res, old, wrong proxy
%         Gubbio & MS, \ce{CaCO3}, d13C, d18O & - & 67--62.5 & \citeA{Sinnesael2016} \\
%         % bottaccione and contessa highway
%         % short, but Zumaia doesn't have much signal here
%         % may be worth another look! Contrasts to La11
%         208-1267 & color & 1--6/7 & 58--53 & \citeA{Westerhold2007} \\
%         % a* in \citeA{Westerhold2007}, mostly focused on PETM--Elmo
%         % extended in \cite{Batenburg2018} Newsletters on Strat. (56.042 Ma to 69.070 Ma).
%         % also related: [cite:@Zachos2004]
%         208-1267 & ln(Fe) & 1--2 & 66.5--55.0 & \citeA{Westerhold2008} \\ % very very strong short cycle-- I'm not seeing this anymore.
%         208-1267 & MS & 1--6/7 & 68.6--66 & \citeA{Husson2011} fig. 3 and 4. \\
%         % raw MS data from Blum2005 https://doi.pangaea.de/10.1594/PANGAEA.266605
%         % Husson2011 uses MBSF but need to use RMCD -> core gaps etc.!
%         % splice table from \cite{Westerhold2008} PDF https://doi.pangaea.de/10.1594/PANGAEA.592301
%         % I spent some time reworking this!
%         %% 208-1267 & \ce{CaCO3} & - & 50-47.8 & \citeA{Sexton2011}\\ % too young
%         342-U1403 & MS, ln(Fe/Ca) & 1--7 & 68.8--66 & \citeA{Batenburg2018}\\
%         % not sure why I didn't pick the one above
%         Zumaia & \ce{CaCO3} & \(-\)3--3 & 67.5--65.5 & \citeA{tenKateSprenger1993}\\ % too short?
%         % also shown in \cite{Westerhold2008}, next to 1267
%         Zumaia & L*, MS, d13C & 1--9 & 70--66 & \citeA{Batenburg2012} \\
%         Sopelana & L*, MS & 9--13 & 71.1--69.6 & \citeA{Batenburg2014}\\
%         198-1210B & d13C, d18O, XRF Ba & 2--12 & 71.5--66.25 & \citeA{Jung2012,Kim2022}\\
%         % some low-res intervals in crucial parts?
%         % higher res around 68 Ma, Kim2022 pretty high res but only d13C
%         74-525A & grayscale & 6--8.5 & 69--67.8 & \citeA{Husson2011}\\
%         % have emailed requesting data, but has left academia
%         122-762C & grayscale & 8--17+ & 77.5--69 & \citeA{Husson2011,Thibault2012}\\
%         % Thibault has published most data and sent me some upon request!
%         % the above may be useful to look at:
%         % some core gaps but high amplitude short ecc in Ma405_8 and Ma405_10!
%         207-1258A & MS & 8--14 & - & \citeA{Husson2011}\\
%         % raw data not available, but can reproduce by getting data from JANUS
%         % database
%         % taner filters not so nice, some big core gaps
%         % Gubbio & d13C, d18O & - & 84.2--72.1 & \citeA{Voigt2012}\\
%         % Gubbio & d13C & - & 84.2--72.1 & \citeA{Sabatino2018}\\ % high-res
%         % Furlo & color, d13C, d18O & - & 96--90 & \citeA{Batenburg2016}\\ % too old for us, but cool for future?
%         % Levant Platform & TOC, d13C, d18O & - & 96.2--90.9 & \citeA{Wendler2014}\\
%     \end{tabular}
% \end{table}

\subsection{Age Model}\label{sec:agemodel}

\ijk{[@REZ: can I remove these notes now?]}
\ijk{re: past/present tense: actually I'm used to writing in the past tense for methods/results, but I wasn't being very consistent. Rewrote everything to past tense now.}

\ijk{something about spectral analysis? \(\to{}\), just rely on original study that showed this. But do I need to mention this? And if so, where?}

The age model for the Zumaia and Sopelana sites was based on the identification of long eccentricity minima in the field by \citeA{Batenburg2012,Batenburg2014}.
The original study relied on foraminifera biostratigraphy, magnetostratigraphy, and astrochronology by tuning these long eccentricity minima to the long eccentricity filter of the La11 solution~\cite{Laskar2011a}. % not sure if sentence is needed
In order to avoid circularity, we assumed a duration of \qty{405}{\kiloyear} for each of these long eccentricity cycles to arrive at an initial floating age model.
A previous best-estimate of the age of the \gls{KT} \cite<\qty{65.9}{\millionyearago},>[]{ZeebeLourens2022EPSL} anchored the floating age model to absolute time.
We then performed optimizations for each astronomical solution that allowed the age of the \gls{KT} to shift by \qty{\pm200}{\kiloyear}
and to adjust the depth for each of the long eccentricity minima tie-points (see \cref{sec:algorithm} for details).

\subsection{Astronomical Solutions}\label{sec:astro}

In this study we compare the available astronomical solutions that had previously shown the best matches to data from the Paleocene \cite{ZeebeLourens2019} and Eocene \cite{ZeebeLourens2022EPSL}.
This includes solutions La10a and La10b~\cite{Laskar2011}
as well as solutions ZB18a~\cite{ZeebeLourens2019}
and ZB20a, ZB20b, ZB20c, and ZB20d~\cite{ZeebeLourens2022EPSL}.
\cref{tab:astronomical-solutions} shows the different parameter variations that were used to generate the latter 5 astronomical solutions.

\begin{table}
\begin{threeparttable}
\caption{Properties of astronomical solutions.\tnote{a}\label{tab:astronomical-solutions}}
\centering
\begin{tabular}{lcccc}
 & \(\Delta{}t\) (days) & \(J_{2}\tnote{b}~\times10^{7}\) & \(N_{\text{ast}}\) & \(N_{\text{LWP}}\) \\
\hline
ZB18a & 2 & \num{1.3050} & 10 & 0 \\
ZB20a & 2 & \num{1.4700} & 50 & 40 \\
ZB20b & 2 & \num{1.3310} & 10 & 0 \\
ZB20c & 2 & \num{1.1708} & 10 & 0 \\
ZB20d & 6 & \num{1.3050} & 33 & 33 \\
\end{tabular}
\begin{tablenotes}
  \item [a] \(\Delta{}t\) = timestep, \(J_{2}\) = solar quadrupole moment, \(N_{\text{ast}}\) = N\textsuperscript{o} of asteroids, NLWP = N\textsuperscript{o} of lightweight particles \cite<table adapted from>[]{ZeebeLourens2022EPSL}.
  \item [b] For discussion of \(J_{2}\) values, see \citeA{ZeebeLourens2022EPSL} section A1.
\end{tablenotes}
\end{threeparttable}
\end{table}



\subsection{Removing Long-Term Trends}\label{sec:detrend}

Several rapid shifts in the lithology---from red, clay-rich marly intervals to whiter limestones---were observed in the Zumaia section on the order of every \qty{50}{\metre}.
This has previously been suggested to potentially indicate a \qty{1.2}{\millionyear} cycle~\cite{Batenburg2014}.
These shifts in lithology were recorded in the proxy archives of Zumaia as very rapid transitions, likely as a result of crossing some threshold value in a nonlinear climate system response.
This could hamper spectral analysis and potentially filtering.
Therefore we used several methods of detrending the records prior to filtering: We
(1) Subtracted a linear fit and scaled (subtracted the mean and divided by the standard deviation) the result;
(2) Fit a \gls{GAM} to the record, subtracted the fit from each value, then scaled the results;
(3) Applied a lowpass filter to the record with a frequency of \qty{0.025}{\metre\per cycle}, subtracted it from each value, then scaled the results;
and (4) Applied a piecewise linear fit based on depth intervals where the lithology changes dramatically, or we observed long-term trends that were at a larger scale than long eccentricity.
The linear fits were subtracted from each value, and the results were scaled.
We have tried many ways of detrending the records, for example with larger and smaller chunks in the depth domain (\cref{fig:detrend}), or with different frequencies for the lowpass filter, and report sensitivity to this in the appendix.
% The effects that these trends can have on our results were limited because we filter only in the time-domain, after applying the age model. % Not sure if true, commented out for now.
%TODO: add link to new appendix figs

% where do we talk about filtering/spectral analysis?
% \rez{you could pad more 5x, 10x, any difference?}
% \ijk{I didn't read the docs correctly, \texttt{padfac} defaults to 5.
%     I quickly tried out 5, 10, 15, 100 times and no diff for 405 and 100 kyr filters.
%     Also tried different padfacs for taner, it just becomes way slower but other than that it's nearly identical once you have over 3x the data.}

\subsection{Finding the Best Fit}\label{sec:algorithm}

In order to test which astronomical solution provides the best match to the data
(and hence is more likely to reflect the true astronomical time scale)
we adopted and extended the approach of \citeA{ZeebeLourens2019,ZeebeLourens2022EPSL}.
We applied the initial age model (\cref{sec:agemodel}) to the detrended data (\cref{sec:detrend}).
After linearly interpolating to a timestep that is a multiple of the timestep of the astronomical solution,
we filtered out the long and short eccentricity cycles.
We used a taner filter with a roll of \num{e10}, targeting periods of \qtylist{405;110}{\kiloyear} (frequency \qty{\pm25}{\percent}) (\cref{fig:filter-windows}).
These filter parameters were selected after analyzing the sensitivity to different ways of filtering (rectangular, gaussian, different values for the taner filter roll parameter in \cref{fig:filter-windows}).

The filtered signals were added and scaled to construct an artificial ``eccentricity'' curve.
The scaled astronomical solutions were then interpolated/subset to the same timesteps as the data.
Then we calculated how well the filtered records matched the astronomical solution via the \gls{RMSD}:
\begin{equation}\label{eqn:rmsd}
    \text{RMSD} = \sqrt{\frac{1}{n}\sum_{i=1}^{n}(e_{i} - f_{i})^{2}}
\end{equation}
where \(e\) is the scaled eccentricity of the astronomical solution and \(f\) is the scaled and filtered combination of long- and short eccentricity-related peaks in the data.

To study the effect of the two eccentricity cycles on the match with the astronomical solution
we assigned different weights to the two filtered signals
(relative weights of 1:0, 1:0.25, 1:0.5, 1:0.75, 1:1, 0.75:1, 0.5:1, 0.25:1, and 0:1 for the long and short eccentricity cycles respectively).
In the main manuscript, we limit our analysis to the simplest 1:1 combination because this preserves the relative amplitude of the long and short eccentricity components from the original data.

An updated age model was created by shifting the age of the \gls{KT}
by up to \qty{\pm200}{\kiloyear} in increments of \qty{2.8}{\kiloyear} (the median sampling rate for this site)
and selecting the offset that gives the lowest \gls{RMSD}.
% This means that a new estimate for the age of the \gls{KT}
% \rez{[this was quite tricky in ZL22. should probably look
% at KTB from both younger and older sides. we can discuss]}
% for each astronomical solution is an additional outcome of this study.
To account for potential errors in the tie-points---the long eccentricity minima depths as identified in the field---we iteratively shifted each tie-point from the youngest to oldest by a range of values between \qtyrange[range-phrase=~to~]{-1.6}{1.6}{\metre} in \qty{20}{\centi\metre} increments (about \qty{10}{\percent} of the long eccentricity period).
After tweaking the tie-point depth, the updated age model was applied, the record was filtered to target the long and short eccentricity cycles, and an artificial eccentricity curve was created to calculate the \gls{RMSD} as before.
Once the optimal (lowest \gls{RMSD}) tie-point depth was found, we fixed it and moved on to the next-youngest tie-point.
This was repeated until all tie-points were adjusted, resulting in the overall best fit.
We performed this analysis separately for the Zumaia and Sopelana sites, so that potential errors in the depth correlation between sites was accounted for.
Furthermore,  scaling the two sites separately should correct for differences in amplitude between the records that could be the result of differences in paleogeographic setting (i.e., the Sopelana site was likely located farther offshore than the Zumaia site~\cite{Batenburg2014}).
We also calculated a square root of the cumulative sum of the squared differences that started at the \gls{KT} and moved to older data points.
This allowed us to visualize where in the time domain the data differs most from the astronomical solutions.
\ijk{[TODO:\ talk about \texttt{depth\_chunk}s for different sedimentation rate intervals vs. rolling RMSD? Maybe only in appendix text? Or not at all?]}

\section{Results}\label{sec:results}

% first talk about how the matched filtered records look!
The matched eccentricity construct for the \gls{L*} record visually matched the different astronomical solutions reasonably well for the Zumaia--Sopelana composite across the Maastrichtian (\cref{fig:rolling-depth-age}). % too nonscientific?
% global patterns we observe
The long eccentricity cycles were closely matched to each solution, while the short eccentricity filter was in general alignment with the solutions in those intervals where the solution showed a high short eccentricity amplitude.

% describe stuff from young to old, from K/T going left
\ijk{[Not sure if this paragraph is interesting enough to talk about.]}
For astronomical solutions ZB18a, ZB20c, and ZB20d, the youngest Maastrichtian long eccentricity bundle \ma{1} showed a single short eccentricity cycle that is either in anti-phase with the filtered data, or has a higher amplitude shortly before the \gls{KT}.
This could be related to the gradual transition that we see in the \gls{L*} data towards the base of the \gls{KT} or to edge effects related to the bandpass filtering.
% should we talk about Ma405 bundle 2  that doesn't have high amplitude in the proxy records,
% but does have high amplitude in the relief in the field/photographs
For the next long eccentricity bundle \ma{2} the eccentricity construct had a relatively low amplitude and so did most astronomical solutions.
However, for \ma{2} visual indications based on the relief and contrast from the section photographs in \citeA<>[Fig. 2]{Batenburg2012} and \citeA<>[Fig. 5a]{Dinares-Turell2013} show relatively high amplitude in short eccentricity, with prominent precession-related marls.
\ijk{[@REZ: ZB20a has low amplitude here! Do I need to reflect on this at the end of the discussion?]}

We observed patterns in the relative amplitude of the long eccentricity and short eccentricity filters that may be related to \acrshortpl{VLN}.
It is difficult to distinguish \acrshortpl{VLN} in the filtered data directly, because there may have been other factors than a \gls{VLN} that resulted in a low amplitude in the short eccentricity-related component of the record.
For example, large changes in sedimentation rate or lithology could have disrupted the band-pass filtering.
Therefore, a more conservative approach focuses on bundles with a high amplitude in the short eccentricity-related cycle in the data.
The records show a higher amplitude of the short eccentricity-related filter in bundle \ma{5}, matching well with all astronomical solutions under evaluation here.
Crucially, however, some astronomical solutions have a low amplitude in the short eccentricity component in intervals where the data showed a relatively high amplitude.
This could indicate that a \gls{VLN} was absent from this interval and that the computed astronomical solution was different from the actual astronomical forcing that the Earth experienced.

In bundle \ma{5} (\appr\qty{67.8}{\millionyearago}) ZB20b has a relatively lower amplitude (with a \gls{VLN} in \ma{6}) than our eccentricity construct, whereas ZB20c and ZB20d matches the data more closely and ZB20a matches it in amplitude but seems slightly out of phase.

The highest-amplitude interval in the record occurred in bundle \ma{8} (\appr\qty{68.9}{\millionyearago}).
This bundle features a darker marly interval at around \qty{117}{\metre}, which corresponds to a strong excursion in both \gls{L*} and \gls{MS} (in the opposite direction) that closely matches the duration of one short eccentricity cycle.
It is referred to as ``escal\'{o}n'' (step), and was previously also associated with a short eccentricity maximum, while other causes such as regional factors or tectonic events could not be excluded~\cite{Dinares-Turell2013}.
% TODO: double-check the cycle counts between the two studies
% There is a giant offset in the composite depth between Dinares-Turell 2013 and Batenburg 2012/2014. At least 86% more section in DT13.
% Dinares-Turell 2013 depth of 220 m is coincident with Batenburg 2012/2014 depth of 180 m = 40 m
% and 100 m = 120 m. = 20 m
% in Dinares-Turell 2013 the escalón occurs between short E31 and E32 (based on filter of CaCO3 for upper part)
% if I count the short ecc cycles identified in the field by Batenburg 2012, escalón is between E33 and E34
This high amplitude of our eccentricity construct is significant, because solutions ZB18a, and ZB20c have a low amplitude of short eccentricity in this interval.
La10b may have a \gls{VLN} somewhere between \ma{7} and \ma{8} and which has a poor match with the data.

During \ma{10} (\appr\qty{69.7}{\millionyearago}) the composite record also indicated a high amplitude of short eccentricity, resulting in a mismatch with ZB20b and to a lesser extent La10c (where the \gls{VLN} seems to occur between \ma{9} and \ma{10}).

% then describe simple "what is the best RMSD-scoring solution?"
The overall fits between our eccentricity constructs and the astronomical solutions (as measured by the \gls{RMSD}) were relatively similar between the different astronomical solutions.
The best (lowest \gls{RMSD}) match was observed between astronomical solution ZB20a and the \gls{L*} eccentricity construct (\cref{fig:full-RMSD-all}).
The \gls{MS} proxy matched the La10c and ZB20d solutions best, while
the \gls{d13C}-proxy, as indicated in the methods, does not capture short-eccentricity variability to such an extent that it can be used to differentiate meaningfully between astronomical solutions.

The cumulative \gls{RMSD} scores indicate when our eccentricity constructs diverged to what extent from the astronomical solutions (\cref{fig:cum-RMSD-all}).
Solutions ZB18a, ZB20c, and ZB20d all show a rapid increase in the cumulative \gls{RMSD} score at around \qty{65.9}{\millionyearago}, which represents the mismatch in the first short eccentricity cycle just prior to the \gls{KT}.
Overall for \gls{L*}, solutions ZB20a and ZB20b outperformed the other solutions up to \appr\qty{69.6}{\millionyearago}, after which ZB20b also develops a mismatch for the Sopelana site.
The rolling \gls{RMSD} scores also highlight that none of the astronomical solutions showed an amplitude that is as large as the eccentricity construct in \ma{8}, resulting in a rapid increase of the cumulative \gls{RMSD} scores for all astronomical solutions under evaluation for this interval.
The rolling \gls{RMSD} score only increased slightly in \ma{10} for ZB20b, whereas we see a clear coincidence of a \gls{VLN} in the solution paired with a relatively high amplitude in the eccentricity construct.
\ijk{[@REZ: this is weird, we see a clear visual mismatch but the rolling RMSD isn't increasing by much here. Any ideas? Maybe because the RMSD is a cumulative sum, new deviations result in a smaller amplitude change as we get to the older parts of the record?]}
During this interval, the rolling \gls{RMSD} score for ZB20a steadily increases at a slower rate, and therefore results in a better overall match.
It appears that the strong negative excursions in our eccentricity construct
at \qty{69.1}{\millionyearago} in ZB20a (oldest part of \ma{8})
and at \qty{70.4}{\millionyearago} in ZB20b (oldest part of \ma{11})
have a great deal of influence on the rolling \gls{RMSD} scores.
\ijk{[@REZ: method could be changed to compare to filters of LEC and SEC of solutions?]}

% OLD notes on chunked-up analysis
% but we can also look at how it changes through time
%% \ijk{TODO:\ update this older section where I looked at discrete chunks.}
%% This analysis can be improved by analyzing specific depth intervals separately, however.
%% The full Zumaia record gives the lowest \gls{RMSD}-scores for the La10b (\num{1.055}) and ZB20b (\num{1.120}) solutions,
%% while the Sopelana site gives the best results for ZB20a (\num{0.872}) and ZB18a (\num{0.894}), and to a lesser extent ZB20c (\num{0.959}) and La10c (\num{0.968}).
%% For Zumaia we also analyzed the record separately for two depth intervals to account for a change in sedimentation rate at around \qty{109.26}{\metre}.
%% The older interval shows a better match with the La10c (\num{0.981}) and ZB20b (1.001) solutions.
%% The younger interval has a slightly better match for La10b (\num{1.104}), ZB20a (\num{1.105}), and ZB20b (\num{1.116}), but results show similar \gls{RMSD} scores.

\ijk{[@REZ: what do you think of this section? It feels like it might belong in the discussion, but I \emph{am} talking about the descriptions/patterns as results].}
Unsurprisingly, these descriptions of the patterns in our eccentricity constructs and the \gls{RMSD} scores and cumulative \gls{RMSD} scores indicate that none of the astronomical solutions fully explain the patterns that we extracted from the proxy archives of the Zumaia and Sopelana field sections.
This is because even the most well-preserved astronomically forced signals from sediments do not reflect only astronomical forcing, but also a combination of the climate signal, the paleo-environment, the lithology that records the signal, effects of diagenesis, which proxy was used to reconstruct the climate signal, and the parameters and choices of our analysis (see \cref{sec:sensitivity} and the Appendix).
Furthermore, the potential for generating slightly different valid astronomical solutions from small parameter variations is, due to the chaotic nature of the solar system, infinite.
This means that a comparison against 7 different calculations of astronomical forcing is not likely to represent an exactly accurate match.
However, both qualitatively (with the visual comparison of the \acrshortpl{VLN}) and quantitatively (with the (rolling) \gls{RMSD}), we can distinguish which of these solutions best matches the astronomical forcing that is detectable in the proxy archive.

% FIGURE 1
\begin{figure}[htb]
  \centering
  \includegraphics[width=\textwidth]{Lstar-vs-solutions.png}
  \caption{\label{fig:rolling-depth-age}
    \textbf{Maastrichtian Zumaia (purple) and Sopelana (orange) inverted filtered/normalized \gls{L*} records tuned to astronomical solutions (black, first 7 panels).}
    % This uses short linear detrending to correct for changes in sediment composition (\cref{sec:detrend}).
    %We also show Zumaia above and below \qty{109.26}{\metre} to account for the change in the sedimentation rate and lithology.
    Bottom three panels show the record in the depth domain.
    Bottom panel shows the log of the two sites, adapted from \citeA{Batenburg2014}.
    Then \gls{L*} values are shown with different ways of detrending in coloured lines (\cref{sec:detrend}).
    The record after piecewise-linear detrendeding and normalization is shown after that.
    Vertical lines show long eccentricity minima as identified in the field,
    with adjustments of up to \qty{\pm1.6}{\metre} (horizontal segments)
    and how they were matched to each astronomical solution in the time domain to minimize \gls{RMSD} (see \cref{sec:algorithm}).
    % \ijk{TODO: I'd like to illustrate long ecc/short ecc filters separately,
    %      but they are different for each one (= too many lines).
    %      A nice place to do this would be in the 3rd panel from the bottom (detrended record),
    %      but I never filter in the depth domain for my analysis and that might be confusing?}
    % -> rez said to just leave it out/potentially put it in the supplement since this is done in Z&L papers already.
    \ijk{TODO:\ add \ma{} numbering to make referring to certain intervals easier? Did it but not sure if happy with it. @REZ?}
    }
\end{figure}


% FIGURE 2
\begin{figure}[htb]
    \centering
    \includegraphics[width=0.6\textwidth]{full_RMSD_scores_all.pdf}
    \caption{\label{fig:full-RMSD-all} % all being all proxies
      \textbf{Best overall matches.}
        \gls{RMSD} scores of compilation of Zumaia and Sopelana proxy records against several orbital solutions.
        % Tie-point depths were adjusted to arrive at the best match with each solution (\cref{sec:algorithm}). % not needed?
    }
\end{figure}

% FIGURE 3
\begin{figure}[htb]
  \centering
  \includegraphics[width=0.6\textwidth]{cumulative_rRMSD_all.png}
  \caption{\label{fig:cum-RMSD-all}
    \textbf{Best matches through time.}
    Square root cumulative sum squared difference scores of
    different astronomical solutions versus the Zumaia and Sopelana Maastrichtian composite record
    for \gls{d13C}, \gls{L*}, and \gls{MS}.
    %\gls{L*} scores (middle panel) diverge most from solutions at around \ijk{TODO}.
    The root cumulative sum of squared differences is calculated as \(\text{RCSD}_{i} = \sqrt{\sum_{i=1}^{k}(e_{i} - f_{i})^{2}}\), where \(k\) is the total number of differences.
  }
\end{figure}


\subsection{Sensitivity Analysis}\label{sec:sensitivity}

Our results were relatively sensitive to the choices of the proxy archive we used, how we detrended the record to subtract non-periodic rapid shifts, and how we performed band-pass filtering.
% proxy
We show the eccentricity constructs based on \gls{MS} and \gls{d13C} against the astronomical solutions in the time domain in the appendix (\cref{fig:rolling-age-MS,fig:rolling-age-d13C}) and include their \gls{RMSD} scores in \cref{fig:full-RMSD-all}.
The \gls{d13C} archive from Zumaia shows a better match with ZB20c and La10b, but the differences are very minor.
Visually, differences between the fits of the \gls{d13C} proxy to different astronomical solutions are hard to distinguish in the time domain (\cref{fig:rolling-age-d13C}).
The \gls{MS} proxy results in the best fits with ZB20d, ZB20c, and to a lesser extent La10c, ZB18a, and La10b, which can be partially explained because the \gls{MS} record shows a high amplitude in \ma{6}, whereas ZB20a, ZB20b, and La10c have a \gls{VLN} here.
\gls{MS} also shows a high amplitude during \ma{8}, which coincides with a \gls{VLN} in ZB20c, ZB18a, and to a lesser extent La10b.
Finally, \ma{10} and \ma{11} also show high amplitude, which rules out ZB20b and La10c.
See the methods for reasons to prefer the \gls{L*} proxy over the \gls{MS} and \gls{d13C} proxies for our purposes.

% boot
One approach to estimate the uncertainty of our \gls{RMSD} scores, is to perform bootstrapping.
The data were randomly resampled many (i.e., \num{e5}) times to the same size as the original data, but with repeated sampling (i.e., some measurements were included twice or more while others were excluded randomly).
Then, the \gls{RMSD} scores were calculated for each of these subsets.
This resulted in a distribution of \gls{RMSD} scores for each proxy against each astronomical solution (\cref{fig:full-RMSD-boot}).
These bootstrapped scores indicate that the differences between eccentricity constructs based on the \gls{d13C} proxy records for the different astronomical solutions were likely not significant,
while the lower score for the \gls{L*} ZB20a solution
and the higher score for \gls{MS} solution ZB20b
likely was.
% commented out b/c calculating an average from a bivariate dist isn't fun
%For a simple average of the bootstrapped values across all proxies, the means overlapped at the \qty{68}{\percent} confidence level.
%After excluding the \gls{d13C} proxy, all astronomical solutions except for ZB20b (which performed worse because of the high \gls{RMSD} score for \gls{MS}) showed a similar overlap.
We hesitate to perform further statistical testing, however, since the next sections will show how sensitive our \gls{RMSD} scores were to different parameter choices that resulted in a different alignment between the proxy record and the astronomical solutions.

% tie-points
The algorithm that allowed changing the depth of each long eccentricity minimum as identified in the field improved the \gls{RMSD} scores quite dramatically (\cref{fig:full-RMSD-tie}).
However, it is sensitive to the extent to which we shifted the record in order to arrive at the best fit and to which we allowed the tie point depths to vary.
% shift
For example, if we let the record shift by up to \qty{400}{\kiloyear} \rez{too much}\ijk{[yeah that's what I'm trying to illustrate here!]} instead of the \qty{200}{\kiloyear} used throughout the manuscript, as one would expect, an offset of more than one long eccentricity cycle was often observed.
This is probably incorrect but happened to result in a better fit with those particular astronomical solutions.
Shifting the record by anywhere between \qtyrange{100}{300}{\kiloyear} typically resulted in an identical fit, however.

% tiepoint tweaks
Furthermore, allowing each tie point depth to vary by larger distance than our default value of \qty{1.6}{\metre} (about \qty{10}{\percent} of the long eccentricity cycle) usually results in slightly better \gls{RMSD} scores (\cref{fig:full-RMSD-tie-error}).
\ijk{[Do I want \emph{another} figure of the RMSD scores vs. the tie point error? I've included it but might want to omit it.]}
A tie point error of up to 3 meters can result in an \gls{RMSD} score for the ZB18a solution that is similar to that of the ZB20a solution.
Visually, the presence of the \gls{VLN} in \ma{8} in ZB18a coincident with the large amplitude does not agree with the simple score, however.
As a quality control, we visualized the \gls{RMSD} scores for each tie point adjustment against depth to make sure that a local minimum was found and that the resulting eccentricity construct did not show any too rapid changes in sedimentation rate (see \cref{fig:tiepoint-RMSD-optima} for an example).

% Changing this to 3, 4, or 5 meters usually results in a better fit that looks nicer and gets a better RMSD score
% 200 kyr + 3 m offset always gives visually similar results but with better RMSD scores.
%For ZB20a, our 1.6 m result goes from \num{1.02} to 3 m \num{0.986}.
%For ZB18a, our 1.6 m result goes from \num{1.08} to 3 m \num{0.989} and is then on-par with ZB20a.
%Setting it to 3, 4, or 5 m would move the K/T boundary up by 3 m, and the same for the minimum after Ma405-1 for ZB18a
%But could their 400 kyr minima idenification really be off by that much?]}
%\ijk{[talk about short ecc count offset between \cite{Batenburg2012} and \cite{Dinares-Turell2013}. Not sure if this should be discussion.]}
%\rez{It doesn't change the results, [illegible]}
% OK comment out for now.

% detrend
Different ways of detrending the proxy records to subtract non-cyclical patterns also lead to different results (\cref{fig:full-RMSD-detrend}).
Overall, the different ways of detrending the record (\cref{fig:detrend}) showed similar patterns, but the \gls{RMSD} scores for some solutions varied more between different ways of detrending than between different astronomical solutions (\cref{fig:full-RMSD-detrend}).
The large changes in lithology result in a very large amplitude of the short eccentricity filter in the simple detrending approach.
For most solutions, this simply results in a worse fit.
But for ZB20a and La10b, using a different detrending strategy makes our algorithm shift the long- eccentricity tie point depths enough that it is offset by about one short eccentricity cycle (\cref{fig:Lstar-detrend}), resulting in an even worse fit.

% filter
% window: gaussian, rectangular, taner
The bandpass filtering window used also affected the results (\cref{fig:filter-windows}).
The gaussian window performed worst (after rescaling it so the comparison uses similar filter widths, see \cref{fig:filter-windows}), followed by a rectangular filter, while the taner filter with intermediate roll parameters resulted in a lower \gls{RMSD} (performed better).
% taner roll parameter: more rectangular to more gaussian to more t-dist
As one would expect, taner filters with a very high roll parameter (i.e., \num{e100}) performed very similar to the rectangular filter.
Taner roll parameters with a relatively smooth window (\(\texttt{roll} \ge 10^{8}\) and \(\le 10^{12}\)), made our algorithm shift the eccentricity construct by one short eccentricity cycle against the ZB20a solution, resulting in a much better fit (\cref{fig:full-RMSD-filter,fig:filter-windows}).
% frac: window width, ±XX% of the target frequency
The same type of shift occurred when we changed the width of the filters (\cref{fig:full-RMSD-frac}).

% comb: weight of long and short ecc
Applying a different weighting to the long- and short eccentricity filters, typically resulted in better \gls{RMSD} scores than the 1:1 combination (\cref{fig:full-RMSD-comb}).
However, these effects seemed relatively uniform across all astronomical solutions.
We therefore prefer using the 1:1 solution so that the relative filter amplitude is preserved from the original data.

Together, these sensitivity analyses show that it is worthwhile to explore the parameter space for both robustness and correctness.
Our results were sensitive to choices in data processing, but careful visual/qualitative comparison in combination with the quantitative \gls{RMSD} scores indicated which set of parameter choices was best able to distinguish between astronomical solutions.
\ijk{[@REZ: any suggestions on how to improve this paragraph as a nice summary for the sensitivity section?]}




\section{Discussion}\label{sec:discussion}

This study aims to determine which, if any, of the currently available astronomical solutions best match the Maastrichtian proxy records of the Zumaia and Sopelana sites, in order to arrive at an astronomically calibrated time scale.
Ultimately, these results will allows us to anchor events in Earth's geologic past and to distinguish forcings and feedbacks in the climate system.

During the Paleocene, solutions La10b and La10c showed a poor match with data from the Walvis Ridge, lacking a \gls{VLN} at around \qty{61}{\millionyearago}~\cite{ZeebeLourens2022EPSL}.
Our analysis shows that solutions ZB18a and ZB20c and to a lesser extent La10b are incompatible with the high amplitude in the data from Zumaia at \ma{5}.
Then, as we move to the older part of the record, the Sopelana data best matches ZB20a because ZB20b has a very long eccentricity node at \ma{10} (\appr\qty{69.7}{\millionyearago}), whereas the amplitude of the filtered data related to the short-eccentricity cycle is large there.
Therefore, the ZB20a solution is most compatible with the available Maastrichtian data and could be used for tuning up to \appr\qty{71}{\millionyearago}.

Our sensitivity analysis shows that the methodology is quite sensitive to parameter choices, however.
This may be the result of the proxy archive itself, which, while showing a clear preference for the ZB20a solution when looking at the \gls{L*} proxy, shows a slight preference for the ZB20d solution when looking at the \gls{MS} proxy (\cref{fig:full-RMSD-all}) and no clear differences when looking at the \gls{d13C} proxy.
In \cref{sec:data}, we articulate why the color reflectance \gls{L*} record most directly reflects the orbital forcing and is our preferred proxy archive to make inferences about the most accurate astronomical forcing scenario in this instance.
Different ways of pre-processing the data are discussed in \cref{sec:sensitivity}, and highlight the importance careful selection of a set of parameters that results in the best matches, while not violating constraints from other data (e.g., sedimentation rates, the age of the \gls{KT}, etc.).
After careful consideration of these factors, we argue that it is possible to select an optimal set of parameter and filtering frequencies/widths that results in a reliable outcome to the extent that these data allow for.

\ijk{[This paragraph is not done: Can we look at multiple sites? literature comparison, only visual]}
%Previous studies have also indicated the presence of a high amplitude of the short-eccentricity cycle in \ma{8} in ODP Site 1262 and to a lesser extent in 1209. % where did I see this? Probably mistook Pc8 for Ma405-8
Contrast to:
Zumaia section photographs show high amplitude in \ma{2}~\cite{Batenburg2012,Dinares-Turell2013}.
Visually, the Zumaia \ce{CaCO3} record by \citeA{tenKateSprenger1993} shows a high amplitude in the short eccentricity-related peak for \ma{2} in  wavelet analysis~\cite{Dinares-Turell2013} % \ma{1--4} % also in \cite{Westerhold2008} figure 4.
% TODO: but where they put the Ma3 annotations is out of phase to how Husson2011 and Batenburg2012/2014 do it?
% below is too short to help us
%ODP 1262 \ma{1--2}~\cite{Westerhold2008} figure 4
%and \cite{Hilgen2010}. % Fig 2., Fig. 6 log(MS) + log(Fe)
% found via \cite{Dinares-Turell2014}, % log(Fe) and log(MS) Fig. 2
%1267 \ma{1} and \ma{2}~\cite{Westerhold2008} figure 4.
% this is all Husson2011:
%525A gray level bundle 1 (fig 4)
1267B MS 1--6.5 (Fig. 4) but note that they did not use the splice table from~\cite{Westerhold2008} to convert from MBSF to RMCD.
762C bundles 1 to 5.5 (Fig. 4)
525A gray level smoothed 6--9 (Fig. 3) hard to tell...
762C gray level smoothed bundles 8 to 16 (Fig. 3) indicates potential hiatus within this bundle but also high change
1258A MS \ma{8--14} (Fig. 3) no recovery in 10
of~\citeA{Husson2011}.
The key intervals of \ma{5}, \ma{8}, and \ma{10} are \ijk{XXX}.
IODP Leg 342 Site U1403 \ma{1--7} ln(Fe/Ca) and \gls{MS}~\cite<Fig. 6 in>[]{Batenburg2018} shows high amplitude in the short eccentricity component of \ma{5} % WHOAH did I see this study/site before?

Future studies should aim to generate high resolution proxy records that have recorded the expression of both long and short eccentricity and connect to the oldest geologically constrained parts of the astronomical solutions.
It may be wise to target low latitudes for this, since eccentricity-related cyclicity tends to dominate in the tropics~\cite<Fig. 7a in>[]{LaeppleLohmann2009}. % during the late Pleistocene (750 ka to present) in model
\ijk{[but of course the above depends on which proxy you're looking at... cut the sentence out?]}
We further suggest that future studies generate new astronomical solutions with similar initial settings to the ZB20a solution and, using appropriate parameter variations, explore this crucial chaotic horizon of the solar system.

%%

%  Numbered lines in equations:
%  To add line numbers to lines in equations,
%  \begin{linenomath*}
%  \begin{equation}
%  \end{equation}
%  \end{linenomath*}



%% Enter Figures and Tables near as possible to where they are first mentioned:
%
% DO NOT USE \psfrag or \subfigure commands.
%
% Figure captions go below the figure.
% Table titles go above tables;  other caption information
%  should be placed in last line of the table, using
% \multicolumn2l{$^a$ This is a table note.}
%
%----------------
% EXAMPLE FIGURES
%
% \begin{figure}
% \includegraphics{example.png}
% \caption{caption}
% \end{figure}
%
% Giving latex a width will help it to scale the figure properly. A simple trick is to use \textwidth. Try this if large figures run off the side of the page.
% \begin{figure}
% \noindent\includegraphics[width=\textwidth]{anothersample.png}
%\caption{caption}
%\label{pngfiguresample}
%\end{figure}
%
%
% If you get an error about an unknown bounding box, try specifying the width and height of the figure with the natwidth and natheight options. This is common when trying to add a PDF figure without pdflatex.
% \begin{figure}
% \noindent\includegraphics[natwidth=800px,natheight=600px]{samplefigure.pdf}
%\caption{caption}
%\label{pdffiguresample}
%\end{figure}
%
%
% PDFLatex does not seem to be able to process EPS figures. You may want to try the epstopdf package.
%

%
% ---------------
% EXAMPLE TABLE
% Please do NOT include vertical lines in tables
% if the paper is accepted, Wiley will replace vertical lines with white space
% the CLS file modifies table padding and vertical lines may not display well
%
 % \begin{table}
 % \caption{Time of the Transition Between Phase 1 and Phase 2$^{a}$}
 % \centering
 % \begin{tabular}{l c}
 % \hline
 %  Run  & Time (min)  \\
 % \hline
 %   $l1$  & 260   \\
 %   $l2$  & 300   \\
 %   $l3$  & 340   \\
 %   $h1$  & 270   \\
 %   $h2$  & 250   \\
 %   $h3$  & 380   \\
 %   $r1$  & 370   \\
 %   $r2$  & 390   \\
 % \hline
 % \multicolumn{2}{l}{$^{a}$Footnote text here.}
 % \end{tabular}
 % \end{table}

%% SIDEWAYS FIGURE and TABLE
% AGU prefers the use of {sidewaystable} over {landscapetable} as it causes fewer problems.
%
% \begin{sidewaysfigure}
% \includegraphics[width=20pc]{figsamp}
% \caption{caption here}
% \label{newfig}
% \end{sidewaysfigure}
%
%  \begin{sidewaystable}
%  \caption{Caption here}
% \label{tab:signif_gap_clos}
%  \begin{tabular}{ccc}
% one&two&three\\
% four&five&six
%  \end{tabular}
%  \end{sidewaystable}

%% If using numbered lines, please surround equations with \begin{linenomath*}...\end{linenomath*}
%\begin{linenomath*}
%\begin{equation}
%y|{f} \sim g(m, \sigma),
%\end{equation}
%\end{linenomath*}

%%% End of body of article

%%%%%%%%%%%%%%%%%%%%%%%%%%%%%%%%
%% Optional Appendix goes here
%
% The \appendix command resets counters and redefines section heads
%
% After typing \appendix
%
%\section{Here Is Appendix Title}
% will show
% A: Here Is Appendix Title
%
\appendix
% this template also turns the open research section into an appendix...

% We show some more combinations of \cref{fig:full-RMSD-all} in \cref{fig:full-RMSD-detrend}.
% We show an adaptation of \cref{fig:rolling-depth-age} for \gls{MS} (\cref{fig:rolling-age-MS}) and for \gls{d13C} (\cref{fig:rolling-age-d13C}).

% it actually didn't reset the figure numbers
\renewcommand\thefigure{A\arabic{figure}}
\setcounter{figure}{0}
% maybe some of this needs to be digital supplement rather than appendix?

% Appendix Figure: MS filters vs solutions
\begin{figure}[htb]
  \centering
  \includegraphics[width=\textwidth]{MS-vs-solutions.png}
  \caption{\label{fig:rolling-age-MS}
    \textbf{Maastrichtian Zumaia (purple) and Sopelana (orange) filtered/normalized \gls{MS} records tuned to astronomical solutions (black, first 7 panels).}
    % This uses short linear detrending to correct for changes in sediment composition (\cref{sec:detrend}).
    %We also show Zumaia above and below \qty{109.26}{\metre} to account for the change in the sedimentation rate and lithology.
    Bottom three panels show the record in the depth domain.
    Bottom panel shows the log of the two sites, adapted from \citeA{Batenburg2014}.
    Then \gls{MS} values are shown with different ways of detrending in coloured lines (\cref{sec:detrend}).
    The record after piecewise-linear detrendeding and normalization is shown after that.
    Vertical lines show long eccentricity minima as identified in the field,
    with adjustments of up to \qty{\pm1.6}{\metre} (horizontal segments)
    and how they were matched to each astronomical solution in the time domain to minimize their \gls{RMSD} (see \cref{sec:algorithm}).
    }
\end{figure}

% Appendix Figure: d13C filters vs solutions
\begin{figure}[htb]
  \centering
  \includegraphics[width=\textwidth]{d13C-vs-solutions.png}
  \caption{\label{fig:rolling-age-d13C}
    \textbf{Maastrichtian Zumaia (purple) inverted filtered/normalized \gls{d13C} records tuned to astronomical solutions (black, first 7 panels).}
    % This uses short linear detrending to correct for changes in sediment composition (\cref{sec:detrend}).
    %We also show Zumaia above and below \qty{109.26}{\metre} to account for the change in the sedimentation rate and lithology.
    Bottom three panels show the record in the depth domain.
    Bottom panel shows the log of the two sites, adapted from \citeA{Batenburg2014}.
    Then \gls{d13C} values are shown with different ways of detrending in coloured lines (\cref{sec:detrend}).
    The record after piecewise-linear detrendeding and normalization is shown after that.
    Vertical lines show long eccentricity minima as identified in the field,
    with adjustments of up to \qty{\pm1.6}{\metre} (horizontal segments)
    and how they were matched to each astronomical solution in the time domain to minimize their \gls{RMSD} (see \cref{sec:algorithm}).
    \ijk{TODO: comment on overfitting detrending and lower resolution of data.}
    \ijk{@REZ: do you see the \appr\qty{1.6}{\millionyear} cycles here before detrending? Might be something to look at?}
    }
\end{figure}

% Appendix Figure: detrend fits compared to raw data vs. depth
% TODO: remove some of these?
\begin{figure}[htbp]
  \centering
  \includegraphics[width=.9\linewidth]{depth_detrend.png}
  \caption{\label{fig:detrend}
    \textbf{Zumaia (purple) and Sopelana (yellow) trend removal strategies.}
    The raw data and the lines that were fit (other colours), which were subtracted from the record prior to filtering.
    The piecewise linear fit that we subtract in the main manuscript corresponds to \texttt{lin\_pred\_med}.
    % The detrended resultant record is callend \texttt{lin\_scl\_med}.
  }
\end{figure}

% Appendix Figure: effects of detrending in time domain
\begin{figure}[htb]
  \centering \includegraphics[width=\textwidth]{sol_SD_detrend.pdf}
  \caption{\label{fig:Lstar-detrend}
  \textbf{Effects of different detrending strategies in the time domain.}
    Illustration of how simple scaling of \gls{L*} with subsequent linear detrending (lower opacity lines)
    compares to the piecewize linear detrending strategy used in the main manuscript.
    Both use the same taner filter to extract long and short eccentricity components.
    }
\end{figure}

% Appendix Figure: Bootstrapped RMSD
\begin{figure}[htb]
    \centering
    \includegraphics[width=0.6\textwidth]{full_RMSD_boot.png}
    \caption{\label{fig:full-RMSD-boot}
        \textbf{\gls{RMSD} scores of compilation of Zumaia and Sopelana proxy records against several orbital solutions.}
        Shaded intervals represent the bootstrapped (\(N = 10^{5}\)) \qtyrange{5}{95}{\percent} confidence intervals of the \gls{RMSD} (in increments of \qty{5}{\percent}).
        % Tie-point depths were adjusted to arrive at the best match with each solution (\cref{sec:algorithm}).
        This illustrates that while differences between astronomical solutions are not large, they may be significant.
    }
\end{figure}

% Appendix Figure: RMSD scores before and after tie-point adjustments
\begin{figure}[htb]
  \centering \includegraphics[width=\textwidth]{sol_SD_tie.png}
  \caption{\label{fig:full-RMSD-tie}
    \textbf{Illustration of how RMSD scores change after tie-point depth optimization.}
    The change in \gls{RMSD} scores for the
    %piecewize linearly detrended
    Zumaia (not the Zumaia/Sopelana composite!) proxy records (column panels)
    % where the long and short eccentricity components where filtered with a taner filter
    when we allow tie-point depths identified in the field to vary by \qty{\pm1.6}{\metre}.
    This uses the taner filter from the main text on the piecewise linearly detrended record.
    }
\end{figure}

% Appendix Figure: RMSD scores before and after tie-point adjustments
\begin{figure}[htb]
  \centering \includegraphics[width=0.7\textwidth]{full_RMSD_tiepoint-error.pdf}
  \caption{\label{fig:full-RMSD-tie-error}
    \textbf{Changing RMSD scores after a wider range of tie-point depth optimization.}
    Note the two alternative \gls{RMSD} scores for a maximum adjustment size of \qty{3}{\metre} for La10c are the result of a shift of \qty{200}{\kiloyear} (bottom point, better score) or \qty{300}{\kiloyear} (top point).
    This uses the \gls{L*} record with a taner filter from the main text on the piecewise linearly detrended record.
    }
\end{figure}

% Appendix Figure: RMSD scores with different tie-point adjustments
\begin{figure}[htb]
  \centering \includegraphics[width=\textwidth]{tiepoint_RMSD_optima.png}
  \caption{\label{fig:tiepoint-RMSD-optima}
    \textbf{Illustration of how RMSD scores change by performing tie-point depth optimization.}
    The \gls{RMSD} scores as they evolve when each tie point is iteratively shifted for the \gls{L*} record of Zumaia (top panel) against ZB18a.
    We show the eccentricity construct (black) versus ZB18a (yellow).
    This uses the taner filter from the main text on the piecewise linearly detrended record.
    }
\end{figure}


% Appendix Figure: RMSD scores for different ways of detrending
\begin{figure}[htb]
  \centering \includegraphics[width=\textwidth]{full_RMSD_detrend_comparison.png}
  \caption{\label{fig:full-RMSD-detrend}
    \textbf{Sensitivity analysis of various ways of detrending the data.}
    This uses optimized tie-point depths and limits the results to a rectangular filter.
    We show main-paper taner filters for \texttt{lin\_scl\_med} with different colours for each proxy for context.
    %Note that the \texttt{lin\_scl\_fine} for the taner filter is very similar to the one for \texttt{lin\_scl\_med}.
    }
\end{figure}

% Appendix Figure:  RMSD scores for different filter strategies
\begin{figure}[htb]
  \centering \includegraphics[width=\textwidth]{full_RMSD_filter_comparison.png}
  \caption{\label{fig:full-RMSD-filter}
    \textbf{Sensitivity analysis of different types of bandpass filtering.}
    %Note that gaussian window also narrows filter width (see \cref{fig:filter-windows}).
    This uses the \texttt{lin\_scl\_med} detrend type and tie-point depth optimization with a target frequency fraction of \qty{25}{\percent}.
    The taner filter has the main-text roll parameter of \num{e10}.
    Note that the way \texttt{astrochron} parametrizes the gaussian (dark red triangles) window width is not directly comparable to the taner (black circles) and rectangular (grey squares) filters.
    Therefore, we have reparametrized the gaussian filter so that it intersects the taner filters where they cross the rectangular filter,
    % the upper and lower boundaries correspond to \(\pm2\sigma\). -> that was frac * 1.5
    by multiplying the fraction (\qty{25}{\percent})
    by \num{2.1227} for a gaussian alpha of 3,
    or by \(\frac{2.1227}{3}\times5\) for an alpha of 5.
    This results in a gaussian filter that is comparable to a taner filter with roll parameter \num{e4}.
    % We also show the original for \gls{L*} but this is effectively a narrower filter range.
    See \cref{fig:filter-windows} for illustrations of filter windows and alternative taner roll parameters.
    }
\end{figure}

% Appendix Figure:  Filter Windows + RMSD scores for different taner filters
\begin{figure}[htb]
  \centering \includegraphics[width=0.9\textwidth]{filter_windows.pdf}
  \caption{\label{fig:filter-windows}
    \textbf{Sensitivity analysis for taner filter roll parameters.}
    Illustration of different taner roll parameter filter windows and the rectangular and gaussian filter windows (top panel)
    % The rectangular (black/gray), gaussian (dark red, red), and taner filters (dark blue to green to yellow)
    useing the same filter boundary frequencies of \(\frac{1}{405}\) and \(\frac{1}{110}\)~\si[per-mode=power]{\per\kiloyear} \qty{\pm25}{\percent} as in the main text.
    The taner filters go from the pointy \num{e3} (narrowest peak, widest shoulders, deep purple)
    moving outwards from the peak to \num{e4}, \num{e6}, \num{e8}, (shades of blue)
    \num{e10} (the preferred choice in the main text),
    \num{e12} (shades of green), and \num{e100} (yellow),
    approaching the rectangular filter (black).
    Note that the way \texttt{astrochron} parameterizes the Gaussian window width is not directly comparable to the taner and rectangular filters.
    %Setting the Gaussian filter alpha to a value to match the width of the taners could work, but would cut off any lower/higher frequencies.
    Therefore, we also show a reparametrized Gaussian filter
    %where the upper and lower boundaries correspond to \(\pm2\sigma\) (darkred gaussian).
    that intersects with where the taner filters cross each other and the rectangular filter (see \cref{fig:full-RMSD-filter} for details).
    We show MTM-spectral peaks of the astronomical solutions analyzed in this study in black for reference.

    The bottom panel contrasts \gls{RMSD} scores for the \texttt{lin\_scl\_med} detrended \gls{L*} data with tie-point depth optimization for different roll parameters.
    Note that these scores differ from \cref{fig:full-RMSD-filter} because they use a slightly wider fraction of \qty{\pm30}{\percent}) than the \qty{\pm25}{\percent} in the main text (to avoid re-computation and to illustrate the effects of changing the fraction).
    }
\end{figure}

% IJK: commented this out b/c it was a messy figure and the above two figures should show the narrowing of the filters as well.
% % Appendix Figure:  RMSD scores for different filter strategies
% \begin{figure}[htb]
%   \centering \includegraphics[width=0.7\textwidth]{full_RMSD_frac.pdf}
%   \caption{\label{fig:full-RMSD-frac}
%     \textbf{Sensitivity analysis of different bandpass filter widths.}
%     The frac parameter is the fraction that determines the upper and lower boundaries of the target frequency.
%     %% For example, targeting the \(\frac{1}{405}}\)\si{\per\kiloyear} frequency \(\pm\texttt{frac}\times\texttt{freq}\).
%     %Note that gaussian window also narrows filter width (see \cref{fig:filter-windows}).
%     This uses \gls{L*} proxy with the \texttt{lin\_scl\_med} detrend type and tie-point depth optimization.
%     The taner filter has a roll parameter of \num{e10}.
%     }
% \end{figure}



% Appendix Figure:  RMSD scores for different weights of long- and short ecc
\begin{figure}[htb]
  \centering \includegraphics[width=\textwidth]{full_RMSD_comb_comparison.png}
  \caption{\label{fig:full-RMSD-comb}
    \textbf{Sensitivity analysis of different weights for long- and short eccentricity.}
    Different astronomical solutions (colours) show very similar patterns,
    where most combinations of the short and long eccentricity filters other than 1:1 result in lower \gls{RMSD} scores.
    Still we use the 1:1 weights in the main text because they preserve the relative filter amplitudes from the data.
    This uses the \texttt{lin\_scl\_med} detrend type and tie-point depth optimization.
    The taner filters have the main-text roll parameter of \num{e10} and a filter fraction of \num{0.25}.
    }
\end{figure}


%%%%%%%%%%%%%%%%%%%%%%%%%%%%%%%%%%%%%%%%%%%%%%%%%%%%%%%%%%%%%%%%%%%%%%%%%%
%                   some spectral analysis figures                       %
%%%%%%%%%%%%%%%%%%%%%%%%%%%%%%%%%%%%%%%%%%%%%%%%%%%%%%%%%%%%%%%%%%%%%%%%%%

% \begin{figure}[htb]
%   \centering \includegraphics[width=\textwidth]{Zumaia-Sopelana_mtm_raw.pdf}
%   \caption{\label{fig:spectral-depth}
%     \textbf{Spectral analysis in the depth domain.}
%     % Do I need refs for all of these?
%     \ijk{In the end probably show analysis only in time-domain?}
%     BT = Blackman-Tukey,
%     FFT = Fast Fourier Transform,
%     LOWSPEC = Robust Locally-Weighted Regression Spectral Background Estimation \cite{Meyers2012},
%     LS = Lomb-Scargle,
%     MTLS = Multi-taper Averaged Lomb-Scargle periodogram of (un)evenly
% spaced data \cite{Springford2020},
%     MTM = Multitaper method \cite{Thomson1982}.
%     Shaded intervals for the MTM and LOWSPEC indicate AR1 fit and AR1-power and LOWSPEC fit and power confidence levels.
%   }
% \end{figure}
%
% \begin{figure}[htb]
%   \centering \includegraphics[width=\textwidth]{Zumaia-Sopelana_mtm.pdf}
%   \caption{\label{fig:spectral-depth}
%     \textbf{Spectral analysis in the depth domain.}
%     % Do I need refs for all of these?
%     \ijk{This is the same as above but after linear detrending with \texttt{lin\_scl\_fine}.}
%     BT = Blackman-Tukey,
%     FFT = Fast Fourier Transform,
%     LOWSPEC = Robust Locally-Weighted Regression Spectral Background Estimation \cite{Meyers2012},
%     LS = Lomb-Scargle,
%     MTLS = Multi-taper Averaged Lomb-Scargle periodogram of (un)evenly
% spaced data \cite{Springford2020},
%     MTM = Multitaper method \cite{Thomson1982}.
%     Shaded intervals for the MTM and LOWSPEC indicate AR1 fit and AR1-power and LOWSPEC fit and power confidence levels.
%   }
% \end{figure}
%
%
% \begin{figure}[htb]
%   \centering \includegraphics[width=\textwidth]{Zumaia_Sopelana_spectra_filters_raw.pdf}
%   \caption{\label{fig:spectral-age-raw}
%     \textbf{Spectral analysis in the time domain.}
%     % Do I need refs for all of these?
%     \ijk{This is raw values, only linear detrend}
%     % BT = Blackman-Tukey,
%     FFT = Fast Fourier Transform,
%     LOWSPEC = Robust Locally-Weighted Regression Spectral Background Estimation \cite{Meyers2012},
%     % LS = Lomb-Scargle,
%     MTLS = Multi-taper Averaged Lomb-Scargle periodogram of (un)evenly
% spaced data \cite{Springford2020},
%     MTM = Multitaper method \cite{Thomson1982}.
%     Shaded intervals for the MTM and LOWSPEC indicate AR1 fit and AR1-power and LOWSPEC fit and power confidence levels.
%   }
% \end{figure}
%
%
% \begin{figure}[htb]
%   \centering
%   \includegraphics[width=1.2\textwidth]{Zumaia_MS_1-1_solutions_simple_with_log.pdf}
%   \caption{\label{fig:rolling-age-MS}
%     Same as \cref{fig:rolling-depth-age} but for \gls{MS}.}
% \end{figure}

% \begin{figure}[htb]
%   \centering
%   \includegraphics[width=0.9\textwidth]{Zumaia_d13C_1-1_solutions_simple_with_log.pdf}
%   \caption{\label{fig:rolling-age-d13C}
%     Same as \cref{fig:rolling-depth-age} but for \gls{d13C}.}
% \end{figure}


%%%%%%%%%%%%%%%%%%%%%%%%%%%%%%%%%%%%%%%%%%%%%%%%%%%%%%%%%%%%%%%%
%
% Optional Glossary, Notation or Acronym section goes here:
%
%%%%%%%%%%%%%%
% Glossary is only allowed in Reviews of Geophysics
%  \begin{glossary}
%  \term{Term}
%   Term Definition here
%  \term{Term}
%   Term Definition here
%  \term{Term}
%   Term Definition here
%  \end{glossary}

%
%%%%%%%%%%%%%%
% Acronyms
%   \begin{acronyms}
%   \acro{Acronym}
%   Definition here
%   \acro{EMOS}
%   Ensemble model output statistics
%   \acro{ECMWF}
%   Centre for Medium-Range Weather Forecasts
%   \end{acronyms}

%
%%%%%%%%%%%%%%
% Notation
%   \begin{notation}
%   \notation{$a+b$} Notation Definition here
%   \notation{$e=mc^2$}
%   Equation in German-born physicist Albert Einstein's theory of special
%  relativity that showed that the increased relativistic mass ($m$) of a
%  body comes from the energy of motion of the body—that is, its kinetic
%  energy ($E$)—divided by the speed of light squared ($c^2$).
%   \end{notation}



\section*{Open Research}

\Gls{MS}, \gls{L*}, and \gls{d13C} data used in this study are from \citeA{Batenburg2012,Batenburg2012}.

Analysis was performed using the R programming language~\cite{RCoreTeam2020} and made use of \texttt{astrochron} \citeA{Meyers2014} and the \texttt{tidyverse} \citeA{Wickham2019}.
The new R package \texttt{AstronomicalSolutions} will be made available on publication on \url{https://github.com/japhir/AstronomicalSolutions} \citeA{Kocken2024}.

\ijk{\textbf{TODO:} come up with nice package name (working title: \texttt{AstronomicalSolutions} so it's broad enough for future additions) and host on github/ archive on Zenodo.}

% AGU requires an Availability Statement for the underlying data needed to understand, evaluate, and build upon the reported research at the time of peer review and publication.

% Authors should include an Availability Statement for the software that has a significant impact on the research. Details and templates are in the Availability Statement section of the Data and Software for Authors Guidance: \url{https://www.agu.org/Publish-with-AGU/Publish/Author-Resources/Data-and-Software-for-Authors#availability}

% It is important to cite individual datasets in this section and, and they must be included in your bibliography. Please use the type field in your bibtex file to specify the type of data cited. Some options include Dataset, Software, Collection, ComputationalNotebook. Ex:
% \\
% \begin{verbatim}

% @misc{https://doi.org/10.7283/633e-1497,
%   doi = {10.7283/633E-1497},
%   url = {https://www.unavco.org/data/doi/10.7283/633E-1497},
%   author = {de Zeeuw-van Dalfsen, Elske and Sleeman, Reinoud},
%   title = {KNMI Dutch Antilles GPS Network - SAB1-St_Johns_Saba_NA P.S.},
%   publisher = {UNAVCO, Inc.},
%   year = {2019},
%   type = {dataset}
% }

% \end{verbatim}

% For physical samples, use the IGSN persistent identifier, see the International Geo Sample Numbers section:
% \url{https://www.agu.org/Publish-with-AGU/Publish/Author-Resources/Data-and-Software-for-Authors#IGSN}
%%%%%%%%%%%%%%%%%%%%%%%%%%%%%%%%%%%%%%%%%%%%%%%

\acknowledgments
% This section is optional. Include any Acknowledgments here.
% The acknowledgments should list:\\
% All funding sources related to this work from all authors\\
% Any real or perceived financial conflicts of interests for any author\\
% Other affiliations for any author that may be perceived as having a conflict of interest with respect to the results of this paper.\\
% It is also the appropriate place to thank colleagues and other contributors. AGU does not normally allow dedications.

This work was supported by the Heising-Simons Foundation (\#2021-2800), under the CycloAstro
Cohort project 3 and U.S. NSF grants OCE20-01022, OCE20-34660 to R.E.Z.

%% ------------------------------------------------------------------------ %%
%% References and Citations

%%%%%%%%%%%%%%%%%%%%%%%%%%%%%%%%%%%%%%%%%%%%%%%
%
% \bibliography{<name of your .bib file>} don't specify the file extension
%
% don't specify bibliographystyle

% In the References section, cite the data/software described in the Availability Statement (this includes primary and processed data used for your research). For details on data/software citation as well as examples, see the Data & Software Citation section of the Data & Software for Authors guidance
% https://www.agu.org/Publish-with-AGU/Publish/Author-Resources/Data-and-Software-for-Authors#citation

%%%%%%%%%%%%%%%%%%%%%%%%%%%%%%%%%%%%%%%%%%%%%%%

\bibliography{references}


%Reference citation instructions and examples:
%
% Please use ONLY \cite and \citeA for reference citations.
% \cite for parenthetical references
% ...as shown in recent studies (Simpson et al., 2019)
% \citeA for in-text citations
% ...Simpson et al. (2019) have shown...
%
%
%...as shown by \citeA{jskilby}.
%...as shown by \citeA{lewin76}, \citeA{carson86}, \citeA{bartoldy02}, and \citeA{rinaldi03}.
%...has been shown \cite{jskilbye}.
%...has been shown \cite{lewin76,carson86,bartoldy02,rinaldi03}.
%... \cite <i.e.>[]{lewin76,carson86,bartoldy02,rinaldi03}.
%...has been shown by \cite <e.g.,>[and others]{lewin76}.
%
% apacite uses < > for prenotes and [ ] for postnotes
% DO NOT use other cite commands (e.g., \citet, \citep, \citeyear, \citealp, etc.).
% \nocite is okay to use to add references from your Supporting Information
%


\end{document}



% More Information and Advice:

%% ------------------------------------------------------------------------ %%
%
%  SECTION HEADS
%
%% ------------------------------------------------------------------------ %%

% Capitalize the first letter of each word (except for
% prepositions, conjunctions, and articles that are
% three or fewer letters).

% AGU follows standard outline style; therefore, there cannot be a section 1 without
% a section 2, or a section 2.3.1 without a section 2.3.2.
% Please make sure your section numbers are balanced.
% ---------------
% Level 1 head
%
% Use the \section{} command to identify level 1 heads;
% type the appropriate head wording between the curly
% brackets, as shown below.
%
%An example:
%\section{Level 1 Head: Introduction}
%
% ---------------
% Level 2 head
%
% Use the \subsection{} command to identify level 2 heads.
%An example:
%\subsection{Level 2 Head}
%
% ---------------
% Level 3 head
%
% Use the \subsubsection{} command to identify level 3 heads
%An example:
%\subsubsection{Level 3 Head}
%
%---------------
% Level 4 head
%
% Use the \subsubsubsection{} command to identify level 3 heads
% An example:
%\subsubsubsection{Level 4 Head} An example.
%
%% ------------------------------------------------------------------------ %%
%
%  IN-TEXT LISTS
%
%% ------------------------------------------------------------------------ %%
%
% Do not use bulleted lists; enumerated lists are okay.
% \begin{enumerate}
% \item
% \item
% \item
% \end{enumerate}
%
%% ------------------------------------------------------------------------ %%
%
%  EQUATIONS
%
%% ------------------------------------------------------------------------ %%

% Single-line equations are centered.
% Equation arrays will appear left-aligned.

% Math coded inside display math mode \[ ...\]
%  will not be numbered, e.g.,:
%  \[ x^2=y^2 + z^2\]

%  Math coded inside \begin{equation} and \end{equation} will
%  be automatically numbered, e.g.,:
%  \begin{equation}
%  x^2=y^2 + z^2
%  \end{equation}


% % To create multiline equations, use the
% % \begin{eqnarray} and \end{eqnarray} environment
% % as demonstrated below.
% \begin{eqnarray}
%   x_{1} & = & (x - x_{0}) \cos \Theta \nonumber \\
%         && + (y - y_{0}) \sin \Theta  \nonumber \\
%   y_{1} & = & -(x - x_{0}) \sin \Theta \nonumber \\
%         && + (y - y_{0}) \cos \Theta.
% \end{eqnarray}

%If you don't want an equation number, use the star form:
%\begin{eqnarray*}...\end{eqnarray*}

% Break each line at a sign of operation
% (+, -, etc.) if possible, with the sign of operation
% on the new line.

% Indent second and subsequent lines to align with
% the first character following the equal sign on the
% first line.

% Use an \hspace{} command to insert horizontal space
% into your equation if necessary. Place an appropriate
% unit of measure between the curly braces, e.g.
% \hspace{1in}; you may have to experiment to achieve
% the correct amount of space.


%% ------------------------------------------------------------------------ %%
%
%  EQUATION NUMBERING: COUNTER
%
%% ------------------------------------------------------------------------ %%

% You may change equation numbering by resetting
% the equation counter or by explicitly numbering
% an equation.

% To explicitly number an equation, type \eqnum{}
% (with the desired number between the brackets)
% after the \begin{equation} or \begin{eqnarray}
% command.  The \eqnum{} command will affect only
% the equation it appears with; LaTeX will number
% any equations appearing later in the manuscript
% according to the equation counter.
%

% If you have a multiline equation that needs only
% one equation number, use a \nonumber command in
% front of the double backslashes (\\) as shown in
% the multiline equation above.

% If you are using line numbers, remember to surround
% equations with \begin{linenomath*}...\end{linenomath*}

%  To add line numbers to lines in equations:
%  \begin{linenomath*}
%  \begin{equation}
%  \end{equation}
%  \end{linenomath*}
