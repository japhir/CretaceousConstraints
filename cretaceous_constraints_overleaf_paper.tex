%%%%%%%%%%%%%%%%%%%%%%%%%%%%%%%%%%%%%%%%%%%%%%%%%%%%%%%%%%%%%%%%%%%%%%%%%%%%
% adapted from AGUJournalTemplate.tex: this template file is for articles formatted with LaTeX
%
%% To submit your paper:
\documentclass[draft]{agujournal2019}
% packages from the template
\usepackage{url} % this package should fix any errors with URLs in refs.
\usepackage{lineno} % line numbers
\usepackage[inline]{trackchanges} %for better track changes. finalnew option will compile document with changes incorporated.
\usepackage{soul} % hyphenable spacing, underlining, striking out etc.

% IJK added packages
% for units, type \qty{x}{\unit}
\usepackage{siunitx}
% for chemical equations or, in my case, d13C
\usepackage[version=4]{mhchem}
%\usepackage{chemformula} % alternative for above
% \usepackage[utf8]{luainputenc} % allow UTF-8 input? δ13C?
% \usepackage{hyperref} % links between references etc % doesn't seem to work
% with this template?
\usepackage[capitalise,nameinlink,noabbrev]{cleveref}
% use same macro for acronym, this writes out the first use and then
% abbreviates after (use \gls{key}).
\usepackage{glossaries}
%\makeglossaries % not needed if we don't want to print them at the end
% \usepackage[section]{placeins} % figures were jumping to the end, I prefer them closer

\usepackage{threeparttable} % for table footnotes
%\usepackage{colortbl}
%\definecolor{good}{rgb}{0.8,1,0.8}

\newcommand{\appr}{\raise.17ex\hbox{\(\scriptstyle\sim\)}} % approximately symbol
\newcommand{\ma}[1]{Ma\(_{405}\)#1} % Ma\(_{405}\)8 was a hassle to type

\newcommand{\rez}{\textcolor{magenta}}
% I'll use this to highlight sections that I want to focus your attention on!
\newcommand{\ijk}{\textcolor{blue}}

% write Hawaiʻi correctly
% https://tex.stackexchange.com/questions/424535/how-to-type-a-proper-hawai%CA%BBian-%CA%BBokina
\DeclareRobustCommand{\okina}{%
  \raisebox{\dimexpr\fontcharht\font`A-\height}{%
    \scalebox{0.8}{`}%
  }%
}

% commented this out b/c they don't use lualatex or xelatex, so can't use UTF-8 directly (?)
%\newunicodechar{ʻ}{\okina}

% normally I strongly prefer biblatex, but the AGU template allows us to use \cite,
% \citeA, and \nocite from apacite :(
% \usepackage[giveninits=true,uniquename=false,uniquelist=false,date=year,hyperref=true,mincitenames=1,maxcitenames=2,backend=biber,backref,doi=true,url=false,isbn=false]{biblatex}
% \addbibresource{references.bib}

% IJK package config
\sisetup{%
detect-all,% Detect surrounding font context, like weight, italics etc.
%
% Alternative range-phrase:
% en-dash via '--', but inside \text{}, so it's not 'two minus signs'
range-phrase={\,\text{--}\,},
separate-uncertainty=true,
multi-part-units=single,
list-units=single,% single: Print unit only once, at end
range-units=single,% single: Print unit only once, at end
per-mode=symbol,
}%
\DeclareSIUnit\annum{a} % a year, used in years before present
\DeclareSIUnit\year{yr} % a year as a duration
\DeclareSIUnit\millionyearago{\mega\annum} % a time, e.g. 34 million years before present
\DeclareSIUnit\millionyear{\mega\year} % a duration, e.g. the event lasted for 2 million years
\DeclareSIUnit\kiloyearago{\kilo\annum} % a time, e.g. 14 thousand years ago (before present)
\DeclareSIUnit\kiloyear{\kilo\year} % a duration, e.g. the event lasted 200 thousand years

% this makes it so the first use spells it out, the next uses the abbreviation!
\newacronym{d13C}{\ensuremath{\delta}\ce{^13C}}{carbon isotope ratio}
\newacronym{MS}{MS}{magnetic susceptibility}
\newacronym{L*}{\(L^*\)}{total light reflectance}

\newacronym{RMSD}{RMSD}{root mean square deviation}
% do we need RCSSD?
\newacronym{VLN}{VLN}{very long eccentricity node}
% SOME: we may need a SEC and LEC for short eccentricity and long eccentricity? Or maybe e_{l} and e_{s}
% \newacronym{SEC}{\ensuremath{e_{s}}}{short eccentricity}
% \newacronym{LEC}{\ensuremath{e_{l}}}{long eccentricity}

\newacronym{KT}{KTB}{Cretaceous--Tertiary boundary}
\newacronym{PETM}{PETM}{Paleocene--Eocene thermal maximum}

\newacronym{GAM}{GAM}{generalized additive model}
\newacronym{FFT}{FFT}{fast Fourier transform}
\newacronym{MTM}{MTM}{multi-taper method}
\newacronym{MTLS}{MTLS}{multi-taper averaged Lomb-Scargle periodogram of (un)evenly spaced data}

\linenumbers
%%%%%%%
% As of 2018 we recommend use of the TrackChanges package to mark revisions.
% The trackchanges package adds five new LaTeX commands:
%
%  \note[editor]{The note}
%  \annote[editor]{Text to annotate}{The note}
%  \add[editor]{Text to add}
%  \remove[editor]{Text to remove}
%  \change[editor]{Text to remove}{Text to add}
%
% complete documentation is here: http://trackchanges.sourceforge.net/
%%%%%%%

% \draftfalse

%% Enter journal name below.
%% Choose from this list of Journals:
%
% JGR: Atmospheres
% JGR: Biogeosciences
% JGR: Earth Surface
% JGR: Oceans
% JGR: Planets
% JGR: Solid Earth
% JGR: Space Physics
% Global Biogeochemical Cycles
% Geophysical Research Letters
% Paleoceanography and Paleoclimatology
% Radio Science
% Reviews of Geophysics
% Tectonics
% Space Weather
% Water Resources Research
% Geochemistry, Geophysics, Geosystems
% Journal of Advances in Modeling Earth Systems (JAMES)
% Earth's Future
% Earth and Space Science
% Geohealth
%
% ie, \journalname{Water Resources Research}

\journalname{Paleoceanography and Paleoclimatology}


\begin{document}

%% ------------------------------------------------------------------------ %%
%  Title
%
% (A title should be specific, informative, and brief. Use
% abbreviations only if they are defined in the abstract. Titles that
% start with general keywords then specific terms are optimized in
% searches)
%
%% ------------------------------------------------------------------------ %%

% previously: Cretaceous Constraints on Astronomical Solutions
\title{An Astronomically Tuned Time Scale for the Latest Cretaceous (Maastrichtian)}
% technically, Maastrichtian would be more appropriate but that doesn't alliterate ;-)

%% ------------------------------------------------------------------------ %%
%
%  AUTHORS AND AFFILIATIONS
%
%% ------------------------------------------------------------------------ %%

% List authors by first name or initial followed by last name and
% separated by commas. Use \affil{} to number affiliations, and
% \thanks{} for author notes.
% Additional author notes should be indicated with \thanks{} (for
% example, for current addresses).

% Example: \authors{A. B. Author\affil{1}\thanks{Current address, Antartica}, B. C. Author\affil{2,3}, and D. E.
% Author\affil{3,4}\thanks{Also funded by Monsanto.}}

\authors{I. J. Kocken\affil{1} and R. E. Zeebe\affil{1}}
\affiliation{1}{
School of Ocean and Earth Science and Technology,
University of Hawai\okina{}i at M\=anoa,
1000 Pope Road, MSB 629, Honolulu, HI 96822, USA}

% NOTE: this is not in the template, but useful for VC
\today{}

% \affiliation{2}{Second Affiliation}
%(repeat as many times as is necessary)

%% Corresponding Author:
% Corresponding author mailing address and e-mail address:

% (include name and email addresses of the corresponding author.  More
% than one corresponding author is allowed in this LaTeX file and for
% publication; but only one corresponding author is allowed in our
% editorial system.)

\correspondingauthor{Ilja Kocken}{ikocken@hawaii.edu}

%% Keypoints, final entry on title page.

%  List up to three key points (at least one is required)
%  Key Points summarize the main points and conclusions of the article
%  Each must be 140 characters or fewer with no special characters or punctuation and must be complete sentences

% Example:
% \begin{keypoints}
% \item	List up to three key points (at least one is required)
% \item	Key Points summarize the main points and conclusions of the article
% \item	Each must be 140 characters or fewer with no special characters or punctuation and must be complete sentences
% \end{keypoints}

\begin{keypoints}
% I don't particularly like these yet...
\item We use geological data from the Latest Cretaceous to constrain the chaos of the solar system to understand Earth's orbital history.
\item Sections from Zumaia and Sopelana are most compatible with solution ZB20a for the Maastrichtian.
\item This study provides an astronomical time scale for applications in geology and paleoclimate.
% up to three
% each 140 characters or less
\end{keypoints}

%% ------------------------------------------------------------------------ %%
%
%  ABSTRACT and PLAIN LANGUAGE SUMMARY
%
% A good Abstract will begin with a short description of the problem
% being addressed, briefly describe the new data or analyses, then
% briefly states the main conclusion(s) and how they are supported and
% uncertainties.

% The Plain Language Summary should be written for a broad audience,
% including journalists and the science-interested public, that will not have
% a background in your field.
%
% A Plain Language Summary is required in GRL, JGR: Planets, JGR: Biogeosciences,
% JGR: Oceans, G-Cubed, Reviews of Geophysics, and JAMES.
% see http://sharingscience.agu.org/creating-plain-language-summary/)
%
%% ------------------------------------------------------------------------ %%

%% \begin{abstract} starts the second page

\begin{abstract}
% this is a first draft!
%Problem
Astronomical solutions form the backbone of accurate dating for geology and paleoclimate studies.
Beyond \appr\qty{50}{\millionyearago}, however, the chaos inherent in the solar system makes it impossible to calculate a unique astronomical solution.
Geological data have been used to constrain the chaos in order to arrive at an astronomically calibrated time scale up to the end-Cretaceous.
%Action
Here, we adopt and extend this approach into the latest Cretaceous, by re-analyzing the Zumaia and Sopelana composite proxy records from the Maastrichtian.
%Results
We find that the filtered \gls{L*} record is most compatible with the astronomical solution ZB20a.
However, the results are sensitive to parameter choices in our algorithm, which we describe in detail.
Nevertheless, we present evidence in favor of using solution ZB20a for cyclostratigraphic applications during the latest Cretaceous.
Intervals with \acrfullpl{VLN} (low amplitude in the short eccentricity cycle) in the astronomical solutions that coincide with large amplitudes in the short eccentricity-related peaks in the filtered proxy record rule out alternatives.
% \ijk{need concluding sentence on extending the astronomical time scale?}
% decided to not focus on the \gls{KT} age for now.
% 250 words max
\end{abstract}

\section*{Plain Language Summary}
Astronomical solutions are computations of the history of the solar system over millions of years.
They have provided a framework for precise and accurate dating of geological records.
However, as we move to older time periods---beyond \appr\num{50} million years ago---it becomes impossible to establish a single unique solution from computations alone.
This is because minor changes in initial conditions result in very different outcomes, which is referred to as the chaos in the solar system.
In this study, we use geological data from Zumaia and Sopelana to put constraints on the phasing of Milankovi\'{c} cycles.
Furthermore, the occurrence of \acrfullpl{VLN} can be ruled out when the data show a large amplitude in the short eccentricity cycle.
We find that, while sensitive to parameter choices, the ZB20a solution best matches the data from the Maastrichtian.
This study helps us extend the astronomical time scale, and to explore beyond the predictability horizon of the solar system.
% \200 words max

%%% Suggested section heads:
% \section{Introduction}
%
% The main text should start with an introduction. Except for short
% manuscripts (such as comments and replies), the text should be divided
% into sections, each with its own heading.

% Headings should be sentence fragments and do not begin with a
% lowercase letter or number. Examples of good headings are:

% \section{Materials and Methods}
% Here is text on Materials and Methods.
%
% \subsection{A descriptive heading about methods}
% More about Methods.
%
% \section{Data} (Or section title might be a descriptive heading about data)
%
% \section{Results} (Or section title might be a descriptive heading about the
% results)
%
% \section{Conclusions}

%%% IJK some notes on how I write special symbols and units here:

% how do I want to write d13C?
% \(\delta^{13}\)C % plain
% \ce{\delta^13C} % mhchem <----
% \(\delta\)\ch{^{13}C} % chemformula
% \gls{d13C} % using glossaries and mhchem  <-----

% test my acronyms
% \Gls{MS}
% \Gls{d13C}
% \Gls{L*}
% \Gls{KT} boundary

% test my units. Define them the first time we use them too.
% \qty{5}{\kiloyear}
% \qtylist{2;3;5}{\kiloyear}
% \qtyrange{2}{35}{\kiloyear}
% \qty{66}{\millionyearago}


\section{Introduction}\label{sec:intro}

% Astronomical solutions have improved dating, form the backbone of cyclostratigraphy,
Since the breakthroughs that proved the astronomical theory of the ice ages~(\citeNP<i.e.,>{Emiliani1955,Hays1976};~\citeNP<see>[for review]{Hilgen2010})
\ijk{[I figured out how to add Hilgen2010 for review note using \texttt{citeNP}, but can also remove.]}
% not sure I like the way this is cited, but with apacite I don't know how to put the post-script only for Hilgen2010.
cyclostratigraphy and astrochronology have become essential components of the youngest part of the Geologic Time Scale~\cite<e.g.,>[]{Hilgen2006,Hinnov2018,Meyers2019}.
The aim of astrochronology is to link quasi-periodic signals in geological data to astronomical solutions, providing absolute ages with high precision.
These precise ages allow us to understand the causality of climate events, which ultimately give insights into the mechanisms that drive changes in Earth's climate.

% but CHAOS
% @REZ: Wikipedia has Milankovitch forcing, but write his name like below.
While many of the prominent Milankovi\'{c} cycles affect Earth's climate in a quasi-periodic manner over the past tens of millions of years (\si{\millionyear}),
beyond \appr\num{50} million years ago (\si{\millionyearago}) the calculations of the astronomical solutions result in different outcomes with minor perturbations in initial conditions or parameter values~\cite<e.g.,>[]{Laskar2004,Zeebe2017}.
The chaos is inherent to the solar system, so this uncertainty cannot be resolved by improvements in the astronomical calculations alone.

% how does the chaos manifest in the orbital solutions?
The chaos manifests itself in different astronomical solutions and is expressed, for example, in the phasing of the different cycles.
The short eccentricity cycle (periods of \qtylist{95;99.7;124;132.5}{\kiloyear} for \qtyrange{56}{80}{\millionyearago} in solution ZB20a)
% \rez{should be ca. 95,99,124,131} \ijk{see email}
is variable between different astronomical solutions due resonance transitions related to \(g_{4}\) and \(g_{3}\), which are ``loosely related to the perihelion precession of Earth's and Mars' orbits''~\cite{ZeebeLourens2022EPSL}.
% \ijk{[Should I explain further that the short ecc is made up of 4 spectral peaks, related to different combinations of g2, g3, g4, g5? Or is out of scope/can we expect this background knowledge? I separate regular short ecc timing from VLNs here, but maybe they should be in one paragraph?}
% \rez{if you need it below, yes. else no}
% importance: different solutions have different phases of even 405 kyr cycle!
When the individual components that make up the short eccentricity cycle interfere to cancel each other out (with a frequency of \(g_{4} - g_{3}\)), only the long eccentricity component remains clearly visible~\cite<see>[]{ZeebeLourens2022EPSL}.
These intervals are named \glspl{VLN} and occur with a period of \appr\qty{2.4}{\millionyear} in the recent past.
Chaotic resonance transitions can change the timing of these \glspl{VLN} so that they occur with a period of \ijk{I put up to here b/c it can have a lot of values in between, right? rm: about b/c \appr already does this]} \appr\qty{1.2}{\millionyear} instead~\cite{Laskar2011a,ZeebeLourens2019}.
% \ijk{[Frits really thought that the implications of this for absolute astronomical ages to Ar/Ar and Ur/Pb ages are the frontier of the \gls{KT} right now, should we do something with this?]} \rez{[agreed, clearly interesting but not focus of this study]}
% \ijk{ok commented out}

The long eccentricity cycle (\(g_{2}-g_{5}\), period of \qty{405}{\kiloyear}), on the other hand, is thought to be stable over much longer time scales~\cite{Laskar2004}.
Many studies have suggested that because the long eccentricity cycle is stable in the geologic past, we can rely on it to generate a floating chronology for records older than \appr\qty{40}{\millionyearago}~\cite<e.g.,>{Westerhold2012,Westerhold2017,Dinares-Turell2013}.
While the long eccentricity cycle is much more stable than the short eccentricity cycle, we do observe differences in the phasing of the \qty{405}{\kiloyear} cycle when comparing different astronomical solutions further back in time.
% taner filter \qty{\frac{1}{405}}{\per\kiloyear} \pm 25%
% calculated this myself
For example, the long eccentricity minimum near \qty{65.9}{\millionyearago} differs by up to \qty{68}{\kiloyear} between solutions La11 and ZB20a~\cite<recalculated after>{ZeebeLourens2022EPSL}.
% \ijk{[@REZ: Table B.1 says 65.91 for ZB20a, 65.97 for La11, implied diff of 60 kyr.
%      When I calculate it myself with nice taner filters targeting LEC
%      I get ZB20a 65.898, La11 65.966, diff of 67.99999 kyr]}
% \rez{remove comment, close enough}
%At the long eccentricity minimum around \qty{69.8}{\millionyearago}
%between \ma{13} and \ma{14} (numbers represent the number of the long eccentricity bundle older than the \gls{KT}),
%the age of the \qty{405}{\kiloyear} minimum already differs by up to \qty{85.2}{\kiloyear} between different astronomical solutions---between astronomical solutions La10c~\cite{Laskar2011} and ZB20a~\cite{ZeebeLourens2022EPSL}.
This means that providing absolute ages of Maastrichtian records based on tuning to the long eccentricity cycle of any particular astronomical solution %without first constraining the chaos in this interval with geological records,
may underestimate the true uncertainty.
Over longer geological time scales (several hundred \unit{\millionyear}) the long eccentricity cycle was recently also shown to be unstable~\cite{ZeebeLantink2024}.


% constrain the chaos with geologic data
Previous workers have attempted to put geological constraints on the chaos for the Paleocene and Eocene~\cite{ZeebeLourens2019,ZeebeLourens2022EPSL}
or identified \ijk{the} most suitable existing solutions~\cite{Westerhold2017}.
These studies shed light on the occurrence of chaotic resonance transitions that shift the timing of \glspl{VLN} and long- and short eccentricity.
The Walvis Ridge records used in \citeA{ZeebeLourens2019,ZeebeLourens2022EPSL} are of exceptional quality.
Finding comparable records that capture precession-scale alternations and patterns of amplitude modulation for older time periods will likely become increasingly challenging.

In this work, we use the best available records immediately prior to the \gls{KT}\ijk{[gls command keeps track of first use of acronyms, see PDF]} (informal use) to constrain which astronomical solutions most accurately reflect Earth's orbital past.
%for the Maastrichtian (data falls between \qtyrange{71.1}{65.9}{\millionyearago}).
We argue that these records are the Zumaia and Sopelana proxy records generated by \citeA{Batenburg2012,Batenburg2014} (see \cref{sec:data}).
These study sites have been previously recommended to serve as primary candidates to help select astronomical solutions~\cite{Dinares-Turell2013}.
The goal of the present study is to provide an astronomical time scale for applications in geology and paleoclimate in the Maastrichtian.


\section{Material and Methods}\label{sec:mm}

%% Decide on consistent naming for
%% scaled vs.\ <----
%% normalized vs.\
%% standardized:
%% subtract mean, divide by standard deviation.

\subsection{Datasets Used}\label{sec:data}

In order to find the most suitable Maastrichtian proxy record that best reflects amplitude variations in long- and short eccentricity cycles,
we have considered several coastal sections of marine successions and sea floor drill core sites.
Proxy records from the following sites and studies were considered:
Gubbio~\cite{Voigt2012,Sinnesael2016}, % Fig. 12, d13C low-res
ODP Leg 208 Site 1267~\cite{Westerhold2008,Husson2011}, % log(Fe) only Ma405-1 and Ma405-2, MS up to Ma405-6 but no splice
IODP Leg 342 Site U1403~\cite{Batenburg2018}, % why didn't we look at this in more detail again? It is one of the studies listed in \cite{Smith2023}.
Zumaia and Sopelana~\cite{tenKateSprenger1993,Batenburg2012,Batenburg2014,Dinares-Turell2013}, % <3
ODP Leg 198 Site 1210B~\cite{Jung2012,Kim2022}, % longer record but lower resolution and d13C. Also cited in Voigt2012
ODP Leg 74 Site 525A~\cite{Husson2011}, % Fig. 3 grey levels smoothed, Ma405-6--9, Fig. 4 only Ma405-1
ODP Leg 207 Site 1258A~\cite{Husson2011}, % Fig 3. MS Ma405-8 to 14, some core gaps (also at Ma10 unfortunately)
and ODP Leg 122 Site 762C~\cite{Husson2011,Thibault2012}. % Fig 3. grey levels smoothed \ma{8--16}, Fig. 4 \m{1--6}
For the present study, we selected the Zumaia and Sopelana I\rez{???}\ijk{[it's what they call it in Dinares-Turell2013, there's apparently also a Sopelana II section]} field sections,
which show the clearest orbitally-forced patterns for the Maastrichtian~\cite{tenKateSprenger1993,Batenburg2012,Dinares-Turell2013},
and continuous high-resolution proxy records are available~\cite{Batenburg2012,Batenburg2014}.

% some setting/why is the site so epic?
The coastal cliffs of Zumaia and Sopelana are located in the Bay of Biscay in the Basque Country, Northern Spain~\cite{Herm1965,Pujalte1993}
and comprise quasi-periodic alternations between limestones and marls, interbedded with turbidites~\cite{tenKateSprenger1993,Pujalte1999}. % this is cited as Pujalte 1998 in Batenburg2012, year off-by-one!
\citeA{Dinares-Turell2013} concluded that these successions could play ``a primary role as a geologic aid for critical developments of the astronomical solutions''.
% does the below add anything? unfortunately the quote is also quoted on wikipedia...
The Zumaia section is one of the first 100 geological heritage sites, which was described as ``One of the best exposed, most continuous and highly studied outcrops of deep marine sediments in the world'' \ijk{(\citeNP[pp.~71--72]{Hilario2022}; \citeNP<see>[and references therein for some historical context of the study sites]{Batenburg2014})}.
% this has better refs than their 2012 paper I think.
\ijk{[wasn't sure if you meant like this, inside parens, or as a separate sentence outside parens]}
% quote from Batenburg2014: (Herm 1965; Percival & Fischer 1977; Lamolda 1990; Ward et al. 1991; Ten Kate & Sprenger 1993; Ward & Kennedy, 1993; elorza & García-Garmilla 1998; Pujalte et al. 1998; Dinarès-Turell et al. 2003, 2013; Gómez-Alday et al. 2008; Kuiper et al. 2008) and a reference section for the K–Pg boundary (Molina et al. 2009).
% they also say ten Kate & Sprenger 1993 first cyclostrat paper of the site.

% careful: turbidites
Care must be taken when considering the lower Maastrichtian at the Zumaia site due to frequent turbidite successions,
while the upper Maastrichtian was tectonically more stable~\cite{Pujalte1999}.
The upper Maastrichtian section of Zumaia also contains many thin turbidites,
whose occurrence results in many random spectral peaks, but may also include an eccentricity and obliquity signal for cycles \ma{1--4}~\cite<the first four long eccentricity bundles prior to the \gls{KT}; Ma refers to Maastrichtian here,>{tenKateSprenger1993}. % this part is not mentioned in Batenburg
However, the pattern of limestone--marl alternations in the Zumaia succession that has been associated with astronomical forcing remained undisturbed by the turbidites in the upper Maastrichtian~\cite{Batenburg2012}.
The Sopelana section covers the lower part of the Maastrichtian, and has a similar lithology but without indications for turbidites, likely because it was located farther offshore~\cite{Batenburg2012}.

% lithology/cycles: summary of Batenburg2012 table 1
The alternations in color and bed resistance that occurred at periods of \appr\qty{70}{\cm} were logged in detail~\cite{Batenburg2012,Dinares-Turell2013}
and have been associated with climatic precession forcing (periods of \appr\qtyrange{19}{23}{\kiloyear}).
Bundles of five of these couplets with darker, redder, thicker beds---especially in the marls---were related to short eccentricity amplitude modulation (\appr\qty{4}{\metre}, \appr\qty{100}{\kiloyear}),
which in turn were grouped in four bundles that were associated with the long eccentricity cycle (\appr\qty{16}{\metre}, \appr\qty{405}{\kiloyear}) and expressed as limited thinner marlier parts~\cite{Batenburg2012,Batenburg2014}.

% phase relationship
Eccentricity modulates the amplitude of climatic precession, with eccentricity maxima and strong precession minima linked to enhanced seasonal contrast in the Northern hemisphere.
This is thought to intensify the hydrological cycle in the region, resulting in the deposition of thicker, darker marls~\cite{Batenburg2014}.

% which proxies?
\citeA{Batenburg2012} measured the \gls{MS}, the \gls{d13C}, and the \acrfull{L*} of rock samples (among others) from Zumaia, but we focus on \gls{L*} in the main manuscript (as explained below).
The \citeA{Batenburg2012} proxy record was extended downwards to include Sopelana I in \citeA{Batenburg2014}.
Unfortunately, the \ce{CaCO3} record from \citeA{tenKateSprenger1993} currently spans bundles \ma{1--4} \ijk{(their Fig.~7)} and has not yet been extended to encompass the full Maastrichtian.
The \gls{d13C} proxy showed long eccentricity-related cycles, but the short eccentricity cycles were less clearly expressed.
This bias to lower frequencies of the astronomical forcing in the carbon isotope record can be explained by the long residence time of carbon in the oceans~\cite{Zeebe2017,Kocken2019loscar}.
% Richard doesn't trust/understand MS.
The \gls{MS} signal is affected by dilution with carbonates~\cite{tenKateSprenger1993},
so its link to climate forcing is less direct than, for example, the \ce{CaCO3} content itself.
% it seems that the \citeA{tenKateSprenger1993} record has a _--_ pattern for the amplitudes of the short ecc in the first 4 long ecc cycles prior to the K/T
% \citeA{Dinares-Turell2013} re-analyzed this record and did some additional spectral analysis/filtering.
The \gls{MS} proxy shows a very strong negative correlation with \ce{CaCO3} in bundles \ma{1--4}~\cite{tenKateSprenger1993,Gilabert2022}.
However, clear differences were found between the intercept of this relationship before, during, and after the \gls{KT}~\cite<supplementary figure S2 in>{Gilabert2022}.
Furthermore, \citeA{Batenburg2012} report that their \gls{MS} values show scatter, potentially due to ``surface irregularities and the generally low values''.
The \gls{L*} proxy, on the other hand, likely corresponds to changes in the lithology that were driven by carbonate--clay alternations~\cite{MountWard1986,Batenburg2012},
which were in turn driven by changes in wetness and climate.
We therefore argue that the color reflectance \gls{L*} record most directly reflects the orbital forcing.

% commented out now, just put the relevant refs up above!
% \ijk{Just went through the table again, 1267 still looks promising and so does U1403.}
% \begin{table}
%     \centering
%     \caption{\label{tab:sites}
%         Overview of sites considered for study, sorted from younger to older, shorter to longer.
%         Gubbio consists of Contessa Highway (Danian) and Bottaccione (Maastrichtian).
%     }
%     \begin{tabular}{llccl}
%         Site & Proxy & Ma\textsubscript{405}x & Age (Ma) & Reference \\
%         % the ages and Ma405 cycles are pretty rough estimates from figures/raw data
%         \hline
%         %208-1262 & a*, d13C, d18O & - & 58--53 & \citeA{ZeebeLourens2019} \\
%         % should probably cite original study but won't show in final table here anyway
%         %198-1209 & a*/b* & - & 66--56 Ma & \citeA{ZeebeLourens2022EPSL}\\
%         %Hendaye & section photos & - & 66--64 & Hilgen, pers. comm.\\
%         % Frits says it has a 200 kyr cycle that is unique to this interval!
%         Gubbio & \gls{d13C}, MS & 1--12 & 76--66 & \citeA{Voigt2012}\\
%         % too low-res? Contrasts to La10d
%         % Gubbio & Ca isotopes & - & OAE-2 & \citeA{Kitch2022}\\ % low-res, old, wrong proxy
%         Gubbio & MS, \ce{CaCO3}, d13C, d18O & - & 67--62.5 & \citeA{Sinnesael2016} \\
%         % bottaccione and contessa highway
%         % short, but Zumaia doesn't have much signal here
%         % may be worth another look! Contrasts to La11
%         208-1267 & color & 1--6/7 & 58--53 & \citeA{Westerhold2007} \\
%         % a* in \citeA{Westerhold2007}, mostly focused on PETM--Elmo
%         % extended in \cite{Batenburg2018} Newsletters on Strat. (56.042 Ma to 69.070 Ma).
%         % also related: [cite:@Zachos2004]
%         208-1267 & ln(Fe) & 1--2 & 66.5--55.0 & \citeA{Westerhold2008} \\ % very very strong short cycle-- I'm not seeing this anymore.
%         208-1267 & MS & 1--6/7 & 68.6--66 & \citeA{Husson2011} fig. 3 and 4. \\
%         % raw MS data from Blum2005 https://doi.pangaea.de/10.1594/PANGAEA.266605
%         % Husson2011 uses MBSF but need to use RMCD -> core gaps etc.!
%         % splice table from \cite{Westerhold2008} PDF https://doi.pangaea.de/10.1594/PANGAEA.592301
%         % I spent some time reworking this!
%         %% 208-1267 & \ce{CaCO3} & - & 50-47.8 & \citeA{Sexton2011}\\ % too young
%         342-U1403 & MS, ln(Fe/Ca) & 1--7 & 68.8--66 & \citeA{Batenburg2018}\\
%         % not sure why I didn't pick the one above
%         Zumaia & \ce{CaCO3} & \(-\)3--3 & 67.5--65.5 & \citeA{tenKateSprenger1993}\\ % too short?
%         % also shown in \cite{Westerhold2008}, next to 1267
%         Zumaia & L*, MS, d13C & 1--9 & 70--66 & \citeA{Batenburg2012} \\
%         Sopelana & L*, MS & 9--13 & 71.1--69.6 & \citeA{Batenburg2014}\\
%         198-1210B & d13C, d18O, XRF Ba & 2--12 & 71.5--66.25 & \citeA{Jung2012,Kim2022}\\
%         % some low-res intervals in crucial parts?
%         % higher res around 68 Ma, Kim2022 pretty high res but only d13C
%         74-525A & grayscale & 6--8.5 & 69--67.8 & \citeA{Husson2011}\\
%         % have emailed requesting data, but has left academia
%         122-762C & grayscale & 8--17+ & 77.5--69 & \citeA{Husson2011,Thibault2012}\\
%         % Thibault has published most data and sent me some upon request!
%         % the above may be useful to look at:
%         % some core gaps but high amplitude short ecc in Ma405_8 and Ma405_10!
%         207-1258A & MS & 8--14 & - & \citeA{Husson2011}\\
%         % raw data not available, but can reproduce by getting data from JANUS
%         % database
%         % taner filters not so nice, some big core gaps
%         % Gubbio & d13C, d18O & - & 84.2--72.1 & \citeA{Voigt2012}\\
%         % Gubbio & d13C & - & 84.2--72.1 & \citeA{Sabatino2018}\\ % high-res
%         % Furlo & color, d13C, d18O & - & 96--90 & \citeA{Batenburg2016}\\ % too old for us, but cool for future?
%         % Levant Platform & TOC, d13C, d18O & - & 96.2--90.9 & \citeA{Wendler2014}\\
%     \end{tabular}
% \end{table}

\subsection{Age Model}\label{sec:agemodel}

The age model for the Zumaia and Sopelana sites was based on the identification of long eccentricity minima in the field by \citeA{Batenburg2012,Batenburg2014}.
The original study relied on foraminifera biostratigraphy, magnetostratigraphy, and astrochronology by tuning the long eccentricity minima to the long eccentricity filter of the La11 solution~\cite{Laskar2011a}. % not sure if sentence is needed
In order to avoid circularity, we assumed a duration of \qty{405}{\kiloyear} for each of these long eccentricity cycles to arrive at an initial floating age model.
A previous best-estimate of the age of the \gls{KT} \cite<\qty{65.9}{\millionyearago},>[]{ZeebeLourens2022EPSL} anchored the floating age model to absolute time.
We then performed optimizations for each astronomical solution that allowed the age of the \gls{KT} to shift by \qty{\pm200}{\kiloyear}
and to adjust the depth for each of the long eccentricity minima tie-points (see \cref{sec:algorithm} for details).
Note that this final step is crucial, because for the astronomical solutions under evaluation, the duration of the long eccentricity cycle varied between \qtyrange{389.0}{419.2}{\kiloyear} between \qtyrange{71}{65}{\millionyearago} (approximately \(\pm\)\qty{4}{\percent}).
To substantiate the link between the quasi-periodic alternations in the record and astronomical forcing,
we rely on the spectral analyses and discussion in \citeA{Batenburg2012,Batenburg2014}.

\subsection{Astronomical Solutions}\label{sec:astro}

In this study we compare the available astronomical solutions that had previously shown the best matches to data from the Paleocene and Eocene \cite{ZeebeLourens2019,ZeebeLourens2022EPSL}.
This includes solutions La10b and La10c~\cite{Laskar2011}
as well as solutions ZB18a~\cite{ZeebeLourens2019}
and ZB20a, ZB20b, ZB20c, and ZB20d~\cite{ZeebeLourens2022EPSL}.
\cref{tab:astronomical-solutions} shows the different parameter variations that were used to generate the latter five astronomical solutions.

Note that while a previous study indicated a preference for astronomical solutions La10d and La11 over La10a, La10b, and La10c based on data from the Eocene~\cite{Westerhold2012},
a later study by some of the same authors revisited these results and recommended La10b and La10c instead~\cite{Westerhold2017}.
We have excluded La10a, La10d, and La11 because of their bad match with data from ODP Site 1262 during the Paleocene~\cite{ZeebeLourens2022EPSL}.

% tried to do this using guidelines from template but it messes things up
% so using threeparttable instead
% \begin{table}
% \centering
% \caption{Properties of astronomical solutions.$^a$\label{tab:astronomical-solutions}}
% \begin{tabular}{lcccc}
%  & \(\Delta{}t\) (days) & \(J_{2}^{b}~\times10^{7}\) & \(N_{\text{ast}}\) & \(N_{\text{LWP}}\) \\
% \hline
% ZB18a & 2 & \num{1.3050} & 10 & 0 \\
% ZB20a & 2 & \num{1.4700} & 50 & 40 \\
% ZB20b & 2 & \num{1.3310} & 10 & 0 \\
% ZB20c & 2 & \num{1.1708} & 10 & 0 \\
% ZB20d & 6 & \num{1.3050} & 33 & 33 \\
% \hline
% \multicolumn{2}{l}{Table adapted from \citeA{ZeebeLourens2022EPSL}.}
% \multicolumn{2}{l}{$^a$\(\Delta{}t\) = timestep, \(J_{2}\) = solar quadrupole moment, \(N_{\text{ast}}\) = N\textsuperscript{o} of asteroids, NLWP = N\textsuperscript{o} of lightweight particles.}
% \multicolumn{2}{l}{$^b$For discussion of \(J_{2}\) values, see \citeA{ZeebeLourens2022EPSL} section A1.}
% \end{tabular}
% \end{table}

\begin{table}
\begin{threeparttable}
\caption{Properties of astronomical solutions.\tnote{a}\label{tab:astronomical-solutions}}
\centering
\begin{tabular}{lcccc}
 & \(\Delta{}t\) (days) & \(J_{2}\tnote{b}~\times10^{7}\) & \(N_{\text{ast}}\) & \(N_{\text{LWP}}\) \\
\hline
ZB18a & 2 & \num{1.3050} & 10 & 0 \\
ZB20a & 2 & \num{1.4700} & 50 & 40 \\
ZB20b & 2 & \num{1.3310} & 10 & 0 \\
ZB20c & 2 & \num{1.1708} & 10 & 0 \\
ZB20d & 6 & \num{1.3050} & 33 & 33 \\
\end{tabular}
\begin{tablenotes}
  \item Table adapted from \citeA{ZeebeLourens2022EPSL}.
  \item [a] \(\Delta{}t\) = timestep, \(J_{2}\) = solar quadrupole moment, \(N_{\text{ast}}\) = N\textsuperscript{o} of asteroids, \(N_\text{LWP}\) = N\textsuperscript{o} of lightweight particles.
  \item [b] For discussion of \(J_{2}\) values, see \citeA{ZeebeLourens2022EPSL} section A1.
  \ijk{[comment on higher Nast and Nlwp in discussion?]}
\end{tablenotes}
\end{threeparttable}
\end{table}





\subsection{Removing Long-Term Trends}\label{sec:detrend}

Several rapid shifts in the lithology---from red, clay-rich marly intervals to whiter limestones---were observed in the Zumaia section on the order of every \qty{50}{\metre}.
This \qty{50}{\metre} cycle has previously been suggested to potentially indicate a \qty{1.2}{\millionyear} cycle~\cite{Batenburg2014}.
The shifts in lithology were recorded in the proxy archives of Zumaia as very rapid transitions, likely as a result of crossing some threshold value in the nonlinear climate system response.
The long-term shifts in lithology could hamper spectral analysis and potentially filtering.
Therefore we used several methods of detrending the records prior to filtering: We
\begin{enumerate}
    \item Fit a linear model to the record, subtracted the fit from each value, then scaled (subtracted the mean and divided by the standard deviation) the result;
    \item Fit a \gls{GAM} to the record, subtracted the fit from each value, then scaled the results;
    \item Applied a lowpass filter to the record with a frequency of \qty{0.025}{\metre\per cycle}, subtracted it from each value, then scaled the results;
and
    \item Applied a piecewise linear fit based on depth intervals where the lithology changed substantially.
    The linear fits were subtracted from each value and the results were scaled.
\end{enumerate}

We have tried multiple ways of detrending the records, for example with larger and smaller chunks in the depth domain or with different frequencies for the lowpass filter, and report sensitivity to the detrending in the appendix.
% The effects that these trends can have on our results were limited because we filter only in the time-domain, after applying the age model. % Not sure if true, commented out for now.

% where do we talk about filtering/spectral analysis?
% \rez{you could pad more 5x, 10x, any difference?}
% \ijk{I didn't read the docs correctly, \texttt{padfac} defaults to 5.
%     I quickly tried out 5, 10, 15, 100 times and no diff for 405 and 100 kyr filters.
%     Also tried different padfacs for taner, it just becomes way slower but other than that it's nearly identical once you have over 3x the data.}

\subsection{Finding the Best Fit}\label{sec:algorithm}

In order to test which astronomical solution provides the best match to the data
(and hence is more likely to reflect the true history of the solar system)
we adopted and extended the approach of \citeA{ZeebeLourens2019,ZeebeLourens2022EPSL}.
We applied the initial age model (\cref{sec:agemodel}) to the detrended data (\cref{sec:detrend}).
After linearly interpolating to a timestep that is a multiple of the timestep of the astronomical solution,
we filtered out the long and short eccentricity cycles.
We used a taner filter with a roll of \num{e10}, targeting periods of \qtylist{405;110}{\kiloyear} (frequency \qty{\pm25}{\percent}) (\cref{fig:filter-windows}).
These filter parameters were selected after analyzing the sensitivity to different ways of filtering (rectangular, gaussian, different values for the taner filter roll parameter, see \cref{fig:filter-windows,fig:full-RMSD-filter}).
The filtered proxy signals were then summed and scaled (see above).
To improve legibility, we refer to the filtered summed scaled data as `filtered record' throughout the manuscript.
The scaled astronomical solutions were then \ijk{subset}\ijk{[I had subset/interpolated before, but actually I'm just subsetting because I have made sure the sample interval of the data is a multiple of the solution]} to the same timesteps as the data.
Then we calculated how well the filtered records matched the astronomical solution via the \gls{RMSD}:
\begin{equation}\label{eqn:rmsd}
    \text{RMSD} = \sqrt{\frac{1}{n}\sum_{i=1}^{n}(e_{i} - f_{i})^{2}}
\end{equation}
where \(e\) is the scaled eccentricity of the astronomical solution, \(f\) is the scaled sum of long- and short eccentricity-related filters in the data (i.e., the filtered record), \ijk{and \(n\) is the total number of datapoints.}

To study the effect of the two eccentricity cycles on the match with the astronomical solution
we assigned different weights to the two filtered signals
(relative weights of 1:0, 1:0.25, 1:0.5, 1:0.75, 1:1, 0.75:1, 0.5:1, 0.25:1, and 0:1 for the long and short eccentricity cycles respectively).
In the main manuscript, we limit our analysis to the simplest 1:1 combination because this preserves the relative amplitude of the long and short eccentricity components from the original data.

An updated age model was created by shifting the age of the \gls{KT}
by up to \qty{\pm200}{\kiloyear} in increments of \qty{2.8}{\kiloyear} (the median sampling rate for this site)
and selecting the offset that gives the lowest \gls{RMSD}.
% This means that a new estimate for the age of the \gls{KT}
% \rez{[this was quite tricky in ZL22. should probably look
% at KTB from both younger and older sides. we can discuss]}
% for each astronomical solution is an additional outcome of this study.
To account for potential errors in the tie-points---the long eccentricity minima depths as identified in the field---we iteratively shifted each tie-point from the youngest to oldest by a range of values between \qtyrange[range-phrase=~to~]{-1.6}{1.6}{\metre} in \qty{20}{\centi\metre} increments (up to \appr\qty{10}{\percent} of the long eccentricity period).
After tweaking the tie-point depth, the updated age model was applied, the record was filtered to target the long and short eccentricity cycles, and they were summed and scaled to calculate the \gls{RMSD} as before.
Once the optimal (lowest \gls{RMSD}) tie-point depth was found, we fixed the depth and moved on to the next-youngest tie-point.
This was repeated until all tie-points were adjusted, resulting in the overall best fit.
We performed this analysis separately for the Zumaia and Sopelana sites, so that potential errors in the depth correlation between sites was accounted for.
Furthermore,  scaling the two sites separately should correct for differences in amplitude between the records that could be the result of differences in paleogeographic setting~\cite<i.e., the Sopelana site was likely located farther offshore than the Zumaia site,>[]{Batenburg2014}.
We also calculated a square root of the cumulative sum of the squared differences that started at the \gls{KT} and moved to older data points.
This allowed us to visualize where in the time domain the data differs most from the astronomical solutions.


\section{Results}\label{sec:results}

% FIGURE 1
\begin{figure}
  \centering
  \includegraphics[width=\textwidth]{Lstar-vs-solutions.png}
  \caption{\label{fig:rolling-depth-age}
    \textbf{Maastrichtian Zumaia (purple) and Sopelana (orange) inverted filtered \gls{L*} records tuned to astronomical solutions (black, first 7 panels).}
    % This uses short linear detrending to correct for changes in sediment composition (\cref{sec:detrend}).
    %We also show Zumaia above and below \qty{109.26}{\metre} to account for the change in the sedimentation rate and lithology.
    Bottom three panels show the record in the depth domain.
    Bottom panel shows the log of the two sites, adapted from \citeA{Batenburg2014}.
    Then \gls{L*} values are shown with different ways of detrending in coloured lines (\cref{sec:detrend}).
    The record after piecewise-linear detrendeding and normalization is shown after that.
    Vertical lines show long eccentricity minima as identified in the field,
    with adjustments of up to \qty{\pm1.6}{\metre} (horizontal segments)
    and how they were matched to each astronomical solution in the time domain to minimize \gls{RMSD} (see \cref{sec:algorithm}).
    }
\end{figure}



% first talk about how the matched filtered records look!
The filtered \gls{L*} record showed reasonable visual congruence with the different astronomical solutions for the Zumaia--Sopelana composite across the Maastrichtian (\cref{fig:rolling-depth-age}). % too nonscientific?
% global patterns we observe
The long eccentricity cycles were closely matched to each solution, while the short eccentricity filter was in general alignment with the solutions in those intervals where the solution showed a large short eccentricity amplitude.

% describe stuff from young to old, from K/T going left
For astronomical solutions ZB18a, ZB20c, and ZB20d, the youngest Maastrichtian long eccentricity bundle \ma{1} showed a single short eccentricity cycle that is either in anti-phase with the filtered record, or has a larger amplitude shortly before the \gls{KT}.
This could be related to the gradual transition that we see in the \gls{L*} data towards the base of the \gls{KT} or to edge effects related to the bandpass filtering.
% should we talk about Ma405 bundle 2  that doesn't have large amplitude in the proxy records,
% but does have large amplitude in the relief in the field/photographs
For the next long eccentricity bundle \ma{2} the filtered record had a relatively low amplitude and so did most astronomical solutions.
% moved below to discussion!
% However, for \ma{2} visual indications based on the relief and contrast from the section photographs in \citeA<>[Fig. 2]{Batenburg2012} and \citeA<>[Fig. 5a]{Dinares-Turell2013} show relatively large amplitude in short eccentricity, with prominent precession-related marls.

\ijk{[should we talk about VLNs here already or only in the discussion?]} \rez{if taken strictly, this seems discussion}
\ijk{[I think I'll leave here fore now, makes results read easier. (Otherwise people might be wondering `why are you talking about large and small amplitudes?')]}
We observed patterns in the relative amplitude of the long eccentricity and short eccentricity filters that may be related to \glspl{VLN}.
It is difficult to distinguish \glspl{VLN} in the filtered record directly, because there are other factors than a \gls{VLN} that can result in a low amplitude in the short eccentricity-related component of the record.
For example, large changes in sedimentation rate or lithology may disrupt the band-pass filtering.
Therefore, a more conservative approach focuses on bundles with a large amplitude in the short eccentricity-related cycle in the data.
The records show a larger amplitude of the short eccentricity-related filter in bundle \ma{5}, matching well with all astronomical solutions under evaluation here (\cref{tab:results}).

Crucially, however, some astronomical solutions have a low amplitude in the short eccentricity component in intervals where the data shows a relatively large amplitude.
This could indicate that a \gls{VLN} was absent from this interval and that the computed astronomical solution was different from the actual astronomical forcing that the Earth experienced.

In bundle \ma{5} (\appr\qty{67.8}{\millionyearago}) ZB20b has a relatively lower amplitude (with a \gls{VLN} in \ma{6}) than our filtered record, whereas ZB20c and ZB20d match the data more closely and ZB20a matches it in amplitude but seems slightly out of phase.
The largest-amplitude interval in the record occurred in bundle \ma{8} (\appr\qty{68.9}{\millionyearago}).
This bundle features a darker marly interval at around \qty{117}{\metre}, which corresponds to a strong excursion in both \gls{L*} and \gls{MS} (in the opposite direction) that closely matches the duration of one short eccentricity cycle.
It is referred to as ``escal\'{o}n'' (step), and was previously also associated with a short eccentricity maximum, while other causes such as regional factors or tectonic events could not be excluded~\cite{Dinares-Turell2013}.
% TODO: double-check the cycle counts between the two studies
% There is a giant offset in the composite depth between Dinares-Turell 2013 and Batenburg 2012/2014. At least 86% more section in DT13.
% Dinares-Turell 2013 depth of 220 m is coincident with Batenburg 2012/2014 depth of 180 m = 40 m
% and 100 m = 120 m. = 20 m
% in Dinares-Turell 2013 the escalón occurs between short E31 and E32 (based on filter of CaCO3 for upper part)
% if I count the short ecc cycles identified in the field by Batenburg 2012, escalón is between E33 and E34
The large amplitude of our filtered record in bundle \ma{8} is significant, because solutions ZB18a, and ZB20c have a low amplitude of short eccentricity in this interval.
La10b may have a \gls{VLN} somewhere between \ma{7} and \ma{8} and has a poor match with the data.

During \ma{10} (\appr\qty{69.7}{\millionyearago}) the composite record also indicates a large amplitude of short eccentricity, resulting in a mismatch with ZB20b and to a lesser extent La10c (where the \gls{VLN} seems to occur between \ma{9} and \ma{10}).

% then describe simple "what is the best RMSD-scoring solution?"
The overall fits between our filtered records and the astronomical solutions (as measured by the \gls{RMSD}) were relatively similar between the different astronomical solutions (\cref{tab:results,fig:full-RMSD-all}).
The best (lowest \gls{RMSD}) match was observed between astronomical solution ZB20a and the filtered \gls{L*} record.
The \gls{MS} proxy matched the La10c and ZB20d solutions best, while
the \gls{d13C}-proxy, as indicated in the methods, does not capture short-eccentricity variability to the extent needed to differentiate meaningfully between astronomical solutions.
Note that the filtered record and the astronomical solutions showed similar patterns when calculating a correlation, with higher correlation coefficients (calculated using \texttt{astrochron::surrogateCor()} for a Pearson, Spearman, and Kendall correlation, not shown) corresponding to lower \gls{RMSD} scores.

\begin{table}
\begin{threeparttable}
\caption{
Ranking of astronomical solutions.
%\gls{RMSD}\tnote{a} score and compatibility with data\tnote{b} for astronomical solutions
\label{tab:results}}
\centering
\begin{tabular}{lccccccc}
 & \multicolumn{3}{c}{\gls{RMSD}~scores\tnote{a}} & \multicolumn{3}{c}{Compatible with data\tnote{b}} & \\
 & \gls{L*}\tnote{c} & \gls{d13C}\tnote{d} & \gls{MS}\tnote{e} & \ma{5} & \ma{8} & \ma{10} & Rank \\
\hline
ZB20a & 1.02 & 1.23 & 1.15 & Yes & Yes & Yes & 1\\
ZB18a & 1.08 & 1.19 & 1.07 & Yes & \textbf{No} & Yes & 2\\
ZB20b & 1.13 & 1.22 & 1.22 & Yes & Yes & \textbf{No} & 3\\
ZB20c & 1.14 & 1.16 & 1.07 & Yes & \textbf{No} & Yes & 4\\
ZB20d & 1.20 & 1.20 & 1.03 & Yes & No\tnote{*} & Yes & 5\\
La10b & 1.09 & 1.18 & 1.09 & Yes & No\tnote{*} & Yes & 6\tnote{f}\\
La10c & 1.11 & 1.17 & 1.03 & Yes & Yes & No\tnote{*} & 7\tnote{f}\\
\hline
\end{tabular}
\begin{tablenotes}
    \item %\gls{L*} \gls{RMSD} scores \(\ge1.1\) and
    \item [a] \Acrlong{RMSD} between normalized solution and normalized filtered record.
    \item [b] Indicates whether the astronomical solution lacks a \acrfull{VLN}, consistent within the interval where the filtered \gls{L*} record shows large amplitude in the short eccentricity related cycle.
    \item [c] \Acrfull{L*}.
    \item [d] \Acrfull{d13C}.
    \item [e] \Acrfull{MS}.
    \item [f] Did not show a \gls{VLN} at around \qty{61}{\millionyearago}, in contrast to the data~\cite{ZeebeLourens2022EPSL}.
    % \item [g] La10a Not compatible with periods older than \appr\qty{47}{\millionyearago}~\cite{Westerhold2012}.
    \item [*] Relatively low amplitude short eccentricity in astronomical solution, \gls{VLN} seems to occur during long eccentricity minimum.
\end{tablenotes}
\end{threeparttable}
\end{table}

% FIGURE 2
\begin{figure}
    \centering
    \includegraphics[width=0.6\textwidth]{full_RMSD_scores_all.pdf}
    \caption{\label{fig:full-RMSD-all} % all being all proxies
      \textbf{Best overall matches.}
        \Acrfull{RMSD} scores (lower is better) of compilation of Zumaia and Sopelana filtered proxy records (colours) against different orbital solutions.
        % Tie-point depths were adjusted to arrive at the best match with each solution (\cref{sec:algorithm}). % not needed?
    }
\end{figure}

The cumulative \gls{RMSD} scores \rez{[rephrase]}\ijk{indicate when our filtered records diverged from the astronomical solutions } (\cref{fig:cum-RMSD-all}).
Solutions ZB18a, ZB20c, and ZB20d all show a rapid increase in the cumulative \gls{RMSD} score at around \qty{65.9}{\millionyearago}, which represents the mismatch in the first short eccentricity cycle just prior to the \gls{KT}.
Overall for \gls{L*}, solutions ZB20a and ZB20b outperformed the other solutions up to \appr\qty{69.6}{\millionyearago}, after which ZB20b also develops a mismatch for the Sopelana site.
Unsurprisingly, the cumulative \gls{RMSD} scores highlight that none of the astronomical solutions showed an amplitude that is as large as the filtered record in \ma{8}, resulting in a rapid increase of the cumulative \gls{RMSD} scores for all astronomical solutions under evaluation for this interval.
\rez{To match the \ma{8} data amplitude would require $e > 0.0\ijk{72}$,
whereas $e \le 0.06\ijk{3}$ in all of our solutions.}
\ijk{[I've just updated above numbers to calculated ones.
Do you want to leave this addition in?]
[Note \(\max{e}\) in La10c \(= 0.0642\), La11 \(= 0.0644\), La10b \(= 0.0645\) for this time interval.]}

The rolling \gls{RMSD} score only increased slightly in \ma{10} for ZB20b, whereas there appears a clear coincidence of a \gls{VLN} in the solution paired with a relatively large amplitude in the filtered record.
\ijk{[@REZ: this is weird, we see a clear visual mismatch but the rolling RMSD isn't increasing by much here. Any ideas? Maybe because the RMSD is a cumulative sum, new deviations result in a smaller amplitude change as we get to the older parts of the record?]}\rez{not sure}
During this interval, the rolling \gls{RMSD} score for ZB20a steadily increases at a slower rate, and therefore results in a better overall match.
It appears that the strong negative excursions in our filtered data
at \qty{69.1}{\millionyearago} in ZB20a (oldest part of \ma{8})
and at \qty{70.4}{\millionyearago} in ZB20b (oldest part of \ma{11})
have a great deal of influence on the rolling \gls{RMSD} scores.
\ijk{[@REZ: method could be changed to compare to filters of LEC and SEC of solutions?]}

% FIGURE 3
\begin{figure}
  \centering
  \includegraphics[width=0.6\textwidth]{cumulative_rRMSD_all.png}
  \caption{\label{fig:cum-RMSD-all}
    \textbf{Best matches through time.}
    Square root cumulative sum squared difference (RCSD) scores of
    different astronomical solutions versus the Zumaia and Sopelana Maastrichtian composite record
    for \gls{d13C}, \gls{L*}, and \gls{MS}.
    The \(\text{RCSD}_{i}\) is calculated as \(\sqrt{\sum_{i=1}^{k}(e_{i} - f_{i})^{2}}\), where \(k\) is the total number of differences.
  }
\end{figure}



\section{Discussion and Conclusions}\label{sec:discussion}

This study aims to determine which, if any, of the currently available astronomical solutions best match the Maastrichtian proxy records of the Zumaia and Sopelana sites, in order to arrive at an astronomically calibrated time scale.
Ultimately, these results will allow us to anchor events in Earth's geologic past and to distinguish forcings and feedbacks in the climate system.


\ijk{[find correct location in discussion for this section]}
Unsurprisingly, these descriptions of the patterns in our filtered data and the \gls{RMSD} scores and cumulative \gls{RMSD} scores indicate that none of the astronomical solutions fully explain the patterns that we extracted from the proxy archives of the Zumaia and Sopelana field sections.
The reason is that even the most well-preserved astronomically forced signals from sediments do not reflect only astronomical forcing, but also a combination of the climate signal, the paleo-environment, the lithology that records the signal deposition, effects of diagenesis, the proxy used to reconstruct the climate signal, and the parameters and choices of our analysis (see \ref{sec:sensitivity}).
Furthermore, the number of slightly different valid astronomical solutions from small parameter variations is, due to the chaotic nature of the solar system, very large for the time interval at hand.
This means that a comparison against seven different calculations of astronomical forcing is not likely to represent an exact match.
However, both qualitatively (with the visual comparison of the \glspl{VLN}) and quantitatively (with the rolling- or full \gls{RMSD} scores), we can distinguish which of the solutions best matches the astronomical forcing that is detectable in the proxy archive.


During the Paleocene, solutions La10b and La10c showed a poor match with data from the Walvis Ridge, lacking a \gls{VLN} at around \qty{61}{\millionyearago}~\cite{ZeebeLourens2022EPSL}.
Our analysis shows that solutions ZB18a and ZB20c and to a lesser extent La10b are incompatible with the large amplitude in the data from Zumaia at \ma{5}.
Next, as we move to the older part of the record, the Sopelana data best matches ZB20a because ZB20b has a very long eccentricity node at \ma{10} (\appr\qty{69.7}{\millionyearago}), whereas the amplitude of the filtered data related to the short-eccentricity cycle is large there.
Therefore, the ZB20a solution is most compatible with the available Maastrichtian data and, at this point, is recommended for tuning up to \appr\qty{71}{\millionyearago}
(see \cref{tab:results} for a ranking of the astronomical solutions according to these criteria).

Our sensitivity analysis (\ref{sec:sensitivity}) shows that the methodology is quite sensitive to parameter choices, however.
This may be the result of the proxy archive itself, which, while showing a clear preference for the ZB20a solution when considering the \gls{L*} proxy, shows a slight preference for the ZB20d solution for the \gls{MS} proxy (\cref{fig:full-RMSD-all}) and no clear differences for the \gls{d13C} proxy.
However, as described above (see \cref{sec:data}), we argue that the color reflectance \gls{L*} record most directly reflects the orbital forcing and is hence the preferred proxy archive for inferences about the most accurate astronomical forcing scenario in this instance.
Different ways of pre-processing the data are discussed in \ref{sec:sensitivity}, and highlight the importance of careful selection of a set of parameters that results in the best matches, while not violating constraints from other data (e.g., sedimentation rates, the age of the \gls{KT}, etc.).
After careful consideration of these factors, we argue that it is possible to select an optimal set of parameter and filtering frequencies/widths that results in a reliable outcome to the extent that the datasets allow for.

One potential caveat to this conclusion is that section photographs from the field site of Zumaia may show indications for a large amplitude of the short eccentricity related cycle in \ma{2}~\cite{Batenburg2012,Dinares-Turell2013}.
A large amplitude of the short eccentricity cycle during \ma{2} would rule out the presence of a \gls{VLN}.
However, astronomical solutions ZB20a, ZB20b, ZB20d, and La10c all have a \gls{VLN} during this time interval.
The \gls{L*} and \gls{MS} records that we analyzed in this study did not show a large amplitude during this interval (\cref{fig:rolling-depth-age,fig:rolling-age-MS}).
It is unclear if the Zumaia \ce{CaCO3} record by \citeA{tenKateSprenger1993} showed large short eccentricity-related amplitudes for \ma{2}~\cite<reprocessed and filtered in>[]{Dinares-Turell2013}. % \ma{1--4} % also in \cite{Westerhold2008} figure 4.
The contemporaneous ln(\ce{Fe/Ca}) and \gls{MS} records from IODP Leg 342 Site U1403, spanning \ma{1--7}~\cite<Fig. 6 in>[]{Batenburg2018} do not indicate a large amplitude of the short eccentricity-related filter during \ma{2}.
Note that this study does show a large amplitude in the short eccentricity component of \ma{5}, which is in agreement with the present study~\cite{Batenburg2018}.
The \gls{MS} record from ODP Leg 208 Site 1267B, spanning \ma{1--6.5} shows a single large peak within \ma{2}, while the smoothed gray level of ODP Leg 122 Site 762C showed a small amplitude in \ma{2}~\cite<Fig. 4 in>[]{Husson2011}. \rez{[so what is the take-away from this paragraph?]}
\ijk{While proxy archives are in agreement about the small amplitude of the short eccentricity-related cycle during \ma{2}, photographs from the Zumaia and Sopelana sites may give visual indications of a large amplitude.}

% but note that they did not convert from MBSF to RMCD
% TODO: but where they put the Ma3 annotations is out of phase to how Husson2011 and Batenburg2012/2014 do it?
% below is too short to help us
%ODP 1262 \ma{1--2}~\cite{Westerhold2008} figure 4
%and \cite{Hilgen2010}. % Fig 2., Fig. 6 log(MS) + log(Fe)
% found via \cite{Dinares-Turell2014}, % log(Fe) and log(MS) Fig. 2
%1267 \ma{1} and \ma{2}~\cite{Westerhold2008} figure 4.
% this is all Husson2011:
%525A gray level bundle 1 (fig 4)
%525A gray level smoothed 6--9 (Fig. 3) hard to tell...
%762C gray level smoothed bundles 8 to 16 (Fig. 3) indicates potential hiatus within this bundle but also high change
%1258A MS \ma{8--14} (Fig. 3) no recovery in 10
%of~\citeA{Husson2011}.
%The key intervals of \ma{5}, \ma{8}, and \ma{10} are \ijk{XXX}.

\ijk{[we need a better bridge between these paragraphs]}
Future studies should aim to generate high resolution proxy records that have recorded the expression of both long and short eccentricity and connect to the oldest geologically constrained parts (\appr\qty{71}{\millionyearago}) of the astronomical solutions.
It may be wise to target low latitudes for this, since eccentricity-related cyclicity tends to dominate in the tropics~\cite<Fig. 7a in>[]{LaeppleLohmann2009}. % during the late Pleistocene (750 ka to present) in model
\ijk{[but of course the above depends on which proxy you're looking at... cut the sentence out?]}
We further suggest that future studies generate new astronomical solutions with similar initial settings to the ZB20a solution and, using appropriate parameter variations, explore the space beyond the horizon of predictability for a unique solar system solution.

%%

%  Numbered lines in equations:
%  To add line numbers to lines in equations,
%  \begin{linenomath*}
%  \begin{equation}
%  \end{equation}
%  \end{linenomath*}



%% Enter Figures and Tables near as possible to where they are first mentioned:
%
% DO NOT USE \psfrag or \subfigure commands.
%
% Figure captions go below the figure.
% Table titles go above tables;  other caption information
%  should be placed in last line of the table, using
% \multicolumn2l{$^a$ This is a table note.}
%
%----------------
% EXAMPLE FIGURES
%
% \begin{figure}
% \includegraphics{example.png}
% \caption{caption}
% \end{figure}
%
% Giving latex a width will help it to scale the figure properly. A simple trick is to use \textwidth. Try this if large figures run off the side of the page.
% \begin{figure}
% \noindent\includegraphics[width=\textwidth]{anothersample.png}
%\caption{caption}
%\label{pngfiguresample}
%\end{figure}
%
%
% If you get an error about an unknown bounding box, try specifying the width and height of the figure with the natwidth and natheight options. This is common when trying to add a PDF figure without pdflatex.
% \begin{figure}
% \noindent\includegraphics[natwidth=800px,natheight=600px]{samplefigure.pdf}
%\caption{caption}
%\label{pdffiguresample}
%\end{figure}
%
%
% PDFLatex does not seem to be able to process EPS figures. You may want to try the epstopdf package.
%

%
% ---------------
% EXAMPLE TABLE
% Please do NOT include vertical lines in tables
% if the paper is accepted, Wiley will replace vertical lines with white space
% the CLS file modifies table padding and vertical lines may not display well
%
 % \begin{table}
 % \caption{Time of the Transition Between Phase 1 and Phase 2$^{a}$}
 % \centering
 % \begin{tabular}{l c}
 % \hline
 %  Run  & Time (min)  \\
 % \hline
 %   $l1$  & 260   \\
 %   $l2$  & 300   \\
 %   $l3$  & 340   \\
 %   $h1$  & 270   \\
 %   $h2$  & 250   \\
 %   $h3$  & 380   \\
 %   $r1$  & 370   \\
 %   $r2$  & 390   \\
 % \hline
 % \multicolumn{2}{l}{$^{a}$Footnote text here.}
 % \end{tabular}
 % \end{table}

%% SIDEWAYS FIGURE and TABLE
% AGU prefers the use of {sidewaystable} over {landscapetable} as it causes fewer problems.
%
% \begin{sidewaysfigure}
% \includegraphics[width=20pc]{figsamp}
% \caption{caption here}
% \label{newfig}
% \end{sidewaysfigure}
%
%  \begin{sidewaystable}
%  \caption{Caption here}
% \label{tab:signif_gap_clos}
%  \begin{tabular}{ccc}
% one&two&three\\
% four&five&six
%  \end{tabular}
%  \end{sidewaystable}

%% If using numbered lines, please surround equations with \begin{linenomath*}...\end{linenomath*}
%\begin{linenomath*}
%\begin{equation}
%y|{f} \sim g(m, \sigma),
%\end{equation}
%\end{linenomath*}

%%% End of body of article

%%%%%%%%%%%%%%%%%%%%%%%%%%%%%%%%
%% Optional Appendix goes here
%
% The \appendix command resets counters and redefines section heads
%
% After typing \appendix
%
%\section{Here Is Appendix Title}
% will show
% A: Here Is Appendix Title
%
\appendix
% this template also turns the open research section into an appendix...


% We show some more combinations of \cref{fig:full-RMSD-all} in \cref{fig:full-RMSD-detrend}.
% We show an adaptation of \cref{fig:rolling-depth-age} for \gls{MS} (\cref{fig:rolling-age-MS}) and for \gls{d13C} (\cref{fig:rolling-age-d13C}).

% it actually didn't reset the figure numbers
\renewcommand\thefigure{A\arabic{figure}}
\setcounter{figure}{0}
% maybe some of this needs to be digital supplement rather than appendix?

\section{Sensitivity Analysis}\label{sec:sensitivity}

Our results were sensitive to the choices of the proxy archive we used, how we detrended the record to subtract non-periodic rapid shifts, and how we performed band-pass filtering.
% proxy
We show the filtered records based on \gls{MS} and \gls{d13C} against the astronomical solutions in the time domain in \cref{fig:rolling-age-MS,fig:rolling-age-d13C} and include their \gls{RMSD} scores in \cref{fig:full-RMSD-all}.
The \gls{d13C} archive from Zumaia shows a better match with ZB20c and La10b, but the differences are minor.
Visually, differences between the fits of the \gls{d13C} proxy to different astronomical solutions are hard to distinguish in the time domain (\cref{fig:rolling-age-d13C}).
The \gls{MS} proxy results in the best fits with ZB20d, ZB20c, and to a lesser extent La10c, ZB18a, and La10b, which can be partially explained because the \gls{MS} record shows a large amplitude in \ma{6}, whereas ZB20a, ZB20b, and La10c have a \gls{VLN} here.
\gls{MS} also shows a large amplitude during \ma{8}, which coincides with a \gls{VLN} in ZB20c, ZB18a, and to a lesser extent La10b.
Finally, \ma{10} and \ma{11} also show large amplitude, which rules out ZB20b and La10c.
See the methods for reasons to prefer the \gls{L*} proxy over the \gls{MS} and \gls{d13C} proxies for our purposes.

% boot
One approach to estimate the uncertainty of our \gls{RMSD} scores, is to perform bootstrapping.
The data were randomly resampled many (i.e., \num{e5}) times to the same size as the original data, but with repeated sampling (i.e., some measurements were included twice or more while others were excluded randomly).
Then, the \gls{RMSD} scores were calculated for each of these subsets.
This resulted in a distribution of \gls{RMSD} scores for each proxy against each astronomical solution (\cref{fig:full-RMSD-boot}).
The bootstrapped scores indicate that the differences between filtered \gls{d13C} record for the different astronomical solutions were likely not significant,
while the lower score for the \gls{L*} ZB20a solution
and the higher score for \gls{MS} solution ZB20b
likely were.
% commented out b/c calculating an average from a bivariate dist isn't fun
%For a simple average of the bootstrapped values across all proxies, the means overlapped at the \qty{68}{\percent} confidence level.
%After excluding the \gls{d13C} proxy, all astronomical solutions except for ZB20b (which performed worse because of the high \gls{RMSD} score for \gls{MS}) showed a similar overlap.
We hesitate to perform further statistical testing, however, since the remainder of this section will show how sensitive our \gls{RMSD} scores were to different parameter choices that resulted in a different alignment between the proxy record and the astronomical solutions.

% shift
Our results are relatively insensitive to the extent by which we shifted the record in order to arrive at the best fit.
There are upper bounds to the extent by which we can let the parameters of our matching algorithm fluctuate.
To illustrate: if we let the record shift by up to \qty{400}{\kiloyear} instead of the \qty{200}{\kiloyear} used throughout the manuscript, as one would expect, an offset of more than one long eccentricity cycle was often observed.
This is probably incorrect but happened to result in a better fit with those particular astronomical solutions.
Shifting the record by anywhere between \qtyrange{100}{300}{\kiloyear} typically resulted in an identical fit, however.

% tiepoint tweaks
The algorithm that allowed changing the depth of each long eccentricity minimum as identified in the field improved the \gls{RMSD} scores quite dramatically (\cref{fig:full-RMSD-tie} second panel).
Allowing each tie point depth to vary by greater distance than our default value of \qty{1.6}{\metre} (about \qty{10}{\percent} of the long eccentricity cycle) usually results in slightly better \gls{RMSD} scores (\cref{fig:full-RMSD-tie-error} first row).
A tie point error of up to \qty{3}{\metre} can result in an \gls{RMSD} score for the ZB18a solution that is similar to that of the ZB20a solution.
Visually, the presence of the \gls{VLN} in \ma{8} in ZB18a coincident with the large amplitude does not agree with the simple score, however.
As a quality control, we visualized the \gls{RMSD} scores for each tie point adjustment against depth to make sure that a local minimum was found and that the resulting filtered record did not show any very large changes in sedimentation rate (see \cref{fig:tiepoint-RMSD-optima} for an example).
% Changing this to 3, 4, or 5 meters usually results in a better fit that looks nicer and gets a better RMSD score
% 200 kyr + 3 m offset always gives visually similar results but with better RMSD scores.
%For ZB20a, our 1.6 m result goes from \num{1.02} to 3 m \num{0.986}.
%For ZB18a, our 1.6 m result goes from \num{1.08} to 3 m \num{0.989} and is then on-par with ZB20a.
%Setting it to 3, 4, or 5 m would move the K/T boundary up by 3 m, and the same for the minimum after Ma405-1 for ZB18a
%But could their 400 kyr minima idenification really be off by that much?]}
%\ijk{[talk about short ecc count offset between \cite{Batenburg2012} and \cite{Dinares-Turell2013}. Not sure if this should be discussion.]}
%\rez{It doesn't change the results, [illegible]}
% OK comment out for now.

% detrend
Different ways of detrending the proxy records to subtract non-cyclical patterns also lead to different results (\cref{fig:full-RMSD-detrend} middle row panel).
Overall, the different ways of detrending the record showed similar patterns, but the \gls{RMSD} scores for some solutions varied more between different ways of detrending than between different astronomical solutions (\cref{fig:full-RMSD-detrend}).
The large changes in lithology result in a very large amplitude of the short eccentricity filter in the simple detrending approach.
For most solutions, this simply results in a worse fit.
But for ZB20a and La10b, using a different detrending strategy makes our algorithm shift the long- eccentricity tie point depths enough that it is offset by about one short eccentricity cycle (\cref{fig:Lstar-detrend}), resulting in an even worse fit.
% Note that perhaps the \texttt{lin\_scl\_fine} and \texttt{lin\_scl\_med} detrending strategies result in overfitting for the \gls{d13C} record.

% filter
% window: gaussian, rectangular, taner
The bandpass filtering window used also affected the results (\cref{fig:full-RMSD-filter}, bottom row panel).
The bottom panel of \cref{fig:filter-windows} illustrates what the different bandpass filter windows look like in the frequency domain.
The taner filters go from the pointy \num{e3} (narrowest peak, widest shoulders, deep purple)
moving outwards from the peak to \num{e4}, \num{e6}, \num{e8}, (shades of blue)
\num{e10} (the preferred choice in the main text),
\num{e12} (shades of green), and \num{e100} (yellow),
approaching the rectangular filter (black).
The bandpass filters use the \texttt{lin\_scl\_med} detrend type,
the same filter boundary frequencies of \(\frac{1}{405}\) and \(\frac{1}{110}\)~\si[per-mode=power]{\per\kiloyear} \qty{\pm25}{\percent} as in the main text,
and tiepoint depth adjustments of up to \qty{1.6}{\metre} unless stated otherwise.

Note that the way \texttt{astrochron} parameterizes the Gaussian window (red in \cref{fig:filter-windows}) width is not directly comparable to the taner and rectangular filters.
Setting the Gaussian filter alpha to a value to match the width of the taners could work, but would cut off any lower/higher frequencies.
Therefore, we also show a reparametrized Gaussian filter (dark red),
where we multiply the fraction (\qty{25}{\percent})
by \num{2.1227} for a gaussian alpha of 3,
or by \(\frac{2.1227}{3}\times5\) for an alpha of 5.
The reparametrized Gaussian filter approximately intersects the point where the different taner filters cross each other and the rectangular filter and is comparable to a taner filter with a roll parameter of \num{e4} (see bottom panel in \cref{fig:full-RMSD-filter}).

The gaussian window performed worst for \gls{L*} (\cref{fig:full-RMSD-filter}, bottom middle panel),
followed by the rectangular filter,
while the taner filter with intermediate roll parameters resulted in the lowest \gls{RMSD} scores (performed better).
As one would expect, taner filters with a very high roll parameter (i.e., \num{e100}) performed similarly to the rectangular filter.
Taner roll parameters with a relatively smooth window (\(\texttt{roll} \ge 10^{8}\) and \(\le 10^{12}\)), made our algorithm shift the eccentricity construct by one short eccentricity cycle against the ZB20a solution, resulting in a much better fit (\cref{fig:filter-windows}, third panel).
% frac: window width, ±XX% of the target frequency
The same type of shift occurred when we changed the width of the filters (i.e., target frequency \qty{\pm30}{\percent} in stead of the default \qty{\pm25}{\percent}, not shown).
% \cref{fig:full-RMSD-taner} does illustrate different frac though

% comb: weight of long and short ecc
Applying a different weighting to the long- and short eccentricity filters, typically resulted in better \gls{RMSD} scores than the 1:1 combination (\cref{fig:full-RMSD-comb}).
However, these effects seemed relatively uniform across all astronomical solutions.
We therefore prefer using the 1:1 solution so that the relative filter amplitude is preserved from the original data.

Together, the sensitivity analyses show that it is worthwhile to explore the parameter space for both robustness and correctness.
Our results were sensitive to choices in data processing, but careful visual/qualitative comparison in combination with the quantitative \gls{RMSD} scores indicated which set of parameter choices was best able to distinguish between astronomical solutions.

% Appendix Figure: MS filters vs solutions
\begin{figure}
  \centering
  \includegraphics[width=\textwidth]{MS-vs-solutions.png}
  \caption{\label{fig:rolling-age-MS}
    \textbf{Maastrichtian Zumaia (purple) and Sopelana (orange) filtered \gls{MS} records tuned to astronomical solutions (black, first 7 panels).}
    % This uses short linear detrending to correct for changes in sediment composition (\cref{sec:detrend}).
    %We also show Zumaia above and below \qty{109.26}{\metre} to account for the change in the sedimentation rate and lithology.
    Bottom three panels show the record in the depth domain.
    Bottom panel shows the log of the two sites, adapted from \citeA{Batenburg2014}.
    Then \gls{MS} values are shown with different ways of detrending in coloured lines (\cref{sec:detrend}).
    The record after piecewise-linear detrendeding and normalization is shown after that.
    Vertical lines show long eccentricity minima as identified in the field,
    with adjustments of up to \qty{\pm1.6}{\metre} (horizontal segments)
    and how they were matched to each astronomical solution in the time domain to minimize their \gls{RMSD} (see \cref{sec:algorithm}).
    }
\end{figure}

% Appendix Figure: d13C filters vs solutions
\begin{figure}
  \centering
  \includegraphics[width=\textwidth]{d13C-vs-solutions.png}
  \caption{\label{fig:rolling-age-d13C}
    \textbf{Maastrichtian Zumaia (purple) inverted filtered \gls{d13C} records tuned to astronomical solutions (black, first 7 panels).}
    % This uses short linear detrending to correct for changes in sediment composition (\cref{sec:detrend}).
    %We also show Zumaia above and below \qty{109.26}{\metre} to account for the change in the sedimentation rate and lithology.
    Bottom three panels show the record in the depth domain.
    Bottom panel shows the log of the two sites, adapted from \citeA{Batenburg2014}.
    Then \gls{d13C} values are shown with different ways of detrending in coloured lines (\cref{sec:detrend}).
    The record after piecewise-linear detrendeding and normalization is shown after that.
    Vertical lines show long eccentricity minima as identified in the field,
    with adjustments of up to \qty{\pm1.6}{\metre} (horizontal segments)
    and how they were matched to each astronomical solution in the time domain to minimize their \gls{RMSD} (see \cref{sec:algorithm}).
    }
\end{figure}

% \ijk{removed this to reduce publication units. Figs are part of filtered records anyway}
% % Appendix Figure: detrend fits compared to raw data vs. depth
% \begin{figure}[htbp]
%   \centering
%   \includegraphics[width=.9\linewidth]{depth_detrend.png}
%   \caption{\label{fig:detrend}
%     \textbf{Zumaia (purple) and Sopelana (yellow) trend removal strategies.}
%     The raw data and the lines that were fit (other colours), which were subtracted from the record prior to filtering.
%     The piecewise linear fit that we subtract in the main manuscript corresponds to \texttt{lin\_pred\_med}.
%     % The detrended resultant record is callend \texttt{lin\_scl\_med}.
%   }
% \end{figure}

% Appendix Figure: effects of detrending in time domain
\begin{figure}
  \centering \includegraphics[width=\textwidth]{sol_SD_detrend.pdf}
  \caption{\label{fig:Lstar-detrend}
  \textbf{Effects of different detrending strategies in the time domain.}
    Illustration of how simple scaling of \gls{L*} with subsequent linear detrending (lower opacity lines)
    compares to the piecewize linear detrending strategy used in the main manuscript.
    Both use the same taner filter to extract long and short eccentricity components.
    }
\end{figure}

% Appendix Figure: Bootstrapped RMSD
% Appendix Figure: RMSD scores before and after tie-point adjustments
% Appendix Figure:  Filter Windows + RMSD scores for different taner filters
\begin{figure}
    \centering
    \includegraphics[width=0.9\textwidth]{appendix_fig_A4.pdf}
    % \includegraphics[width=0.6\textwidth]{full_RMSD_boot.png}
    % \centering \includegraphics[width=0.7\textwidth]{full_RMSD_tiepoint-error.pdf}
    % \centering \includegraphics[width=0.9\textwidth]{filter_windows.pdf}
    \caption{
        (a--c) \textbf{\gls{RMSD} scores of compilation of Zumaia and Sopelana proxy records against several orbital solutions.}
        (a)\label{fig:full-RMSD-boot} Shaded intervals represent the bootstrapped (\(N = 10^{5}\)) \qtyrange{5}{95}{\percent} confidence intervals of the \gls{RMSD} (in increments of \qty{5}{\percent}).
        (b)\label{fig:full-RMSD-tie-error} A wider range of tie-point depth adjustments for \gls{L*}.
        (c)\label{fig:full-RMSD-taner} Different taner filter roll parameters for \gls{L*}.
        Note that the taner roll parameter \gls{RMSD} scores differ from the main text because they use a slightly wider fraction (\qty{\pm30}{\percent} in stead of \qty{\pm25}{\percent} from the main text) to also illustrate the effects of changing the fraction.
        (d)\label{fig:filter-windows} Rectangular (black/gray square), gaussian (red, rescaled darkred) and taner (purple to blue to yellow) bandpass filters in the frequency domain.
        We show MTM-spectral peaks of the astronomical solutions analyzed in this study in black for reference.
    }
\end{figure}

% Appendix Figure: RMSD scores before and after tie-point adjustments
% Appendix Figure: RMSD scores for different ways of detrending
% Appendix Figure:  RMSD scores for different filter strategies
\begin{figure}
  \centering \includegraphics[width=\textwidth]{appendix_fig_A5.pdf}
  % \centering \includegraphics[width=\textwidth]{sol_SD_tie.png}
  % \centering \includegraphics[width=\textwidth]{full_RMSD_detrend_comparison.png}
  % \centering \includegraphics[width=\textwidth]{full_RMSD_filter_comparison.png}
  \caption{
    \textbf{Illustration of how RMSD scores change with different algorithm choices.}
    (top row)\label{fig:full-RMSD-tie}
    Zumaia (not the composite!) proxy records (column panels)
    before (dashes) and after (solid) we allow tie-point depths identified in the field to vary by \qty{\pm1.6}{\metre}.
    % This uses the taner filter from the main text on the piecewise linearly detrended record.
    (middle row)\label{fig:full-RMSD-detrend} Sensitivity analysis of various ways of detrending the composite record.
    This uses optimized tie-point depths and limits the results to a rectangular filter.
    We show main-text taner filters for \texttt{lin\_scl\_med} with different colours for each proxy for context.
    (bottom row)\label{fig:full-RMSD-filter} Rectangular (gray squares), rescaled gaussian (dark red triangles, see \ref{sec:sensitivity}) and taner (black circles) bandpass filters.
    The taner filter has the main-text roll parameter of \num{e10}.
    See the bottom panel of \cref{fig:filter-windows} for illustrations of filter windows and alternative taner roll parameters.
    }
\end{figure}


% Appendix Figure: RMSD scores with different tie-point adjustments
\begin{figure}
  \centering \includegraphics[width=\textwidth]{tiepoint_RMSD_optima.png}
  \caption{\label{fig:tiepoint-RMSD-optima}
    \textbf{Illustration of how RMSD scores change by performing tie-point depth optimization.}
    The \gls{RMSD} scores as they evolve when each tie point is iteratively shifted for the filtered \gls{L*} record of Zumaia (top panel) against ZB18a.
    We show the filtered record (black) versus ZB18a (yellow).
    This uses the taner filter from the main text on the piecewise linearly detrended record.
    }
\end{figure}


% IJK: commented this out b/c it was a messy figure and the above two figures should show the narrowing of the filters as well.
% % Appendix Figure:  RMSD scores for different filter strategies
% \begin{figure}
%   \centering \includegraphics[width=0.7\textwidth]{full_RMSD_frac.pdf}
%   \caption{\label{fig:full-RMSD-frac}
%     \textbf{Sensitivity analysis of different bandpass filter widths.}
%     The frac parameter is the fraction that determines the upper and lower boundaries of the target frequency.
%     %% For example, targeting the \(\frac{1}{405}}\)\si{\per\kiloyear} frequency \(\pm\texttt{frac}\times\texttt{freq}\).
%     %Note that gaussian window also narrows filter width (see \cref{fig:filter-windows}).
%     This uses \gls{L*} proxy with the \texttt{lin\_scl\_med} detrend type and tie-point depth optimization.
%     The taner filter has a roll parameter of \num{e10}.
%     }
% \end{figure}



% Appendix Figure:  RMSD scores for different weights of long- and short ecc
\begin{figure}
  \centering \includegraphics[width=\textwidth]{full_RMSD_comb_comparison.png}
  \caption{\label{fig:full-RMSD-comb}
    \textbf{Sensitivity analysis of different weights for long- and short eccentricity.}
    Different astronomical solutions (colours) show very similar patterns,
    where most combinations of the short and long eccentricity filters other than 1:1 result in lower \gls{RMSD} scores.
    Still we use the 1:1 weights in the main text because they preserve the relative filter amplitudes from the data.
    This uses the \texttt{lin\_scl\_med} detrend type and tie-point depth optimization.
    The taner filters have the main-text roll parameter of \num{e10} and a filter fraction of \num{0.25}.
    }
\end{figure}


%%%%%%%%%%%%%%%%%%%%%%%%%%%%%%%%%%%%%%%%%%%%%%%%%%%%%%%%%%%%%%%%%%%%%%%%%%
%                   some spectral analysis figures                       %
%%%%%%%%%%%%%%%%%%%%%%%%%%%%%%%%%%%%%%%%%%%%%%%%%%%%%%%%%%%%%%%%%%%%%%%%%%

% \begin{figure}
%   \centering \includegraphics[width=\textwidth]{Zumaia-Sopelana_mtm_raw.pdf}
%   \caption{\label{fig:spectral-depth}
%     \textbf{Spectral analysis in the depth domain.}
%     % Do I need refs for all of these?
%     \ijk{In the end probably show analysis only in time-domain?}
%     BT = Blackman-Tukey,
%     FFT = Fast Fourier Transform,
%     LOWSPEC = Robust Locally-Weighted Regression Spectral Background Estimation \cite{Meyers2012},
%     LS = Lomb-Scargle,
%     MTLS = Multi-taper Averaged Lomb-Scargle periodogram of (un)evenly
% spaced data \cite{Springford2020},
%     MTM = Multitaper method \cite{Thomson1982}.
%     Shaded intervals for the MTM and LOWSPEC indicate AR1 fit and AR1-power and LOWSPEC fit and power confidence levels.
%   }
% \end{figure}
%
% \begin{figure}
%   \centering \includegraphics[width=\textwidth]{Zumaia-Sopelana_mtm.pdf}
%   \caption{\label{fig:spectral-depth}
%     \textbf{Spectral analysis in the depth domain.}
%     % Do I need refs for all of these?
%     \ijk{This is the same as above but after linear detrending with \texttt{lin\_scl\_fine}.}
%     BT = Blackman-Tukey,
%     FFT = Fast Fourier Transform,
%     LOWSPEC = Robust Locally-Weighted Regression Spectral Background Estimation \cite{Meyers2012},
%     LS = Lomb-Scargle,
%     MTLS = Multi-taper Averaged Lomb-Scargle periodogram of (un)evenly
% spaced data \cite{Springford2020},
%     MTM = Multitaper method \cite{Thomson1982}.
%     Shaded intervals for the MTM and LOWSPEC indicate AR1 fit and AR1-power and LOWSPEC fit and power confidence levels.
%   }
% \end{figure}
%
%
% \begin{figure}
%   \centering \includegraphics[width=\textwidth]{Zumaia_Sopelana_spectra_filters_raw.pdf}
%   \caption{\label{fig:spectral-age-raw}
%     \textbf{Spectral analysis in the time domain.}
%     % Do I need refs for all of these?
%     \ijk{This is raw values, only linear detrend}
%     % BT = Blackman-Tukey,
%     FFT = Fast Fourier Transform,
%     LOWSPEC = Robust Locally-Weighted Regression Spectral Background Estimation \cite{Meyers2012},
%     % LS = Lomb-Scargle,
%     MTLS = Multi-taper Averaged Lomb-Scargle periodogram of (un)evenly
% spaced data \cite{Springford2020},
%     MTM = Multitaper method \cite{Thomson1982}.
%     Shaded intervals for the MTM and LOWSPEC indicate AR1 fit and AR1-power and LOWSPEC fit and power confidence levels.
%   }
% \end{figure}
%
%
% \begin{figure}
%   \centering
%   \includegraphics[width=1.2\textwidth]{Zumaia_MS_1-1_solutions_simple_with_log.pdf}
%   \caption{\label{fig:rolling-age-MS}
%     Same as \cref{fig:rolling-depth-age} but for \gls{MS}.}
% \end{figure}

% \begin{figure}
%   \centering
%   \includegraphics[width=0.9\textwidth]{Zumaia_d13C_1-1_solutions_simple_with_log.pdf}
%   \caption{\label{fig:rolling-age-d13C}
%     Same as \cref{fig:rolling-depth-age} but for \gls{d13C}.}
% \end{figure}


%%%%%%%%%%%%%%%%%%%%%%%%%%%%%%%%%%%%%%%%%%%%%%%%%%%%%%%%%%%%%%%%
%
% Optional Glossary, Notation or Acronym section goes here:
%
%%%%%%%%%%%%%%
% Glossary is only allowed in Reviews of Geophysics
%  \begin{glossary}
%  \term{Term}
%   Term Definition here
%  \term{Term}
%   Term Definition here
%  \term{Term}
%   Term Definition here
%  \end{glossary}

%
%%%%%%%%%%%%%%
% Acronyms
%   \begin{acronyms}
%   \acro{Acronym}
%   Definition here
%   \acro{EMOS}
%   Ensemble model output statistics
%   \acro{ECMWF}
%   Centre for Medium-Range Weather Forecasts
%   \end{acronyms}

%
%%%%%%%%%%%%%%
% Notation
%   \begin{notation}
%   \notation{$a+b$} Notation Definition here
%   \notation{$e=mc^2$}
%   Equation in German-born physicist Albert Einstein's theory of special
%  relativity that showed that the increased relativistic mass ($m$) of a
%  body comes from the energy of motion of the body—that is, its kinetic
%  energy ($E$)—divided by the speed of light squared ($c^2$).
%   \end{notation}



\section*{Open Research}

\Gls{MS}, \gls{L*}, and \gls{d13C} data used in this study are from \citeA{Batenburg2012,Batenburg2012}.

Analysis was performed using the R programming language~\cite{RCoreTeam2024} and made use of \texttt{astrochron} \citeA{Meyers2014} and the \texttt{tidyverse} \citeA{Wickham2019}.
The algorithm will be made available in the R package \texttt{AstronomicalSolutions} \url{https://github.com/japhir/AstronomicalSolutions}.
\ijk{[come up with nice package name (working title: \texttt{AstronomicalSolutions} so it's broad enough for future additions) and host on github/ archive on Zenodo.]}
\rez{[what is the content of R package AstronomicalSolutions?]}
\ijk{[the R-package would have the R implementation of the filter/matching
  algorithm, as well as some helper functions for bandpass/taner filtering,
  spectral analysis etc. that are wrappers for astrochron, but make the output
  more consistent for easy plotting etc.]}
% AGU requires an Availability Statement for the underlying data needed to understand, evaluate, and build upon the reported research at the time of peer review and publication.

% Authors should include an Availability Statement for the software that has a significant impact on the research. Details and templates are in the Availability Statement section of the Data and Software for Authors Guidance: \url{https://www.agu.org/Publish-with-AGU/Publish/Author-Resources/Data-and-Software-for-Authors#availability}

% It is important to cite individual datasets in this section and, and they must be included in your bibliography. Please use the type field in your bibtex file to specify the type of data cited. Some options include Dataset, Software, Collection, ComputationalNotebook. Ex:
% \\
% \begin{verbatim}

% @misc{https://doi.org/10.7283/633e-1497,
%   doi = {10.7283/633E-1497},
%   url = {https://www.unavco.org/data/doi/10.7283/633E-1497},
%   author = {de Zeeuw-van Dalfsen, Elske and Sleeman, Reinoud},
%   title = {KNMI Dutch Antilles GPS Network - SAB1-St_Johns_Saba_NA P.S.},
%   publisher = {UNAVCO, Inc.},
%   year = {2019},
%   type = {dataset}
% }

% \end{verbatim}

% For physical samples, use the IGSN persistent identifier, see the International Geo Sample Numbers section:
% \url{https://www.agu.org/Publish-with-AGU/Publish/Author-Resources/Data-and-Software-for-Authors#IGSN}
%%%%%%%%%%%%%%%%%%%%%%%%%%%%%%%%%%%%%%%%%%%%%%%

\acknowledgments
% This section is optional. Include any Acknowledgments here.
% The acknowledgments should list:\\
% All funding sources related to this work from all authors\\
% Any real or perceived financial conflicts of interests for any author\\
% Other affiliations for any author that may be perceived as having a conflict of interest with respect to the results of this paper.\\
% It is also the appropriate place to thank colleagues and other contributors. AGU does not normally allow dedications.

This work was supported by the Heising-Simons Foundation (\#2021-2800), under the CycloAstro
Cohort project 3 and U.S. NSF grants OCE20-01022, OCE20-34660 to R.E.Z.

%% ------------------------------------------------------------------------ %%
%% References and Citations

%%%%%%%%%%%%%%%%%%%%%%%%%%%%%%%%%%%%%%%%%%%%%%%
%
% \bibliography{<name of your .bib file>} don't specify the file extension
%
% don't specify bibliographystyle

% In the References section, cite the data/software described in the Availability Statement (this includes primary and processed data used for your research). For details on data/software citation as well as examples, see the Data & Software Citation section of the Data & Software for Authors guidance
% https://www.agu.org/Publish-with-AGU/Publish/Author-Resources/Data-and-Software-for-Authors#citation

%%%%%%%%%%%%%%%%%%%%%%%%%%%%%%%%%%%%%%%%%%%%%%%

\bibliography{references}


%Reference citation instructions and examples:
%
% Please use ONLY \cite and \citeA for reference citations.
% \cite for parenthetical references
% ...as shown in recent studies (Simpson et al., 2019)
% \citeA for in-text citations
% ...Simpson et al. (2019) have shown...
%
%
%...as shown by \citeA{jskilby}.
%...as shown by \citeA{lewin76}, \citeA{carson86}, \citeA{bartoldy02}, and \citeA{rinaldi03}.
%...has been shown \cite{jskilbye}.
%...has been shown \cite{lewin76,carson86,bartoldy02,rinaldi03}.
%... \cite <i.e.>[]{lewin76,carson86,bartoldy02,rinaldi03}.
%...has been shown by \cite <e.g.,>[and others]{lewin76}.
%
% apacite uses < > for prenotes and [ ] for postnotes
% DO NOT use other cite commands (e.g., \citet, \citep, \citeyear, \citealp, etc.).
% \nocite is okay to use to add references from your Supporting Information
%


\end{document}



% More Information and Advice:

%% ------------------------------------------------------------------------ %%
%
%  SECTION HEADS
%
%% ------------------------------------------------------------------------ %%

% Capitalize the first letter of each word (except for
% prepositions, conjunctions, and articles that are
% three or fewer letters).

% AGU follows standard outline style; therefore, there cannot be a section 1 without
% a section 2, or a section 2.3.1 without a section 2.3.2.
% Please make sure your section numbers are balanced.
% ---------------
% Level 1 head
%
% Use the \section{} command to identify level 1 heads;
% type the appropriate head wording between the curly
% brackets, as shown below.
%
%An example:
%\section{Level 1 Head: Introduction}
%
% ---------------
% Level 2 head
%
% Use the \subsection{} command to identify level 2 heads.
%An example:
%\subsection{Level 2 Head}
%
% ---------------
% Level 3 head
%
% Use the \subsubsection{} command to identify level 3 heads
%An example:
%\subsubsection{Level 3 Head}
%
%---------------
% Level 4 head
%
% Use the \subsubsubsection{} command to identify level 3 heads
% An example:
%\subsubsubsection{Level 4 Head} An example.
%
%% ------------------------------------------------------------------------ %%
%
%  IN-TEXT LISTS
%
%% ------------------------------------------------------------------------ %%
%
% Do not use bulleted lists; enumerated lists are okay.
% \begin{enumerate}
% \item
% \item
% \item
% \end{enumerate}
%
%% ------------------------------------------------------------------------ %%
%
%  EQUATIONS
%
%% ------------------------------------------------------------------------ %%

% Single-line equations are centered.
% Equation arrays will appear left-aligned.

% Math coded inside display math mode \[ ...\]
%  will not be numbered, e.g.,:
%  \[ x^2=y^2 + z^2\]

%  Math coded inside \begin{equation} and \end{equation} will
%  be automatically numbered, e.g.,:
%  \begin{equation}
%  x^2=y^2 + z^2
%  \end{equation}


% % To create multiline equations, use the
% % \begin{eqnarray} and \end{eqnarray} environment
% % as demonstrated below.
% \begin{eqnarray}
%   x_{1} & = & (x - x_{0}) \cos \Theta \nonumber \\
%         && + (y - y_{0}) \sin \Theta  \nonumber \\
%   y_{1} & = & -(x - x_{0}) \sin \Theta \nonumber \\
%         && + (y - y_{0}) \cos \Theta.
% \end{eqnarray}

%If you don't want an equation number, use the star form:
%\begin{eqnarray*}...\end{eqnarray*}

% Break each line at a sign of operation
% (+, -, etc.) if possible, with the sign of operation
% on the new line.

% Indent second and subsequent lines to align with
% the first character following the equal sign on the
% first line.

% Use an \hspace{} command to insert horizontal space
% into your equation if necessary. Place an appropriate
% unit of measure between the curly braces, e.g.
% \hspace{1in}; you may have to experiment to achieve
% the correct amount of space.


%% ------------------------------------------------------------------------ %%
%
%  EQUATION NUMBERING: COUNTER
%
%% ------------------------------------------------------------------------ %%

% You may change equation numbering by resetting
% the equation counter or by explicitly numbering
% an equation.

% To explicitly number an equation, type \eqnum{}
% (with the desired number between the brackets)
% after the \begin{equation} or \begin{eqnarray}
% command.  The \eqnum{} command will affect only
% the equation it appears with; LaTeX will number
% any equations appearing later in the manuscript
% according to the equation counter.
%

% If you have a multiline equation that needs only
% one equation number, use a \nonumber command in
% front of the double backslashes (\\) as shown in
% the multiline equation above.

% If you are using line numbers, remember to surround
% equations with \begin{linenomath*}...\end{linenomath*}

%  To add line numbers to lines in equations:
%  \begin{linenomath*}
%  \begin{equation}
%  \end{equation}
%  \end{linenomath*}
